\documentclass[aspectratio=169]{../latex_main/tntbeamer}  % you can pass all options of the beamer class, e.g., 'handout' or 'aspectratio=43'
\input{../latex_main/preamble}

\title[Introduction]{DS: SQL}
\subtitle{Database Schemas}

\graphicspath{ {./figure/} }
%\institute{}


\begin{document}
	
	\maketitle
	\begin{frame}{Relational DBMS Terminology}
	    
	    \begin{columns}
	        \begin{column}{.75\textwidth}
	        \vspace{.4cm}\\
	        In a relational database, each table is called a relation.\\
            Each row of relation is called a record or tuple. Rows do not have names.\\
            Each column of a relation is called an attribute or field.

	                \begin{itemize}
	                    \item Attributes have names (e.g. temperature, city, legs).
	                    \item Attributes have data types (e.g. INTEGER, CHAR(20)).
	                    \item Attributes may also have constraints (e.g. must be non-negative).
	                    \item Attributes may be marked as primary or foreign keys.
	                    \begin{itemize}
	                        \item Primary key must be unique. Example on next slide.
	                        \item Foreign key means that an attribute is some other table’s primary key.
	                        \begin{itemize}
	                            \item Explicitly shows how tables are linked.
	                        \end{itemize}
	                    \end{itemize}
	                \end{itemize}
	        \end{column}
	        
	        
	        
	        \begin{column}{.25\textwidth}
	                \begin{center}
	                    \includegraphics[scale=.35]{Bild1}
	                \end{center}
	        \end{column}
	    \end{columns}
	\end{frame}
	
	
	
	\begin{frame}{Example Relation Schema}
	    \begin{columns}
	        \begin{column}{.6\textwidth}
	             \begin{center}
	                \includegraphics[scale=.35]{Bild2}
	             \end{center}
	             Given the table animal above, it is impossible to:
	             \begin{itemize}
	                 \item Insert a record with the same name as another.
	                 \item Insert a record with a negative value for legs or weight.
	                 \item Insert a record with a non-integer legs or weight.
	             \end{itemize}
	        \end{column}
	        
	        
	        
	        \begin{column}{.4\textwidth}
	                \begin{center}
	                    \includegraphics[scale=.35]{Bild3}
	                \end{center}
	        \end{column}
	    \end{columns}
	\end{frame}
	
	
	
	\begin{frame}[c]{Database Schema}
	    A relational database is a set of relations.
	    \begin{itemize}
	        \item Set of the schemas of those relations is called the database schema.
	        \item If the database schema includes foreign key relations, schema effectively includes a description how the tables refer to one another.
	    \end{itemize}
	\end{frame}
	
	
	\begin{frame}{Example Database Schema}
	    \includegraphics[scale=.4]{Bild4}
	\end{frame}
	
	
	
	
	\begin{frame}{Database Implementations}
	    We can query relational databases with SQL, but there are many implementations of SQL. 
	    \begin{itemize}
	        \item And many other database implementations that are not SQL based / relational.
	    \end{itemize}
	    \includegraphics[scale=.4]{Bild5}\\
	    \vspace{.9cm}
	    \url{https://db-engines.com/en/ranking}
	\end{frame}
\end{document}