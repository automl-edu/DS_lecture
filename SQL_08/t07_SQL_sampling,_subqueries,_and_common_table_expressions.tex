\documentclass[aspectratio=169]{../latex_main/tntbeamer}  % you can pass all options of the beamer class, e.g., 'handout' or 'aspectratio=43'
\input{../latex_main/preamble}

\title[Introduction]{DS: SQL}
\subtitle{SQL Sampling, Subqueries, and Common Table Expressions}

\graphicspath{ {./figure/} }
%\institute{}


\begin{document}
	
	\maketitle
	\begin{frame}{Sampling with LIMIT?}
	    \includegraphics[scale=.35]{Bild30}
	\end{frame}
	
	
	\begin{frame}{Sampling with LIMIT?}
	    \includegraphics[scale=.35]{Bild31}
	\end{frame}
	
	
	\begin{frame}{Random Sampling}
	    The random sampling methods available depend on the database engine. Suppose we want to draw a SRS from an SQL table.\\
	    \bigskip
	    One common approach (with SQLite):
	    \begin{itemize}
	        \item SELECT * FROM action\_movie ORDER BY RANDOM() LIMIT 3
	    \end{itemize}
	    \\\bigskip
	    May seem inefficient to order the entire table by some random values, then to only select 3.
	    \begin{itemize}
	        \item Query optimization under the hood will make this much more efficient.
	        \item Reminder: SQL is a declarative language. You say “what”, not “how”.
	    \end{itemize}
	\end{frame}
	
	
	\begin{frame}{Random Sampling}
	    Suppose we want to pick 3 random years.\\
	    \includegraphics[scale=.5]{Bild32}
	\end{frame}
	
	
	\begin{frame}{Random Sampling}
	    Suppose we want to pick 3 random years.\\
	    \includegraphics[scale=.45]{Bild33}
	\end{frame}
	
	
	
	\begin{frame}{Random Sampling}
	    Suppose we want to pick 3 random years.\\

	    \includegraphics[scale=.45]{Bild34}
	\end{frame}
	
	
	
	\begin{frame}{Subqueries}
	    A query within another query can be used to create a temporary table.
	    \begin{itemize}
	        \item In a FROM clause: Describe a table instead of naming it.
	    \end{itemize}
	    E.g., join table u with a simple random sample from table t:\\
	    \hspace{.5cm} \includegraphics[scale=.4]{Bild35}
	    \bigskip
	    \begin{itemize}
	        \item In a WHERE clause: Describe a one-column table instead of a list; used with IN.
	    \end{itemize}
	    E.g., select rows in a top-3 most popular month:\\
	    \hspace{.5cm}
	    \includegraphics[scale=.4]{Bild36}
	\end{frame}
	
	
	\begin{frame}{Common Table Expressions}
	    \begin{itemize}
	        \item A Common Table Expressions (CTE) allows for the creation of “temporary tables” to help organize complex queries.
	        \begin{itemize}
	            \item CTEs can help make complex queries more readable.
	            \item CTEs can be used instead of subqueries.
	        \end{itemize}
	    \end{itemize}
	    \bigskip
	    \includegraphics[scale=.4]{Bild37}
	\end{frame}
	
\end{document}