\documentclass[aspectratio=169]{../latex_main/tntbeamer}  % you can pass all options of the beamer class, e.g., 'handout' or 'aspectratio=43'
\input{../latex_main/preamble}

\title[Introduction]{DS: SQL}
\subtitle{Databases}

\graphicspath{ {./figure/} }
%\institute{}


\begin{document}
	
	\maketitle
	\begin{frame}{Brief Databases Overview}
	    \begin{itemize}
	        \item A database is an organized collection of data.
	        \item A database management system (DBMS) is a software system that stores, manages, and facilitates access to one or more databases. 
	        \item Why use DBMSes? 
	        \begin{itemize}
	            \item Our data might not be stored in a simple-to-read format such as a CSV (comma-separated values) file.
	            \item Think of a CSV like an Excel sheet or a sheet in Google sheets.
	            \item In Data 8, most of the data were given to you in CSV files, but that will not always be the case in the real world. 
	            \item If our data are stored in a DBMS, we must use languages such as Structured Query Language  (SQL) to query for our data.
	        \end{itemize}
	    \end{itemize}
	\end{frame}   
	    
	    
	\begin{frame}{Advantages of DBMS over CSV (or similar)}
	        Data Storage:
	        \begin{itemize}
	            \item Reliable storage to survive system crashes and disk failures.
	            \item Optimize to compute on data that does not fit in memory.
	            \item Special data structures to improve performance.
	        \end{itemize}
	        Data Management:
	        \begin{itemize}
	            \item Configure how data is logically organized and who has access.
	            \item Can enforce guarantees on the data (e.g. non-negative bank account balance).
	            \begin{itemize}
	                \item Can be used to prevent data anomalies.
	                \item Ensures safe concurrent operations on data.
	            \end{itemize}
	        \end{itemize}
	\end{frame}
	
	
	
	
	
	
\end{document}