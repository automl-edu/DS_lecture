\documentclass[aspectratio=169]{../latex_main/tntbeamer}  % you can pass all options of the beamer class, e.g., 'handout' or 'aspectratio=43'
\input{../latex_main/preamble}

\title[Introduction]{DS: SQL}
\subtitle{SQL Predicates and Casting}

\graphicspath{ {./figure/} }
%\institute{}


\begin{document}
	
	\maketitle
	\begin{frame}{SQL Predicates}
	    In addition to numerical comparisons (=, <, >), SQL has built-in predicates.
	    \begin{itemize}
	        \item Example: The IN operator tests whether a value is in a list.
	        \begin{itemize}
	            \item E.g., select rows whose month is either January, March, or May:
	            \includegraphics[scale=.4]{Bild27}
	        \end{itemize}
	    \end{itemize}
	    
	    
	    \begin{itemize}
	        \item Example: The LIKE operator tests whether a string matches a pattern (similar to a regex, but much simpler syntax):
	        \begin{itemize}
	            \item E.g. select rows where the time string is on the hour, such as 8:00 or 12:00 pm.
	            \includegraphics[scale=.4]{Bild28}
	        \end{itemize}
	    \end{itemize}
	\end{frame}
	
	
	\begin{frame}{SQL Casting}
	    Can use CAST to convert fields from one type to another:
	    \begin{itemize}
	        \item Handy when combined with WHERE:
	    \end{itemize}
	    \vspace{1cm}
	    \includegraphics[scale=.45]{Bild29}
	\end{frame}
	
	
	
	
\end{document}