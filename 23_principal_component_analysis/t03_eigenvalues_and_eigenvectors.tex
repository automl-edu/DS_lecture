\documentclass[aspectratio=169]{../latex_main/tntbeamer}  % you can pass all options of the beamer class, e.g., 'handout' or 'aspectratio=43'
\input{../latex_main/preamble}

\title[Introduction]{DS: Principal Component Analysis}
\subtitle{Eigenvalues and Eigenvectors}

\graphicspath{ {./figure/} }
%\institute{}


\begin{document}
	
	\maketitle
	\begin{frame}{Eigenvalues and Eigenvectors}
	    An eigenvector z is a special vector of a given matrix A, that has the following property:
	    \begin{equation*}
	        Az = \lambda z
	    \end{equation*}
	    Here, $\lambda$ is a scalar, and it is the eigenvalue corresponding to z. So this means that when a matrix is multiplied by one of its eigenvectors, the result is a scaled copy of the eigenvector.\\
	    A matrix may have multiple eigenvectors, but most vectors are not eigenvectors.\\
	    \bigskip
	    
        We can refer to the eigenvalues and eigenvectors of a matrix like so. Remember, an eigenvector is paired with a specific eigenvalue, and vice versa.
        \begin{equation*}
            (\lambda_1,z_1),(\lambda_2,z_2),...
        \end{equation*}
        

	\end{frame}
	
	
	\begin{frame}{An Example}
	    Here is an example matrix A.
	    \begin{equation*}
	        A = \left[\begin{array}{cc}
	           0  &  -1\\
	           2  &   3
	        \end{array}\right]
	         \end{equation*}
	        It has two eigenvalue and eigenvector pairs:
	        \begin{equation*}
	            \lambda_1 = 2, z_1 = \left[\begin{array}{c}
	           1\\
	           -2
	        \end{array}\right]
	        \hspace{1cm} \lambda_2 = 1, z_2 = \left[\begin{array}{c}
	           1\\
	           -1
	        \end{array}\right]
	        \end{equation*}
	        We can verify that the relationship		$Az = \lambda z$	   holds:
	        \begin{equation*}
	            Az_1 = \left[\begin{array}{cc}
	           0  &  -1\\
	           2  &   3
	        \end{array}\right] \left[\begin{array}{c}
	           1\\
	           -2
	        \end{array}\right] = \left[\begin{array}{c}
	           2\\
	           -4
	        \end{array}\right] = 2 \left[\begin{array}{c}
	           1\\
	           -2
	        \end{array}\right] = \lambda_1 z_1
	        \end{equation*}
	\end{frame}
	
	
	\begin{frame}{Eigenvectors of the Covariance Matrix}
	    \begin{columns}
	        \begin{column}{.5\textwidth}
	                We can determine the eigenvectors and eigenvalues of our covariance matrix.
	                \begin{align*}
	                    \lambda_1 = .051, z_1 = \left[\begin{array}{c}
	                             0.573\\
	                             0.820
	                    \end{array}\right]\\
	                    \lambda_2 = .007, z_2 = \left[\begin{array}{c}
	                             -0.820\\
	                             0.573
	                    \end{array}\right]
	                \end{align*}
	                The plot on the right shows these eigenvectors, relatively scaled by their corresponding eigenvalues.
	        \end{column}
	        
	        
	        \begin{column}{.5\textwidth}
	                \begin{figure}
	                    \centering
	                    \includegraphics[scale=.5]{Bild3}
	                \end{figure}
	        \end{column}
	    \end{columns}
	\end{frame}
\end{document}