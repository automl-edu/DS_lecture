\documentclass[aspectratio=169]{../latex_main/tntbeamer}  % you can pass all options of the beamer class, e.g., 'handout' or 'aspectratio=43'
\input{../latex_main/preamble}

\title[Introduction]{DS: Principal Component Analysis}
\subtitle{Singular Value Decomposition}

\graphicspath{ {./figure/} }
%\institute{}


\begin{document}
	
	\maketitle
	\begin{frame}{Singular Value Decomposition}
	    We can decompose a (centered, design) matrix X into the following three matrices:
	    \begin{equation*}
	        X = U\Sigma V^T = \left[\begin{array}{cccc}
	        \vdots & \vdots & \dots &\vdots \\
	       u_1 & u_2 & \dots & u_d \\
	              \vdots &     \vdots   &   \dots  &  \vdots 
	    \end{array}\right]
	    \left[\begin{array}{cccc}
	        \sigma_1 & 0 & \dots & 0 \\
	        0 & \sigma_2 & \dots & 0 \\
	              \vdots &     \vdots   &   \ddots  &  \vdots  \\
	       0 & 0 & \dots &\sigma_d
	    \end{array}\right]
	    \left[\begin{array}{ccc}
	       \dots & v^T_1 & \dots  \\
	       \dots & v^T_2 & \dots  \\
	        &     \vdots   &  \\
	       \dots & v^T_d & \dots  
	    \end{array}\right]
	    \end{equation*}
	    U has n rows and d columns—it has the same shape as X.\\
	    $\Sigma$ has d rows and d columns. Every off-diagonal entry is 0.\\
	    $V^T$ has d rows and d columns.
	    
	\end{frame}
	
	\begin{frame}{$\Sigma$ and $V^T$}
	    \begin{columns}
	        \begin{column}{.5\textwidth}
	               The diagonal entries of $\Sigma$ (written as σi) are called the singular values of X. These are the square roots of the eigenvalues of $X^TX$.\\
	               The singular values are sorted in descending order.
	               \begin{equation*}
	               \left[\begin{array}{cccc}
	                        \sigma_1 & 0 & \dots & 0 \\
	                        0 & \sigma_2 & \dots & 0 \\
	                        \vdots &     \vdots   &   \ddots  &  \vdots  \\
                            0 & 0 & \dots &\sigma_d
	                \end{array}\right]
	                \end{equation*}
	        \end{column}
	        
	        \begin{column}{.5\textwidth}
	               Each row of $V^T$ (or alternatively, each column of V), is an eigenvector of $X^TX$. The eigenvalue for $V_1$ is $\sigma_1^2$.\\
	               The rows of $V^T$ are orthogonal to each other.
                   \begin{equation*}
	               \left[\begin{array}{ccc}
	                    \dots & v^T_1 & \dots  \\
	                    \dots & v^T_2 & \dots  \\
	                      &     \vdots   &  \\
	                    \dots & v^T_d & \dots  
	                \end{array}\right]
	                \end{equation*}
	        \end{column}
	    \end{columns}
	\end{frame}
\end{document}