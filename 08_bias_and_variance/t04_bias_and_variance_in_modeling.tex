\documentclass[aspectratio=169]{../latex_main/tntbeamer}  % you can pass all options of the beamer class, e.g., 'handout' or 'aspectratio=43'
\input{../latex_main/preamble}

\title[Statistics]{DS: Bias and Variance}
\subtitle{Bias and Variance in Modeling}

\graphicspath{ {./figure/} }
%\institute{}


\begin{document}
	
	\maketitle
	\begin{frame}{A Constant Model}
	    Let’s say you want to estimate how often a coin lands on heads when flipped.
	    \begin{itemize}
	        \item The result of a coin flip follows a Bernoulli-$p$ distribution, and you want to estimate $p$.
	        \item You do not collect any data, but instead you are given a choice between two models.
	        \item Suppose you are also told that p = .5.
	    \end{itemize}
	    Which of the following is the better model?\\
	    \bigskip
	    \textbf{Model A}: 
	    \begin{itemize}
	        \item Select a random number between 0 and 1 as estimate of $p$.
	        \item equivalent to running \texttt{np.random.random()} in Python
	     \end{itemize}
	    \bigskip
	    \textbf{Model B}: Select $.75$ as your estimate of $p$.
	\end{frame}
	
	
	\begin{frame}{A Constant Model}
	
	   How do we define \alert{“better”}?\\
	   \bigskip
	    We can calculate the expected MSE of each model. This is called the \alert{model risk}, a term which we will formalize later on. The lower the risk, the better.\\
	    \bigskip
	    \textbf{Model A}: 
	    \begin{itemize}
	        \item On average, we will select .5 as our estimate, so we \alert{expect} $0$ error.
	        \item As we are only selecting one number, we could select a number really far away from $.5$.\\
	    \end{itemize}
	    
	    \bigskip
	    \textbf{Model B}: 
	    \begin{itemize}
	        \item we will never be exactly correct
	        \item zero chance of a really terrible prediction
	    \end{itemize}
	\end{frame}
	
	
	\begin{frame}[c]{The Bias-Variance Tradeoff}
	   When building models, we generally face a tradeoff between bias and variance.
	   \begin{itemize}
	       \item Lower bias means that our model will predict closer to the truth, on average.
	       \item Lower variance means that our model will not change too much given the sample.
	   \end{itemize}
	   \bigskip
	   \begin{itemize}
	    \item Goal: low bias and low variance
	    \item Often trade-off between both
	    \item \textbf{Model A} has zero bias, but high variance
	    \item \textbf{Model B} has zero variance, but large bias.
	   \end{itemize}

	   \bigskip
	   So which is better? The answer will be revealed later in the lecture.
	\end{frame}
	
	
	\begin{frame}[c]{Three Sources of Error in Our Predictions}
	   Irreducible error: Recall the data generating process: $Y = g(x) + \epsilon$
	   \begin{equation*}
	       \mathbb{V}ar(\epsilon) = \sigma^2 \hspace{5mm} \mathbb{V}ar(Y) = \mathbb{V}ar(g(x) + \epsilon) = \mathbb{V}ar(g(x)) + \mathbb{V}ar(\epsilon) = \mathbb{V}ar(\epsilon) = 0 + \sigma^2
	   \end{equation*}
	   There will be chance error in our predictions due to the natural randomness of the world.\\
	   \bigskip
	   \alert{Model variance:} 
	   \begin{itemize}
	       \item Our fitted model is based on a random sample.
	       \item The sample could have come out differently, then the fitted model would have been different.
	   \end{itemize}

        \alert{Model bias:}
        \begin{itemize}
            \item This is the difference between the expected predictions, and the true $g(x)$.
            \item Our model may be too limited to find the correct $g(x)$, for example if we pick a quadratic model to fit to cubic data.
        \end{itemize}

	\end{frame}
	
	
	\begin{frame}[c]{Simulation}
	  \begin{columns}
	      \begin{column}{.6\textwidth}

	               \includegraphics[scale=.5]{Bild9}
	               
	      \end{column}
	      
	      \begin{column}{.4\textwidth}
	      \\
	      \bigskip
	      \bigskip
	      \bigskip
	            Let’s simulate the sampling and modeling process for a strictly \alert{linear model}:\\
	            $g(x) = \theta_0 + \theta_1x$
	      \end{column}
	  \end{columns}
	\end{frame}
	
	\begin{frame}[c]{Simulation}
	  \begin{columns}
	      \begin{column}{.6\textwidth}
	           \begin{figure}
	               \includegraphics[scale=.5]{Bild10}
	           \end{figure} 
	      \end{column}
	      
	      \begin{column}{.4\textwidth}
	      \\
	      \bigskip
	      \bigskip
	      \bigskip
	            Let’s simulate the sampling and modeling process for a \alert{quadratic model}:\\
	            $g(x) = \theta_0 + \theta_1x + \theta_2x^2$
	      \end{column}
	  \end{columns}
	\end{frame}
	
	
	\begin{frame}[c]{Simulation}
	  \begin{columns}
	      \begin{column}{.6\textwidth}
	           \begin{figure}
	               \includegraphics[scale=.5]{Bild11}
	           \end{figure} 
	      \end{column}
	      
	      \begin{column}{.4\textwidth}
	      \\
	      \bigskip
	      \bigskip
	      \bigskip
	            Let’s simulate the sampling and modeling process for a \alert{cubic model}:\\
	            $g(x) = \theta_0 + \theta_1x + \theta_2x^2 + \theta_3x^3$
	      \end{column}
	  \end{columns}
	\end{frame}
	
	
	
	\begin{frame}[c]{Simulation}
	  \begin{columns}
	      \begin{column}{.6\textwidth}
	           \begin{figure}
	               \includegraphics[scale=.5]{Bild12}
	           \end{figure} 
	      \end{column}
	      
	      \begin{column}{.4\textwidth}
	      \\
	      \bigskip
	      \bigskip
	      \bigskip
	           Let’s simulate the sampling and modeling process for a \alert{septic model}.\\
	            $g(x) = \theta_0 + \sum\limits_{i=1}^7 \theta_ix^i$
	            
	            \alert{Hint}: That one is not better than the cubic model
	      \end{column}
	      
	  \end{columns}
	\end{frame}
	
	\begin{frame}[c]{Diagram}
	  \begin{columns}
	     
	      
	      \begin{column}{.4\textwidth}
	      \\
	      \bigskip
	      \bigskip
	      \bigskip
	          Red line (fixed): $g(X)$\\
	          Green line $\mathbb{E}(\widehat{Y}(x))$
	          \begin{itemize}
	              \item This is fixed, given our model.
	          \end{itemize}
	          Gray lines: Possible $\widehat{Y}(x)$\\
	          Blue points: $Y = g(x) + \epsilon$
	      \end{column}
	      
	       \begin{column}{.6\textwidth}
	       
	               \includegraphics[scale=.3]{Bild13}
	               
	      \end{column}
	  \end{columns}
	\end{frame}
\end{document}