\documentclass[aspectratio=169]{../latex_main/tntbeamer}  % you can pass all options of the beamer class, e.g., 'handout' or 'aspectratio=43'
\input{../latex_main/preamble}

\title[Introduction]{DS: Bias and Variance}
\subtitle{Complexity}

\graphicspath{ {./figure/} }
%\institute{}


\begin{document}
	
	\maketitle
	\begin{frame}[c]{Modeling Goals}
	    \begin{itemize}
	        \item Try to minimize all three of observation variance, model bias, and model variance.
	    \end{itemize}
	    But
	    \begin{itemize}
	        \item Observation variance is often out of our control
	        \item Reducing complexity to reduce model variance can increase bias
	        \item Increasing model complexity to reduce bias can increase model variance
	        \item Domain knowledge matters: the right model structure!
	    \end{itemize}
	\end{frame}
	
	
	\begin{frame}{Bias Variance Plot}
	    \centering
	    \includegraphics[scale=.4]{Bild14}
	\end{frame}
	
	\begin{frame}{The right model structure matters!}
	    \begin{columns}
	        \begin{column}{.5\textwidth}
	                \begin{figure}
	                    \includegraphics[scale=.35]{Bild15}
	                \end{figure}
	                Ptolemaic Astronomy, a geocentric model based on circular orbits (epicycles and deferents).\\
	                \bigskip
	                High accuracy but very high model complexity.
	        \end{column}
	        
	        
	        \begin{column}{.5\textwidth}
	                \begin{figure}
	                    \includegraphics[scale=.35]{Bild16}
	                \end{figure}
	                Copernicus and Kepler: a heliocentric model with elliptical orbits.\\
	                \bigskip
	                Small model complexity yet high accuracy.

	        \end{column}
	    \end{columns}
	\end{frame}
\end{document}