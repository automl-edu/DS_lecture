\documentclass[aspectratio=169]{../latex_main/tntbeamer}  % you can pass all options of the beamer class, e.g., 'handout' or 'aspectratio=43'
\input{../latex_main/preamble}

\title[Introduction]{DS: Bias and Variance}
\subtitle{Data Generation Process}

\graphicspath{ {./figure/} }
%\institute{}


\begin{document}
	
	\maketitle
	\begin{frame}{Data Generation Process}
	    \begin{columns}
	        \begin{column}{.4\textwidth}
	                \begin{figure}
	                    \includegraphics[scale=.5]{Bild1}
	                \end{figure}
	        \end{column}
	        
	        
	         \begin{column}{.4\textwidth}
	            \\ \bigskip
	            \bigskip
	            \bigskip
	                \begin{itemize}
	                    \item Assume true relation g
	                    \item For example: $g(x) = \theta_0 + \theta_1x$
	                    \item For each individual:
	                    \begin{itemize}
	                        \item fixed value of x, so also g(x)
	                        \item random error $\epsilon$
	                        \item Observation is:
	                        \begin{equation*}
	                            Y = g(x) + \epsilon
	                        \end{equation*}
	                    \end{itemize}
	                \end{itemize}
	        \end{column}
	    \end{columns}
	    Errors $\epsilon$ have expectation 0, and are “iid” across individuals

	\end{frame}
	
	
	\begin{frame}{The Data}
	    \begin{columns}
	        \begin{column}{.4\textwidth}
	                \begin{itemize}
	                    \item At each x, truth is g(x)
	                    \item noise is $\epsilon$
	                    \item Observation is $Y = g(x) + \epsilon$
	                \end{itemize}
	                \begin{figure}
	                    \includegraphics[scale=.4]{Bild1}
	                \end{figure}
	        \end{column}
	        
	        
	         \begin{column}{.4\textwidth}
	         \\ \bigskip
	         \bigskip
	         \bigskip
	         \medskip
	               We only see Y
	                \begin{figure}
	                    \includegraphics[scale=.45]{Bild2}
	                \end{figure}
	        \end{column}
	    \end{columns}
	\end{frame}
	
	
	\begin{frame}{Our Predictions}
	    \begin{itemize}
	        \item We choose a model and fit it to our data
	        \begin{itemize}
	            \item Choosing a model is codifying our assumption of the form of g(x)
	        \end{itemize}
	        \item The red line is our fitted function—the best possible function given g(x)
	    \end{itemize}
	    \begin{columns}
	        \begin{column}{.4\textwidth}
	                \begin{figure}
	                    \includegraphics[scale=.5]{Bild3}
	                \end{figure}
	        \end{column}
	        
	        
	         \begin{column}{.4\textwidth}
	         \\ \bigskip
	         \bigskip
	               At every x, our prediction for Y is
	                \begin{itemize}
	                    \item the height of the red line at x
	                    \item Denote this  $\hat{Y}$(x)
	                \end{itemize}
	        \end{column}
	    \end{columns}
	\end{frame}
\end{document}