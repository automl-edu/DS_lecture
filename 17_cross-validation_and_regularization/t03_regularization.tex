\documentclass[aspectratio=169]{../latex_main/tntbeamer}  % you can pass all options of the beamer class, e.g., 'handout' or 'aspectratio=43'
\input{../latex_main/preamble}

\title[Introduction]{DS: Cross-Validation and Regularization}
\subtitle{Regularization}

\graphicspath{ {./figure/} }
%\institute{}


\begin{document}
	
	\maketitle
	\begin{frame}{Basic Idea}
	    \begin{equation*}
	        \hat{\theta} = \underset{\theta}{arg min} \frac{1}{n}\sum\limits_{i=1}^n\text{Loss}(y_i,f_\theta(x_i))
	    \end{equation*}
	    \underline{such that:}
	    \begin{equation*}
	        f_\theta \text{ does not 'overfit'}
	    \end{equation*}
	    Can we make this more formal?
	\end{frame}
	
	
	\begin{frame}[c]{Basic Idea}
	    \begin{equation*}
	        \hat{\theta} = \underset{\theta}{arg min} \frac{1}{n}\sum\limits_{i=1}^n\text{Loss}(y_i,f_\theta(x_i))
	    \end{equation*}
	    \underline{such that:}
	    \begin{equation*}
	         \text{Complexity}(f_\theta) \leq \beta
	    \end{equation*}
	    Complexity: How do we define this?\\
	    $\beta$:  Regularization Hyperparameter

	\end{frame}
	
	
	\begin{frame}[c]{Idealized Notion of Complexity}
	   \begin{equation*}
	         \text{Complexity}(f_\theta) \leq \beta
	    \end{equation*}
        \begin{itemize}
            \item Focus on complexity of linear models:
            \begin{itemize}
                \item Number and kinds of features
            \end{itemize}
            \item Ideal definition:
            \begin{equation*}
                \text{Complexity}(f_\theta) = \sum\limits_{j=1}^d\mathbb{I}[\theta_j \neq 0] 
            \end{equation*}
            \item Why?
        \end{itemize}
	\end{frame}
	
	
	\begin{frame}[c]{Ideal “Regularization”}
	Find the best value of $\theta$ which uses fewer than $\beta$ features.
	  \begin{equation*}
	        \hat{\theta} = \underset{\theta}{arg min} \frac{1}{n}\sum\limits_{i=1}^n\text{Loss}(y_i,f_\theta(x_i))
	    \end{equation*}
	    \underline{such that}
	    \begin{equation*}
                \text{Complexity}(f_\theta) = \sum\limits_{j=1}^d\mathbb{I}[\theta_j \neq 0] \leq \beta
        \end{equation*}
        Combinatorial search problem – NP-hard to solve in general.

	\end{frame}
	
	
	\begin{frame}{Norm Balls}
	    \includegraphics[scale=.35]{Bild5}
	\end{frame}
	
	\begin{frame}{Norm Balls}
	    \includegraphics[scale=.35]{Bild6}
	\end{frame}
	
	\begin{frame}{Norm Balls}
	    \includegraphics[scale=.35]{Bild7}
	\end{frame}
	
	\begin{frame}{Norm Balls}
	    \includegraphics[scale=.35]{Bild8}
	\end{frame}
	
	\begin{frame}{Norm Balls}
	    \includegraphics[scale=.35]{Bild9}
	\end{frame}
	
	
	\begin{frame}{Norm Balls}
	    \includegraphics[scale=.35]{Bild10}
	\end{frame}
	
	\begin{frame}{Norm Balls}
	    \includegraphics[scale=.35]{Bild11}
	\end{frame}
	
	\begin{frame}{Norm Balls}
	    \includegraphics[scale=.35]{Bild12}
	\end{frame}
	
	\begin{frame}{Norm Balls}
	    \includegraphics[scale=.35]{Bild13}
	\end{frame}
	
	
	\begin{frame}{Norm Balls}
	    \includegraphics[scale=.35]{Bild14}
	\end{frame}
	
	
	\begin{frame}{Norm Balls}
	    \includegraphics[scale=.35]{Bild15}
	\end{frame}
	
	
	\begin{frame}{Norm Balls}
	    \includegraphics[scale=.35]{Bild16}
	\end{frame}
	
	\begin{frame}{.}
	    \includegraphics[scale=.4]{Bild17}
	\end{frame}
\end{document}