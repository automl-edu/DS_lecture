\documentclass[aspectratio=169]{../latex_main/tntbeamer}  % you can pass all options of the beamer class, e.g., 'handout' or 'aspectratio=43'
\input{../latex_main/preamble}

\title[Introduction]{DS: Introduction}
\subtitle{Data Science Lifecycle}

\graphicspath{ {./figure/} }
%\institute{}


\begin{document}
	
	\maketitle

    \begin{frame}[c]{}

        \begin{figure}
            \includegraphics[scale=.6]{bild15}
        \end{figure}
        The “data science lifecycle” you will see out in the wild may be slightly different than the one we teach you, but the core ideas are all the same.

    \end{frame}

    \begin{frame}{Data science lifecycle}
        \begin{columns}
        \begin{column}{.6\textwidth}
            \begin{figure}
                \includegraphics[scale=.35]{bild16}
            \end{figure}
        \end{column}
        \begin{column}{.4\textwidth}
            The data science lifecycle is a high-level description of the data science workflow.\\
            Note the two distinct entry points!
        \end{column}
        \end{columns}
    \end{frame}

    \begin{frame}{1. Question/Problem Formulation}
        \begin{itemize}
            \item What do we want to know?
            \item What problems are we trying to solve?
            \item What are the hypotheses we want to test?
            \item What are our metrics for success?
        \end{itemize}
    \hfill
    \includegraphics[scale=.45]{bild17}
    \end{frame}

    \begin{frame}{2. Data Acquisition and Cleaning}
        \begin{itemize}
            \item What data do we have and what data do we need?
            \item How will we sample more data?
            \item Is our data representative of the population we want to study?
        \end{itemize}
        \includegraphics[scale=.45]{bild18}
        
    \end{frame}


    \begin{frame}{3. Exploratory Data Analysis & Visualization}
        \begin{columns}
            \begin{column}{.3\textwidth}
            \begin{figure}
                \includegraphics[scale=.45]{bild19}
                
            \end{figure}
            \end{column}
            \begin{column}{.4\textwidth}
            \begin{itemize}
                \item How is our data organized and what does it contain?
                \item Do we already have relevant data?
                \item What are the biases, anomalies, or other issues with the data?
                \item How do we transform the data to enable effective analysis?
            \end{itemize}
            \end{column}
  
        \end{columns}
    \end{frame}

    \begin{frame}{4. Prediction and Inference}
        \begin{itemize}
            \item What does the data say about the world?
            \item Does it answer our questions or accurately solve the problem?
            \item How robust are our conclusions and can we trust the predictions? 
        \end{itemize}
        \hfill
        \includegraphics[scale=.45]{bild20}
    \end{frame}



\end{document}