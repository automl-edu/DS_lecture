\documentclass[aspectratio=169]{../latex_main/tntbeamer}  % you can pass all options of the beamer class, e.g., 'handout' or 'aspectratio=43'
\input{../latex_main/preamble}

\title[Visualization]{DS: Visualization}
\subtitle{Describing quantitative distributions}

\graphicspath{ {./figure/} }
%\institute{}


\begin{document}
	
	\maketitle
	\begin{frame}{Describing distributions}
	    One of the benefits of a histogram or density curve is that they show us the “bigger picture” of our distribution (something we don’t get with a rug plot).\\
	    \bigskip
	    Some of the terminology we use to describe distributions:
	    \begin{itemize}
	        \item Modes.
	        \item Skewness.
	        \begin{itemize}
	            \item Skewed left vs skewed right.
	        \end{itemize}
	        \item Tails.
	        \begin{itemize}
	            \item Left tail vs right tail.	
	        \end{itemize}
	        \item Outliers.
	        \begin{itemize}
	            \item Define these arbitrarily.
	            \item Will see one definition in the next section.
	        \end{itemize}
	    \end{itemize}
	\end{frame}
	
	
	
	\begin{frame}{Modes}
        \begin{columns}
            \begin{column}{.5\textwidth}
            A mode of a distribution is a local or global maximum.
                    \begin{itemize}
                        \item A distribution with a single clear maximum is called unimodal.
                        \item Distributions with two modes are called bimodal.
                        \begin{itemize}
                            \item More than two: multimodal.
                            \item \alert{Note}: Need to distinguish between modes and random noise
                        \end{itemize}
                    \end{itemize}
            \end{column}
            
            
            \begin{column}{.4\textwidth}

                       \centering
                       \includegraphics[scale=.3]{Bild34}

            \end{column}
        \end{columns}
    \end{frame}
    
    
    \begin{frame}{Skew and tails}
        \begin{columns}
            \begin{column}{.5\textwidth}
            
                       \centering
                       \includegraphics[scale=.5]{Bild35}

            \end{column}
            
            
            \begin{column}{.4\textwidth}
                   If a distribution has a long right tail, we call it skewed right.
                   \begin{itemize}
                       \item Such an example is on the left
                       \item In such cases, the mean is typically to the right of the median.
                       \begin{itemize}
                           \item Think of the mean as the “balancing point” of the density.
                       \end{itemize}
                       \item In the event that the tail is on the left, we say the data is skewed left.
                       \item Our distribution can be symmetric, when both tails are of equal size.
                   \end{itemize}
            \end{column}
        \end{columns}
    \end{frame}
    
    
    
    \begin{frame}{Example}
        \begin{columns}
            \begin{column}{.5\textwidth}
            
                       \centering
                       \includegraphics[scale=.6]{Bild36}
                   
            \end{column}
            
            
            \begin{column}{.4\textwidth}
                   Consider the distribution of birth weights shown to the left. We might describe this as being:
                   \begin{itemize}
                       \item Unimodal. There is a single clear peak.
                       \item Symmetric. It doesn’t appear to be skewed in any direction.
                       \begin{itemize}
                           \item Mean is very close to the median.
                       \end{itemize}
                       \item Roughly normal.
                   \end{itemize}
            \end{column}
        \end{columns}
    \end{frame}
\end{document}