\documentclass[aspectratio=169]{../latex_main/tntbeamer}  % you can pass all options of the beamer class, e.g., 'handout' or 'aspectratio=43'
\input{../latex_main/preamble}

\title[Introduction]{DS: Feature Engineering}
\subtitle{Summary}

\graphicspath{ {./figure/} }
%\institute{}


\begin{document}
	
	\maketitle
	\begin{frame}{Feature Engineering Summary}
	    \begin{itemize}
	        \item Feature engineering is the process of creating new useful features from your data to build more sophisticated models
	        \item Feature engineering allows you to utilize non-numerical data
	        \begin{itemize}
	            \item One-hot encoding is a widely used technique
	        \end{itemize} 
	        \item Need to be careful in choosing how many and which features to create
	        \begin{itemize}
	            \item Linearly dependent features
	            \item Too many features
	        \end{itemize}
	        \item Feature engineering is as much an art as it is a skill
	        \begin{itemize}
	            \item Neural networks try to automatically do feature engineering 
	        \end{itemize}
	    \end{itemize}
	\end{frame}
	
\end{document}