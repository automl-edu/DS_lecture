\documentclass[aspectratio=169]{../latex_main/tntbeamer}  % you can pass all options of the beamer class, e.g., 'handout' or 'aspectratio=43'
\input{../latex_main/preamble}

\title[Introduction]{DS: Feature Engineering}
\subtitle{Motivating Feature Engineering}

\graphicspath{ {./figure/} }
%\institute{}


\begin{document}
	
	\maketitle
	\begin{frame}{Can we fit a model to this data?}
	    \centering
	    \includegraphics[scale=.75]{Bild1}
	\end{frame}
	
	
	\begin{frame}{Can we fit a model to this data?}
	    Simple Linear Regression?\\
	    No, because the data is fundamentally nonlinear
	    \begin{figure}
	        \centering
	        \includegraphics[scale=.65]{Bild2}
	    \end{figure}
	\end{frame}
	
	
	\begin{frame}{Can we fit a model to this data?}
	    Multiple Linear Regression?\\
	    No, because there are no other features to add
	    \begin{figure}
	        \centering
	        \includegraphics[scale=.6]{Bild1}
	    \end{figure}
	\end{frame}
	
	
	\begin{frame}{Can we fit a model to this data?}
	    Idea: Create an extra feature to use in the model. What feature should we add?\\
	    Since the data looks like a parabola, let’s add a quadratic feature
	    \begin{figure}
	        \centering
	        \includegraphics[scale=.65]{Bild3}
	    \end{figure}
	\end{frame}
	
	
	\begin{frame}{What is a feature?}
	    A feature is an input to our model
	    \begin{itemize}
	        \item So far, we have just used the raw data as features
	    \end{itemize}
	    We can also create new features to use an inputs to our model
	    \begin{itemize}
	        \item The process of creating new features is called feature engineering
	    \end{itemize}
	    Example: Quadratic model
	    \begin{columns}
	        \begin{column}{.4\textwidth}
	                \begin{figure}
                	        \centering
                	        \includegraphics[scale=.35]{Bild3}
	                 \end{figure}
	        \end{column}
	        
	        
	        \begin{column}{.4\textwidth}
	            \bigskip
	             
	              
	               \begin{equation*}
	                   \theta_0 + \theta_1x + \theta_2x^2 
	               \end{equation*} 
	               Features: x, $x^2$\\
	               Parameters: $\theta_0, \theta_1, \theta_2$

	        \end{column}
	    \end{columns}
	\end{frame}
	
	
	\begin{frame}{Can we really just create our own features?}
	    Yes! (with some restrictions)\\
	    We can create any feature we want as long we can write the model in the form
	    \begin{equation*}
	        \hat{y} = x^T\theta
	    \end{equation*}
	    This is a linear combination of the features. The features cannot depend on the parameters of the model!\\
	    Example: Adding a Quadratic Feature
	               \begin{equation*}
	                   \hat{y} = \theta_0 + \theta_1x_1 + \theta_2x_1^2 = \left[\begin{array}{c}
	                        1\\
	                        x_1\\
	                        x_1^2
	                   \end{array}\right] \cdot \left[\begin{array}{c}
	                        \theta_0\\
	                        \theta_1\\
	                        \theta_1
	                   \end{array}\right] = x^T\theta
	               \end{equation*} 
	\end{frame}
	
	
	\begin{frame}[c]{Lecture Roadmap}
	    We can choose/create x any way we like as long as our model follows the form $\hat{y}  = x^T\theta$
        The rest of the lecture will discuss different techniques we can use to create x:
        \begin{itemize}
            \item How can we create features from quantitative data?
            \item How can we create features from categorical data?
            \item How can we create features from text data? 
        \end{itemize} 
	\end{frame}
\end{document}