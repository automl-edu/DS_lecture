\documentclass[aspectratio=169]{../latex_main/tntbeamer}  % you can pass all options of the beamer class, e.g., 'handout' or 'aspectratio=43'
\input{../latex_main/preamble_2}

\title[MDS]{DS: Dimension Reduction}
\subtitle{Multi-Dimensional Scaling}

\graphicspath{ {./figure/} }
%\institute{}


\begin{document}
	
	\maketitle
	
    \begin{frame}[c]{Motivation}
        
        \begin{itemize}
            \item Let's assume that we have $d$ features in our input space,\\ but we would like to go down to $d'$ features ($d' \ll d$)
            \begin{itemize}
                \item $d'$ could be 2 in case we want to visualize our data
                \item Note: we only talk about $X$ and not about $y$ in any way here
            \end{itemize}
            \medskip
            \item What would be the best way to project down from $d$ features to $d'$ features?
            \item Main idea
            \begin{itemize}
                \item The information loss should be minimal!
                \item Information is best preserved if the distance between two points $x_i$ and $x_j$ in the $d$-dimensional space is roughly the same in the $d'$-dimensional space:
                $$ ||x_i - x_j || \approx || x'_i - x'_j || $$
                \item $||\cdot||$ can be any vector norm, e.g., Euclidean norm, but doesn't have to.
            \end{itemize}
        \end{itemize}

	\end{frame}
	
	\begin{frame}[c]{MDS Objective}
        
         $$\argmin_{x'_1, \ldots x'_M \in \mathbb{R}^{d'}} \sum_{i< j}  (||x_i - x_j || - || x'_i - x'_j ||)^2$$
        
        \begin{itemize}
            
            \item Note that $||x_i - x_j ||$ is fixed since the data is given
            \item We can only change our projected points $x'$
            \item[$\leadsto$] Depending on the vector norm and other assumptions different optimization approaches can be used to solve it
        \end{itemize}
        
	\end{frame}
	
	\begin{frame}[c]{MDS with Extensions}
        
        \begin{columns}
        
        \begin{column}{0.5\textwidth}
        
        \begin{itemize}
            \item Plot predictions in background ($y$)
            \item Plot labels with different points
            \smallskip
            \item Note: 
            \begin{itemize}
                \item Categorical features can lead to separated regions
                \item axes-scales have no real meaning anymore
                \item Scale your data first if your vector does not care about same scales 
            \end{itemize}
        \end{itemize}
        
        \end{column}
        
        \begin{column}{0.5\textwidth}
        
        \begin{center}
            \includegraphics[width=1\textwidth]{mds1.jpg}
        \end{center}
        
        \end{column}
        
        \end{columns}
        
	\end{frame}
	
	\begin{frame}[c]{MDS with Clusters}
        
        \begin{columns}
        
        \begin{column}{0.5\textwidth}
        
        \begin{itemize}
            \item If you have dense clusters in your data, MDS will zoom in on these
            \item Could be a hint that your i.i.d. assumption is violated
        \end{itemize}
        
        \end{column}
        
        \begin{column}{0.5\textwidth}
        
        \begin{center}
            \includegraphics[width=1\textwidth]{mds2.jpg}
        \end{center}
        
        \end{column}
        
        \end{columns}
        
	\end{frame}
	
\end{document}