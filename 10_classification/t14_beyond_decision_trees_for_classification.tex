\documentclass[aspectratio=169]{../latex_main/tntbeamer}  % you can pass all options of the beamer class, e.g., 'handout' or 'aspectratio=43'
\input{../latex_main/preamble}

\title[Introduction]{DS: Decision Trees}
\subtitle{Beyond Decision Trees for Classification}

\graphicspath{ {./figure/} }
%\institute{}


\begin{document}
	
	\maketitle
	\begin{frame}{Trees for Regression}
	    \begin{columns}
	    \begin{column}{.5\textwidth}
	        In earlier lectures we saw how we could use a logistic regression model for classification
            \begin{itemize}
                \item This lecture: We saw decision trees as an alternative technique for classification
            \end{itemize}
            \bigskip
            We could do the same exercise for regression
            \begin{itemize}
                \item Rather than using a linear model, we could build a regression tree
            \end{itemize}
        \end{column}
	    
        \begin{column}{.5\textwidth}
            
        
	        \begin{figure}
	            \centering
	            \includegraphics[scale=.4]{Bild58}
	        \end{figure}
	   \end{column}
	    \end{columns}
	\end{frame}
	
	
	
	\begin{frame}{Summary}
	    Decision trees provide an alternate non-linear framework for classification and regression
	    \begin{itemize}
	        \item The underlying principle is fundamentally different
	        \item Decision boundaries can be more complex
	        \item Danger of overfitting is high
	    \end{itemize}
	    \bigskip
	    Keeping complexity under control is not nearly as mathematically elegant and relies on heuristic rules
	    \begin{itemize}
	        \item Hard constraints
	        \item Pruning rules
	        \item Random forests
	        \begin{itemize}
	            \item Very interesting application of bootstrapping
	        \end{itemize}
	    \end{itemize}
	\end{frame}
	
	\begin{frame}{Summary}
	   In practice, you will see that there are actually even more conceptual frameworks for regression and classification:
	    \begin{itemize}
	        \item Linear models
	        \item Decision trees
	        \item Nearest neighbors
	        \item Support vector machines
	        \item Bayes Nets
	        \item Perceptrons
	        \begin{itemize}
	            \item Neural networks / deep learning
	        \end{itemize}
	        \item And many many many many more
	    \end{itemize}
	    Machine learning classes will go over these ideas in more detail
	    \begin{itemize}
	        \item How they work (and the tradeoffs for using them) are beyond our class
	    \end{itemize}
	\end{frame}
\end{document}