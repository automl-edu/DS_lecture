\documentclass[aspectratio=169]{../latex_main/tntbeamer}  % you can pass all options of the beamer class, e.g., 'handout' or 'aspectratio=43'
\input{../latex_main/preamble}

\title[Regression]{DS: Simple Linear Regression}
\subtitle{Correlation}

\graphicspath{ {./figure/} }
%\institute{}


\begin{document}
	
	\maketitle
	\begin{frame}{Exploring relationships between two variables}
	    \begin{itemize}
	        \item The constant model we saw in the last lecture was only able to capture the distribution of a single variable. It was a summary statistic.
	        \item More commonly, we create models that try to explain the relationships between multiple variables (which we will now denote with x and y).
	        \item When using the \texttt{Tips} dataset as a motivator in the last lecture, we looked at a histogram of the data (with an overlaid KDE).
	        \item When we have two continuous variables, we have several choices.
	        \begin{itemize}
	            \item Scatter plot. These are the simplest choice.
	            \item Hexbin plot.
	            \item Contour plot.
	        \end{itemize}
	    \end{itemize}
	\end{frame}
	
	
	\begin{frame}{Exploring relationships between two variables}
	    \includegraphics[scale=.4]{Bild1}
	\end{frame}
	
	
	
	
	\begin{frame}{Correlation coefficient}
	    The Pearson’s correlation coefficient (r) measures the strength of the linear association between two variables. 
	    \begin{itemize}
	        \item It is a unitless quantity.
	        \item Ranges between -1 and 1
	        \begin{itemize}
	            \item r = 1 indicates a perfect positive \alert{linear} association\\
	            i.e., x and y lie exactly on a straight line that is sloped upwards. 
	            \item r = -1 indicates a perfect negative linear association between x and y.
	            \item The closer r is to 0, the weaker the linear association between x and y is.
	        \end{itemize}
	        \item It says nothing about causality or non-linear association.
	        \begin{itemize}
	            \item Even if r = 1, it does not mean that x causes y! \alert{Correlation does not imply causality}.
	            \item When r = 0, we say our two variables are uncorrelated. They could be related through some non-linear association, though.
	        \end{itemize}
	        \item Very sensitive to outliers, as you will see in discussion. 
	    \end{itemize}
	\end{frame}
	
	
	\begin{frame}{Correlation coefficient}
	    r is the average of the product of x and y, both measured in standard units. \\
Suppose our data looks like ${(x_1, y_1),(x_2,y_2),...,(x_n,y_n)} $  $x_i$ in standard units = $ \frac{x_i - \hat{x}}{\sigma_x}$\\
\bigskip
We then have:
\begin{equation*}
    r=\frac{1}{n}\sum\limits_{i=1}^n\left(\frac{x_i-\overline{x}}{\sigma_x}\right)\left(\frac{y_i-\overline{y}}{\sigma_y}\right)
\end{equation*}
Note: Since    $\sigma_x$    and    $\sigma_y$    are constants, we can pull them out of the sum, and write: 

\begin{equation*}
   \frac{1}{n}\sum\limits_{i=1}^n\left(x_i-\overline{x}\right)\left(y_i-\overline{y}\right) = r\sigma_x\sigma_y
\end{equation*}
	This quantity is called the covariance of two variables.
   
	\end{frame}
	
	
	
	\begin{frame}{Correlation coefficient}
	    \includegraphics[scale=.4]{Bild2}
	\end{frame}
\end{document}