\documentclass[aspectratio=169]{../latex_main/tntbeamer}  % you can pass all options of the beamer class, e.g., 'handout' or 'aspectratio=43'
\input{../latex_main/preamble}

\title[Regression]{DS: Simple Linear Regression}
\subtitle{Recap: Modeling}

\graphicspath{ {./figure/} }
%\institute{}


\begin{document}
	
	\maketitle
	\begin{frame}{Recap: Modeling}
	    \begin{itemize}
	        \item Choose a model.
	        \begin{itemize}
	            \item So far, we’ve seen the constant model $\hat{y} = \theta $
	            \begin{itemize}
	                \item $\theta$ is called a parameter.
	            \end{itemize}
	        \end{itemize}
	       \item Choose a loss function.
            \begin{itemize}
                \item So far, the options have been squared loss or absolute loss.
                \begin{itemize}
                    \item Squared loss: $L_2(y,\hat{y}) = (y-\hat{y})^2$
                    \item Absolute loss:$ L_1(y,\hat{y}) = |y-\hat{y}|$
                \end{itemize}
                \item Loss functions tell us how much to penalize a single prediction
            \end{itemize}
            \item Minimize average loss across our entire dataset, to determine the optimal parameters.
            \begin{itemize}
                \item Smaller average loss values mean a better fit;\\ $\leadsto$ we have to find the parameters that minimize average loss
            \end{itemize}
	    \end{itemize}
	    \centering
	    $MSE(y,\hat{y})= \frac{1}{n}\sum\limits_{i=1}^n(y_i-\hat{y_i})^2$
	    \hspace{3cm}
	    $MAE(y,\hat{y})= \frac{1}{n}\sum\limits_{i=1}^n|y_i-\hat{y_i}|$
	\end{frame}
\end{document}