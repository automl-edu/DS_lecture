\documentclass[aspectratio=169]{../latex_main/tntbeamer}  % you can pass all options of the beamer class, e.g., 'handout' or 'aspectratio=43'
\input{../latex_main/preamble}

\title[Regression]{DS: Ordinary Least Squares}
\subtitle{Residuals}

\graphicspath{ {./figure_ols/} }
%\institute{}


\begin{document}
	
	\maketitle
	\begin{frame}{Residual plots}
	    Residual plots can tell us about the quality of our model.
	    \begin{itemize}
	        \item In the simple linear regression case, with only one independent variable, we typically plot residuals vs. $x$.
	        \item More generally, a residual plot is of residuals vs. fitted values.
	    \end{itemize}
	    Properties:
	    \begin{itemize}
	        \item A good residual plot has no pattern. This means that our model represents the relationship in the data well.
	        \begin{itemize}
	            \item If you see a curve, it is a sign that transformations or additional variables could help.
	        \end{itemize}
	        \item A good residual plot also has a similar vertical spread throughout the entire plot.
	        \begin{itemize}
	            \item If this is not the case, the accuracy of the predictions is not reliable.
	        \end{itemize}
	    \end{itemize}
	\end{frame}
	
	
	\begin{frame}{Residual plots}
	    \centering
	    \includegraphics[scale=.4]{Bild11}
	\end{frame}
	
	
	\begin{frame}{Residual plots}
	    \centering
	    \includegraphics[scale=.45]{Bild12}
	\end{frame}
	
	
	\begin{frame}{Residuals are orthogonal to the span of X}
	    When our linear model has an intercept term (i.e. when our design matrix has a column of all 1s), the following properties hold true:
	    \begin{itemize}
	        \item The sum of the residuals is 0.
            \begin{itemize}
                \item This is true, no matter what the features of model are.
                \item This is why the positive and negative residuals cancel out in any residual plot where the (linear) model contains an intercept term, even if the model is terrible.
            \end{itemize}
            \item The average true $y$ value is equal to the average predicted $\hat{y}$ value.
            \begin{itemize}
                \item This follows from the property above.
            \end{itemize}
	    \end{itemize}
	    
        These properties are true when there is an intercept term, and not necessarily when there isn’t.

	\end{frame}
\end{document}