\documentclass[aspectratio=169]{../latex_main/tntbeamer}  % you can pass all options of the beamer class, e.g., 'handout' or 'aspectratio=43'
\input{../latex_main/preamble}

\title[Regression]{DS: Simple Linear Regression}
\subtitle{Visualizing loss surfaces}

\graphicspath{ {./figure/} }
%\institute{}


\begin{document}
	
	\maketitle
	\begin{frame}{Visualizing loss surfaces}
	
	    \begin{columns}
	        \begin{column}{.55\textwidth}
	        
	                On the left, we have the plots of the loss surfaces for the constant model.
	                \begin{itemize}
	                    \item Top: squared loss (so average loss = MSE).
	                    \begin{itemize}
	                        \item The y-axis shows the MSE for each value of $\theta$ on the x-axis.
	                    \end{itemize}
	                    \item Bottom: absolute loss (so average loss = MAE).
	                \end{itemize}
	                The simple linear regression model has two parameters, a and b (or equivalently,   $\theta_0$    and    $\theta_1$    ). This means the loss surface will be 3D!

	        \end{column}
	        
	        \begin{column}{.45\textwidth}

                        \centering
	                   \includegraphics[scale=.3]{Bild5}

	        \end{column}
	    \end{columns}
	\end{frame}
	
	
	\begin{frame}{Visualizing loss surfaces}
	    \begin{columns}
	        \begin{column}{.5\textwidth}
	                Here, we have 3 axes.
	                \begin{itemize}
	                    \item One for   $\theta_0$ 
	                    \item One for      $\theta_1$ 
	                    \item One that tells us the mean squared error on our dataset, using the model    $\hat{y} = \theta_0 + \theta_1x$                           
	                \end{itemize}
	                The loss surface is nice and smooth (which we touted as a property of the squared loss in the last lecture).\\

	        \end{column}
	        
	        \begin{column}{.5\textwidth}
	               \begin{figure}
	                   \includegraphics[scale=.3]{Bild6}
	               \end{figure}

	        \end{column}
	    \end{columns}
	\end{frame}
\end{document}