\documentclass[aspectratio=169]{../latex_main/tntbeamer}  % you can pass all options of the beamer class, e.g., 'handout' or 'aspectratio=43'
\input{../latex_main/preamble}

\title[Introduction]{DS: Clustering, Part 1}
\subtitle{Overview}

\graphicspath{ {./figure/} }
%\institute{}


\begin{document}
	
	\maketitle
	\begin{frame}{Overview}
	    We’ve completed the core narrative arc of our class
	    \begin{itemize}
	        \item Sampling
	        \item Problem: Tabular Data Manipulation
	        \begin{itemize}
	            \item Solutions: Pandas and SQL
	        \end{itemize}
	        \item Problem: Regression and Classification
	        \begin{itemize}
	            \item Solution: Linear Models and Decision Trees
	        \end{itemize}
	        \item Problem: Dimensionality Reduction
	        \begin{itemize}
	            \item Solution: PCA
	        \end{itemize}
	    \end{itemize}
	    It turns out that these last two problems are examples of Machine Learning algorithms that “learn” from data
	\end{frame}
	
	
	\begin{frame}{Review: Taxonomy of Machine Learning}
	    \begin{figure}
	        \centering
	        \includegraphics[scale=.4]{Bild1}
	    \end{figure}
	\end{frame}
	
	
	\begin{frame}{Review: Taxonomy of Machine Learning}
	    \begin{columns}
	        \begin{column}{.5\textwidth}
	                \begin{figure}
	                    \centering
	                    \includegraphics[scale=.33]{Bild2}
	                \end{figure}
	        \end{column}
	        
	        
	        \begin{column}{.5\textwidth}
	            \\
	            \bigskip In “Supervised Learning”:
	                \begin{itemize}
	                    \item Goal is to create a function that maps inputs to outputs
	                    \item Model is learned from example input/output pairs. Each pair consists of:
	                    \begin{itemize}
	                        \item Input vector
	                        \item Output value (label)
	                    \end{itemize}
	                    \item Regression: Output value is quantitative
	                    \item Classification: Output value is categorical
	                \end{itemize}
	        \end{column}
	    \end{columns}
	\end{frame}
	
	
	
	\begin{frame}{Review: Taxonomy of Machine Learning}
	    \begin{columns}
	    

	        \begin{column}{.5\textwidth}
	               In “Unsupervised Learning”:
	               \begin{itemize}
	                   \item Goal is to identify patterns in unlabeled data
	                   \begin{itemize}
	                       \item We do not have input/output pairs
	                   \end{itemize}
	               \end{itemize}
	               Note that if even if we have labels, we can still use clustering to “figure out” the labels
	        \end{column}
	        
	        
	        \begin{column}{.5\textwidth}
	                 \begin{figure}
	                    \centering
	                    \includegraphics[scale=.33]{Bild3}
	                \end{figure}
	        \end{column}
	    \end{columns}
	\end{frame}
	
	
	
	\begin{frame}{Clustering Example}
	    \begin{columns}
	    

	        \begin{column}{.5\textwidth}
	               Consider the figure shown from Fall 2019 Midterm 2
	               \begin{itemize}
	                   \item Recall that each point represents the 1st and 2nd PC of how much time patrons spent at 8 different zoo exhibits
	               \end{itemize}
	                   Goal of clustering: Assign each point to a cluster\\
	                   \bigskip
	                   Goal of clustering: Assign each point to a cluster
	                   \begin{itemize}
	                       \item We don’t have labels for each visitor
	                       \item Want to infer pattern even without labels
	                   \end{itemize}


	        \end{column}
	        
	        
	        \begin{column}{.5\textwidth}
	                 \begin{figure}
	                    \centering
	                    \includegraphics[scale=.4]{Bild4}
	                \end{figure}
	        \end{column}
	    \end{columns}
	\end{frame}
	
	
	
	\begin{frame}{Clustering Example 1: Netflix}
	    Suppose you’re Netflix and have information on customer viewing habits
	    \begin{itemize}
	        \item Can use clustering to assign each person or show to a “cluster”
	        \item Don’t have to define clusters in advance
	    \end{itemize}
	    \bigskip
	    Clustering is different from classification
	    \begin{itemize}
	        \item With classification, you have to decide on labels in advance
	        \item Clustering discovers groups automatically
	    \end{itemize}
	\end{frame}
	
	
	\begin{frame}{Clustering Example 1: Netflix}
	    \begin{figure}
	        \centering
	        \includegraphics[scale=.4]{Bild5}
	    \end{figure}
	\end{frame}
	
	
	
	\begin{frame}{Clustering Example 2: Clustering Students}
	    In 2018, as a tiny part of a project working to understand factors that affect 61B student success, Prof. Josh Hug tried clustering students based on:
	    \begin{itemize}
	        \item Time and number of posts on Piazza
	        \item Time and number of submissions to Gradescope
	        \item Time and number of submissions to GitHub (basically whenever students saved work)
	    \end{itemize}
	    \bigskip
	    Clustering algorithm automatically identified procrastinating students
	    \bigskip
	    This result wasn’t particularly useful, but it was somewhat interesting

	\end{frame}
	
	
	
	\begin{frame}{Clustering Example 3: Reverse Engineering Biology}
	    \begin{columns}
	    

	        \begin{column}{.5\textwidth}
	               In plot to the right:
	               \begin{itemize}
	                   \item Rows are conditions (e.g. a row might be: “poured acid on the cells”)
	                   \item Columns are genes
	               \end{itemize}
	                   \\
	                   \bigskip
	                   Green indicates that the gene was ~off
	                   \begin{itemize}
	                       \item The ~9 genes on the left all got turned off by the 6 experiments at the top
	                   \end{itemize}
	                   \\
	                   \bigskip
                        Clustering brings similar observations together

	        \end{column}
	        
	        
	        \begin{column}{.5\textwidth}
	                 \begin{figure}
	                    \centering
	                    \includegraphics[scale=.6]{Bild6}
	                \end{figure}
	        \end{column}
	    \end{columns}
	\end{frame}
\end{document}