\documentclass[aspectratio=169]{../latex_main/tntbeamer}  % you can pass all options of the beamer class, e.g., 'handout' or 'aspectratio=43'
\input{../latex_main/preamble}

\title[Introduction]{DS: Introduction to Modeling}
\subtitle{Comparing loss functions}

\graphicspath{ {./figure/} }
%\institute{}


\begin{document}
	
	\maketitle
	\begin{frame}{MSE vs. MAE for toy data}
	    Below, we present the plot of the loss surface for our toy dataset, using L2 loss (left) and L1 loss (right).
	    \begin{itemize}
	        \item A loss surface is a plot of the loss encountered for each possible value of    $\theta$ .
	        \item If our model had 2 parameters, this plot would be 3 dimensional.
	    \end{itemize}
	    \centering
	    \includegraphics[scale=.4]{Bild45}
	\end{frame}
	
	
	\begin{frame}{MSE vs. MAE}
	    What else is different about squared loss (MSE) and absolute loss (MAE)?\\
        Mean squared error (optimal parameter for the constant model is the sample mean)\\
        \begin{itemize}
            \item Very smooth. Easy to minimize using numerical methods (coming later in the course).
            \item Very sensitive to outliers, e.g. if we added 1000 to our largest observation, the optimal theta would become 225 instead of 25.
        \end{itemize}
        \bigskip
        Mean absolute error (optimal parameter for the constant model is the sample median)
        \begin{itemize}
            \item Not as smooth – at each of the “kinks,” it’s not differentiable. Harder to minimize.
            \item Robust to outliers! E.g, adding 1000 to our largest observation doesn’t change the median.
        \end{itemize}
        \bigskip
        It’s not clear that one is “better” than the other. \\
        In practice, we get to choose our loss function!\\
        
        \vspace{-7cm}
        \hspace{6cm} \includegraphics[scale=.33]{Bild46}\\
        \vspace{7cm}
	\end{frame}
	
\end{document}