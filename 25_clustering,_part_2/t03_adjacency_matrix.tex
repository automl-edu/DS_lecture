\documentclass[aspectratio=169]{../latex_main/tntbeamer}  % you can pass all options of the beamer class, e.g., 'handout' or 'aspectratio=43'
\input{../latex_main/preamble}

\title[Introduction]{DS: Clustering, Part 2}
\subtitle{Adjacency Matrix}

\graphicspath{ {./figure/} }
%\institute{}


\begin{document}
	
	\maketitle
	\begin{frame}{Graph}
	    A graph is defined by its vertices, and its edges. Often, you might see the expression G = (V, E)\\
        But how do we express graphs mathematically?\\
        Recall, these two graphs are the same graph, so they should have the same representation.
        \begin{figure}
            \centering
            \includegraphics[scale=.5]{Bild11}
        \end{figure}
	\end{frame}
	
	
	\begin{frame}{Adjacency Matrix}
	   We will represent the edges in an adjacency matrix A.\\
      Two graphs are the same if and only if they have the same adjacency matrix.\\
      A will have one row and one column for each vertex.
        \begin{itemize}
            \item In other words, A will be of size n x n, if the graph has n vertices.
        \end{itemize}
        
        \begin{align*}
            A_{ij} &= \text{ the weight of the edge connecting vertices i and j} \\
                    &= \text{0 if no edge connects vertices i and j}
        \end{align*}
        \begin{figure}
            \centering
            \includegraphics[scale=.245]{Bild12}
        \end{figure}
	\end{frame}
	
	
	
	\begin{frame}{Adjacency Matrix}
	    These two graphs have the same adjacency matrix, so they are the same graph.
	    \begin{figure}
	        \centering
	        \includegraphics[scale=.4]{Bild13}
	    \end{figure}
	\end{frame}
	
	
	
	\begin{frame}{Adjacency Matrix}
	    These two graphs have different adjacency matrices, so they are different graphs.
	    \begin{figure}
	        \centering
	        \includegraphics[scale=.4]{Bild14}
	    \end{figure}
	\end{frame}
	
	
	
	\begin{frame}{Properties}
	    The adjacency matrix gives our graph a mathematical representation.\\
        Notice the matrix is square, and symmetric, that is: $A_{ij} = A_{ij}\forall i,j$\\
        \bigskip
        Can think of our adjacency matrix as “featurizing” our graph:\\
        \begin{itemize}
            \item  Each row represents an individual data point (vertex)
            \item Each column represents that data point’s relationship with other data points (edges)
        \end{itemize}
        \bigskip
        Remember, a larger edge weight represents a stronger connection between two vertices.
	\end{frame}
\end{document}