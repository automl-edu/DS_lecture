\documentclass[aspectratio=169]{../latex_main/tntbeamer}  % you can pass all options of the beamer class, e.g., 'handout' or 'aspectratio=43'
\input{../latex_main/preamble}

\title[Introduction]{DS: Clustering, Part 2}
\subtitle{Spectral Clustering}

\graphicspath{ {./figure/} }
%\institute{}


\begin{document}
	
	\maketitle
	\begin{frame}{A Note}
	    It is now time to present the spectral clustering algorithm.\\
	    \bigskip
	    However, note that spectral clustering is a relatively new area of research, and there is no one single “right” way to go about it.\\
	    \bigskip
	    I will present a modified/simplified version of the algorithm developed by Jordan, Ng, and Weiss (2002). But note that there are other methods, and you might learn spectral clustering differently in other classes.
	    \begin{figure}
	        \centering
	        \includegraphics[scale=.5]{Bild15}
	    \end{figure}
	\end{frame}
	
	
	\begin{frame}{Spectral Clustering Algorithm}
	    Given a graph G = (V, E):\\
	    \begin{itemize}
	        \item[1]  Calculate the Laplacian matrix L = D - A.
	        \item[2] Calculate the eigenvectors of L.
	        \item[3] Pick the number of clusters k, and select the k eigenvectors of L corresponding to the k smallest eigenvalues. Each vertex now has its own coordinate, containing its corresponding element in each of the k eigenvectors.
	        \item[4] Use k-means clustering to cluster these points.
	    \end{itemize}
	\end{frame}
	
	
	
	\begin{frame}{Example}
	    Let’s say we want to divide our graph into k = 2 clusters.
	    \begin{figure}
	        \centering
	        \includegraphics[scale=.5]{Bild15}
	    \end{figure}
	    We select the last two eigenvectors in V. These are $v_n$ and $v_{n-1}$.
	    \begin{itemize}
	        \item Remember, our eigenvectors are sorted by descending order of their eigenvalues. 
	    \end{itemize}
        \begin{align*}
            v_n = \left[\begin{array}{c}
                 -.5 \\
                 -.5\\
                 -.5\\
                 -.5
            \end{array}\right]
            v_{n-1} = \left[\begin{array}{c}
                 -.22  \\
                 -.31 \\
                 -.34 \\
                 .86
            \end{array}\right]
        \end{align*}
	\end{frame}
	
	
	\begin{frame}{Example}
	    Now, each vertex is represented by a specific point in 2D.\\
        For example, the coordinate for vertex P is [$\begin{array}{cc}
             -.5   & -.22 
            \end{array}$]\\
        On the left are these points, and on the right is the result of clustering those points.
        \begin{figure}
            \centering
            \includegraphics[scale=.45]{Bild16}
        \end{figure}
	\end{frame}
\end{document}