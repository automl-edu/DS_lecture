\documentclass[aspectratio=169]{../latex_main/tntbeamer}  % you can pass all options of the beamer class, e.g., 'handout' or 'aspectratio=43'
\input{../latex_main/preamble}

\title[DL in a Nutshell]{DS: Deep Learning}
\subtitle{DL in a Nutshell}

\date{\hspace{0.5em} {\includegraphics[height=1.5em]{../latex_main/figures/Cc-by-nc-sa_icon.svg.png}}; Inspired by \href{https://www.deeplearning.ai/resources/}{Andrew Ng}}

\graphicspath{ {./figure/} }
%\institute{}


\begin{document}
	
	\maketitle
	\begin{frame}{Scaling with Data}

        \begin{itemize}
            \item More and more data is available these days
            \begin{itemize}
                \item with all its benefits and downsides
            \end{itemize}
            \item Models need to scale with data
        \end{itemize}

        \centering
        \includegraphics[width=0.6\textwidth]{figure/dl_scaling}


	\end{frame}

 	\begin{frame}{Layer-wise Feature Engineering}
        % Housing Price Prediction

        \begin{itemize}
            \item By combining features, you can get interesting new features
            \item Main idea: combine features (pairwise) in each layer and build stronger and stronger feature representations
            \item Important: A linear combination of a linear combination of features is still a linear combination $\leadsto$ we need non-linear combinations
        \end{itemize}

        \centering
        \includegraphics[width=0.4\textwidth]{figure/layer_rep}

	\end{frame}

  	\begin{frame}{Application to different Tasks and Data Modalities}
        % table Input, Output, Application, Modality, Network Type

        {\centering
        \begin{tabular}{lllll}
             Input & Output & Application & Modality & Network Type  \\
             \midrule
             Home Features & Price & Real Estate & Table & MLP \\ \pause
             Ad, user info & Click on ad? (binary) & Online Advertising & Table & MLP\\ \pause
             Image & Object (1,\ldots, 1000) & Photo Tagging & Images & CNN\\ \pause
             Audio & Text transcript & Speech recognition & Audio Sequence & RNN\\ \pause
             English & German & Machine Translation & Text & Transformer\\ \pause
             Image, Radar Info & Position of other cars & Autonomous driving & Multi-modal & Custom\\ \pause
        \end{tabular}}

        \vspace{2em}

        $\leadsto$ Consider input, output and data modality to decide which kind of network type you need\\ -- more details later!

	\end{frame}
	

   	\begin{frame}{Structured vs Unstructured Data}
        % Structured Data: Table
        % Unstructured Data: Image, Audio, Text

        \begin{columns}

        \column{0.5\textwidth}

        Structure data: Tables\\[2em]

        \centering
        \begin{tabular}{llc|r}
             Size & bedrooms & \ldots & Price  \\
             \midrule
             2104 & 3       & & 400.000\\
             1600 & 3 & & 330.000\\
             \ldots \\
             3000 & 4 & & 540.000\\
        \end{tabular}

        \column{0.5\textwidth}
            
        
        
        Unstructured Data

        \includegraphics[width=0.4\textwidth]{figure/cat.png}
        \includegraphics[width=0.4\textwidth]{figure/audio.png}

        \vspace{2em}

        Text: \textit{Once upon a time, there was \ldots}

        \end{columns}

        $\leadsto$ So far, Deep Learning is mostly successful on unstructured data (e.g., images, sound, text). 

	\end{frame}

\end{document}