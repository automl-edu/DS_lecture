\documentclass[aspectratio=169]{../latex_main/tntbeamer}  % you can pass all options of the beamer class, e.g., 'handout' or 'aspectratio=43'
\input{../latex_main/preamble}

\title[Introduction]{DS: Logistic Regression, Classification}
\subtitle{Multiclass classification (Extra)}

\graphicspath{ {./figure/} }
%\institute{}


\begin{document}
	
	\maketitle
	\begin{frame}{Note}
	    \begin{itemize}
	        \item We will not cover these slides in lecture.
	        \item They are meant to serve as a reference for the lab assignment that covers multiclass classification in the context of decision trees.
	    \end{itemize}
	\end{frame}
	
	
	\begin{frame}{Multiclass classification}
	    Sometimes we have more than one class that we’re interested in.\\
	    \bigskip
	    Example, we want to predict what kind of animal an image contains, of the following 5 choices.
	    \begin{itemize}
	        \item Dog
	        \item Cat
	        \item Lion
	        \item Zebra
	        \item Other
	    \end{itemize}
	\end{frame}
	
	
	\begin{frame}{Multiclass classification: one vs. rest}
	    The simplest way to do multiclass classification is to build N binary classifiers, one for each category.
	    \begin{itemize}
	        \item Resulting prediction will just be whichever classifier gives highest probability.
	    \end{itemize}
	    Example from before:
	    \begin{itemize}
	        \item Build a dog classifier.
	        \item Build an cat classifier.
	        \item Build a lion classifier.
	        \item Build an zebra classifier…
	    \end{itemize}
	    Given a voter, assign the class which has the highest probability among all N.
	\end{frame}
	
	
	\begin{frame}{Visual example}
	    \begin{figure}
	        \centering
	        \includegraphics[scale=.5]{Bild49}
	    \end{figure}
	\end{frame}
	
	
	\begin{frame}{Visual example}
	    \begin{figure}
	        \centering
	        \includegraphics[scale=.5]{Bild50}
	    \end{figure}
	\end{frame}
	
	\begin{frame}{Visual example}
	    \begin{figure}
	        \centering
	        \includegraphics[scale=.5]{Bild51}
	    \end{figure}
	\end{frame}
	
	
	\begin{frame}{Visual example}
	    \begin{figure}
	        \centering
	        \includegraphics[scale=.5]{Bild52}
	    \end{figure}
	\end{frame}
	
	
	\begin{frame}{Visual example}
	    \begin{figure}
	        \centering
	        \includegraphics[scale=.5]{Bild53}
	    \end{figure}
	\end{frame}
	
	
	
	\begin{frame}{Multiclass classification: softmax}
	    One downside of building N binary classifiers: Class imbalance.\\
        Alternate techniques exist that we will not discuss.\\
        \bigskip
        Example: Softmax.
        \begin{itemize}
            \item Related to neural networks.
            \item Idea: Different theta for every class, i.e. for class j we have   $\theta^{(j)}$       . 
            \item Won’t discuss in Data 100 . See CS188, CS189, CS182, Info 254, or Stat 151A.
        \end{itemize}
        
        \begin{equation*}
            P(Y=j|x) = \frac{exp(x^T\theta^{(j)})}{\sum_{m=1}^kexp(x^T\theta^{(m)})}
        \end{equation*}
	\end{frame}
\end{document}