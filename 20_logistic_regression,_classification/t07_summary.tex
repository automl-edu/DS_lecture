\documentclass[aspectratio=169]{../latex_main/tntbeamer}  % you can pass all options of the beamer class, e.g., 'handout' or 'aspectratio=43'
\input{../latex_main/preamble}

\title[Introduction]{DS: Logistic Regression, Classification}
\subtitle{Summary}

\graphicspath{ {./figure/} }
%\institute{}


\begin{document}
	
	\maketitle
	\begin{frame}{Summary}
	    \begin{itemize}
	        \item Logistic regression models the probability of belonging to class 1.
	        \begin{itemize}
	            \item Designed for binary classification.
	        \end{itemize}
	        \item In order to make classifications, we employ a threshold, or decision rule.
	        \begin{itemize}
	            \item Different thresholds yield different decision boundaries.
	        \end{itemize}
	        \item To evaluate our models, we can look at several numeric and visual metrics.
	        \begin{itemize}
	            \item Accuracy, precision, recall.
	            \item PR curves, ROC curves.
	        \end{itemize}
	        \item Decision boundaries for logistic regression are linear in terms of the model’s features.
	        \item Using regularization for logistic regression is a good idea.
	        \begin{itemize}
	            \item It is necessary when our data is linearly separable to prevent our weights from diverging.
	        \end{itemize}
	    \end{itemize}
	\end{frame}
	
	
\end{document}