\documentclass[aspectratio=169]{../latex_main/tntbeamer}  % you can pass all options of the beamer class, e.g., 'handout' or 'aspectratio=43'
\input{../latex_main/preamble}

\title[Introduction]{DS: Logistic Regression, Classification}
\subtitle{Thresholding}

\graphicspath{ {./figure/} }
%\institute{}


\begin{document}
	
	\maketitle
	\begin{frame}{Classification}
	    Our motivation for performing logistic regression was to predict categorical labels. Specifically, we were looking to perform binary classification, i.e. classification where our outputs are 1 or 0.
	    \begin{figure}
	        \centering
	        \includegraphics[scale=.5]{Bild1}
	    \end{figure}
	    However, the output of logistic regression is a continuous value in the range [0, 1], which we interpret as a probability – specifically,  P(Y=1|x)\\
	    \bigskip
	    In order to classify – that is, to predict a 1 or 0 – we pair our logistic model with a decision rule, or threshold.
	\end{frame}
	
	
	
	\begin{frame}{Thresholds}
	    Given an observation x, the following decision rule outputs 1 or 0, depending on the probability that our model assigns to x belonging to class 1.\\
	    \bigskip
	    Example for T = 0.5:
	    \begin{equation*}
	        classify(x)  =\left\{\begin{array}{cc}
	            1, &  P(Y=1|x) \geq 0.5 \\
	            0, &  P(Y=1|x) < 0.5
	        \end{array}\right.
	            \includegraphics[scale=.25]{Bild2}
	    \end{equation*}
	    \begin{itemize}
	        \item Note: We don’t need to set our threshold to 0.5. Depending on the type of errors we want to minimize, we can increase or decrease it.
	        \begin{itemize}
	            \item 0.5 is the default in scikit-learn’s LogisticRegression, though.
	        \end{itemize}
	        \item Logistic regression paired with a decision rule is a classifier.
	    \end{itemize}
	\end{frame}
	
	
	\begin{frame}{Thresholds}
	    Consider the single-feature logistic regression model from last lecture:
	    \begin{equation*}
	        P(Y=1|x) = \sigma (\theta_1 \cdot FG\_PCT\_DIFF)
	    \end{equation*}
	    Here, the blue line represents our modeled probabilities, and the red lines represent various thresholds.
	    \begin{figure}
	        \centering
	        \includegraphics[scale=.5]{Bild3}
	    \end{figure}
	\end{frame}
	
	
	\begin{frame}{Thresholds in higher dimensions}
	    \begin{columns}
	        \begin{column}{.7\textwidth}
	                Our thresholds work the same way, even if our models have multiple features.\\
	                \bigskip
	                Suppose we fit a model with 2 features – FG\_PCT\_DIFF and PF\_DIFF – along with an intercept term.
	                \begin{align*}
	                    P(Y=1|x) = \sigma (\theta_0  + \theta_1 \cdot FG\_PCT\_DIFF + \theta_2 \cdot PF\_DIFF)
	                \end{align*}
	        \end{column}
	        
	        
	        \begin{column}{.4\textwidth}
	                \begin{figure}
	                    \centering
	                    \includegraphics[scale=.5]{Bild4}
	                \end{figure}
	        \end{column}
	    \end{columns}
	\end{frame}
	
	
	\begin{frame}{Thresholds in higher dimensions}
	    \begin{columns}
	        \begin{column}{.7\textwidth}
	                Our thresholds work the same way, even if our models have multiple features.\\
	                \bigskip
	                Suppose we fit a model with 2 features – FG\_PCT\_DIFF and PF\_DIFF – along with an intercept term.
	                \begin{align*}
	                    P(Y=1|x) = \sigma (\theta_0  + \theta_1 \cdot FG\_PCT\_DIFF + \theta_2 \cdot PF\_DIFF)
	                \end{align*}
	                Any data point whose predicted probability is greater than 0.3 (above the plane) is classified as 1.
	        \end{column}
	        
	        
	        \begin{column}{.4\textwidth}
	                \begin{figure}
	                    \centering
	                    \includegraphics[scale=.35]{Bild5}
	                \end{figure}
	        \end{column}
	    \end{columns}
	\end{frame}
	
	
	
	\begin{frame}{From probabilities to labels}
	    \begin{columns}
	        \begin{column}{.7\textwidth}
	                With different thresholds, we get different predictions.
	                \begin{itemize}
	                    \item Everything above the red line is classified as 1, and everything below is classified as 0.
	                    \item The larger we make T, our threshold, the fewer observations are classified as 1.
	                    \begin{itemize}
	                        \item The “standard” is higher.
	                    \end{itemize}
	                \end{itemize}
	                \begin{figure}
	                    \centering
	                    \includegraphics[scale=.5]{Bild3}
	                \end{figure}
	        \end{column}
	        
	        
	        \begin{column}{.4\textwidth}
	                \begin{figure}
	                    \centering
	                    \includegraphics[scale=.5]{Bild6}
	                \end{figure}
	        \end{column}
	    \end{columns}
	\end{frame}
	
\end{document}