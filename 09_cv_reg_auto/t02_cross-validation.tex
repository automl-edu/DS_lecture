\documentclass[aspectratio=169]{../latex_main/tntbeamer}  % you can pass all options of the beamer class, e.g., 'handout' or 'aspectratio=43'
\input{../latex_main/preamble}

\title[CV, Reg \& AutoML]{DS: Cross-Validation, Regularization \& AutoML}
\subtitle{Cross-Validation}

\graphicspath{ {./figure/} }
%\institute{}


\begin{document}
	
	\maketitle
	\begin{frame}{Training Error vs Test Error}
	    \includegraphics[scale=.4]{Bild1}
	\end{frame}
	
	
	\begin{frame}{Training Error vs Test Error}
	    \includegraphics[scale=.4]{Bild2}
	\end{frame}
	
	
	\begin{frame}{Generalization: The Train-Test Split}
	    \begin{columns}
	        \begin{column}{.5\textwidth}
	               \begin{itemize}
	                   \item Training Data: used to fit model
	                   \item Test Data: check generalization error
	                   \item How to split? 
	                   \begin{itemize}
	                       \item Randomly, Temporally, Geo…
	                       \item Depends on application (usually randomly)
	                   \end{itemize}
	                   \item What size? (90\%-10\%)
	                   \begin{itemize}
	                       \item Larger training set – more complex models
	                       \item Larger test set – better estimate of generalization error 
	                       \item Typically between 75\%-25\% and 90\%-10\%
	                   \end{itemize}
	               \end{itemize} 
	        \end{column}
	        
	        
	       \begin{column}{.5\textwidth}
	                \begin{figure}
	                    \includegraphics[scale=.5]{Bild3}
	                \end{figure}
	        \end{column}
	    \end{columns}
	    \bigskip
	    \alert{Warning:} You can only use the test dataset once after deciding on the model.

	\end{frame}
	
	\begin{frame}{Generalization: Validation Split}
	    \includegraphics[scale=.43]{Bild4}\\
	    Cross validation simulates multiple train test-splits within the training data.

	\end{frame}
	
	\begin{frame}[c]{Recipe for Successful Generalization}
	    \begin{itemize}
	        \item Split your data into training and test sets (90\%, 10\%)
	        \item Use only the training data when designing, training, and tuning the model
	        \begin{itemize}
	            \item Use cross validation to test generalization during this phase
	            \item Do not look at the test data!
	        \end{itemize}
	        \item Commit to your final model and train once more using only the training data.
	        \item Test the final model using the test data. 
	        \begin{itemize}
	            \item If you can afford it, you can also use nested cross-validation:\\ (i) for simulating test data and (ii) for getting validation data
	        \end{itemize}
	        \item Train your model on all available data and deploy it!
	    \end{itemize}
	\end{frame}
\end{document}