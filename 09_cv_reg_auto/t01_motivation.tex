\documentclass[aspectratio=169]{../latex_main/tntbeamer}  % you can pass all options of the beamer class, e.g., 'handout' or 'aspectratio=43'
\input{../latex_main/preamble}

\title[Introduction]{DS: Cross-Validation, Regularization & AutoML}
\subtitle{Overview}

\graphicspath{ {./figure/} }
%\institute{}


\begin{document}
	
	\maketitle
	\begin{frame}[c]{Idea: Validation}
	    
	    \begin{itemize}
	        \item Given: $n$ observation in a dataset $\mathcal{D}$
	        \item What could happen if we use $\mathcal{D}$ to train our machine learning model $\mathcal{M}$ and evaluate it on the same $\mathcal{D}$?
	        \pause
	        \item[$\leadsto$] $\mathcal{M}$ could simply remember $\mathcal{D}$, but is not able to provide reasonable predictions for any point not being in $\mathcal{D}$
	        \bigskip
	        \pause
	        \item So, how can we validate $\mathcal{M}$ s.t. we know how well it will generalize to new data?
	    
	    \end{itemize}
	    
	    
	\end{frame}
	
	\begin{frame}[c]{Idea: Regularization}
	    
	    \begin{itemize}
	        \item Problem: Even if we figure out that $\mathcal{M}$ simply remembered $\mathcal{D}$, how can we deal with it?
	        \begin{itemize}
	            \item Big problem if we have large noise in $\mathcal{D}$, and $\mathcal{M}$ learns the noise and not the important patterns
	            \item Remember \alert{over-fitting} from previous session
	        \end{itemize}
	        \pause
	        \bigskip
	        \item Can we tell $\mathcal{M}$ to learn a good model without remembering $\mathcal{D}$?
	        \item Regularization can be seen as a form of constraining the learning process
	        \item Goal of regularization: Don't learn noise but only the important patterns and concepts
	    \end{itemize}
	\end{frame}
	
	\begin{frame}[c]{Idea: AutoML}
	    
	    \begin{itemize}
	        \item Problem: How strong should we regularize $\mathcal{M}$?
	        \bigskip
	        \item Regularization comes with a hyperparameter that needs to be tuned for your dataset at hand
	        \begin{itemize}
	            \item Too strong regularization can hinder learning at all
	            \item On very noisy datasets you want more regularization than on less noisy datasets
	        \end{itemize}
	        \item Challenge: How to find the regularization strength on a new dataset?
	        \item AutoML helps to answer such questions in an automatic, systematic and efficient way
	    \end{itemize}
	\end{frame}
	
\end{document}