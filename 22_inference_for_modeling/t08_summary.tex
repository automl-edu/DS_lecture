\documentclass[aspectratio=169]{../latex_main/tntbeamer}  % you can pass all options of the beamer class, e.g., 'handout' or 'aspectratio=43'
\input{../latex_main/preamble}

\title[Introduction]{DS: Inference for Modeling}
\subtitle{Summary}

\graphicspath{ {./figure/} }
%\institute{}


\begin{document}
	
	\maketitle
	\begin{frame}{Summary}
	    \begin{itemize}
	        \item Estimators are functions that provide estimates of true population parameters.
	        \item We can bootstrap to estimate the sampling distribution of an estimator.
	        \item Using this bootstrapped sampling distribution, we can compute a confidence interval for our estimator.
	        \begin{itemize}
	            \item This gives us a rough idea of how uncertain we are about the true population parameter.
	            \item Only valid if the original random sample is representative.
	        \end{itemize}
	        \item The assumption when performing linear regression is that there is some true parameter theta that defines a linear relationship between features X and response y.
	        \begin{itemize}
	            \item We can use the bootstrap to determine whether or not an individual feature is significant.
	        \end{itemize}
	        \item Multicollinearity arises when features are correlated with one another.
	    \end{itemize}
	\end{frame}
	
	
	
	\begin{frame}{What’s next}
	    \begin{columns}
	        \begin{column}{.5\textwidth}
	                \begin{itemize}
	                    \item This lecture was a (brief) diversion from the “ML” perspective.
	                    \begin{itemize}
	                        \item ML: Make accurate predictions.
	                        \item Statistics: Infer about a population.
	                    \end{itemize}
	                    \item So far, we’ve covered supervised learning in great detail.
	                    \begin{itemize}
	                        \item Linear regression, logistic regression, decision trees / random forests.
	                    \end{itemize}
	                    \item We will now spend some time talking about unsupervised learning.
	                    \begin{itemize}
	                        \item Dimensionality reduction, PCA.
	                        \item Clustering.
	                    \end{itemize}
	                \end{itemize}
	        \end{column}
	        
	        
	        \begin{column}{.5\textwidth}
	                \begin{figure}
	                    \centering
	                    \includegraphics[scale=.45]{Bild14}
	                \end{figure}
	        \end{column}
	    \end{columns}
	\end{frame}
\end{document}