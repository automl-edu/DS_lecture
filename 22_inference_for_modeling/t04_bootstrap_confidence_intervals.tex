\documentclass[aspectratio=169]{../latex_main/tntbeamer}  % you can pass all options of the beamer class, e.g., 'handout' or 'aspectratio=43'
\input{../latex_main/preamble}

\title[Introduction]{DS: Inference for Modeling}
\subtitle{Bootstrap confidence intervals}

\graphicspath{ {./figure/} }
%\institute{}


\begin{document}
	
	\maketitle
	\begin{frame}{Confidence intervals}
	    \begin{itemize}
	        \item Intuition: Estimate an interval where we think the population parameter is, based on the center and variance of the estimator.
	        \item What does a P\% confidence interval mean?
	        \begin{itemize}
	            \item Imagine the following procedure:
	            \begin{itemize}
	                \item Take a sample from the population.
	                \item Compute P\% confidence interval for the true population parameter, somehow.
	            \end{itemize}
	            \item If we repeat this procedure many times, the population parameter will be in our interval P\% of the time, in the long run.
	        \end{itemize}
	    \end{itemize}
	\end{frame}
	
	
	\begin{frame}{Confidence intervals}
	    An estimator f exists in order to guess the value of an unknown parameter $\theta^*$.
        An estimator ci for a P\% confidence interval for f is a function that takes a sample and returns an interval. This interval will (ideally) contain $\theta^*$ for P\% of samples.
        \begin{columns}
            \begin{column}{.5\textwidth}
                    \begin{figure}
                        \centering
                        \includegraphics[scale=.35]{Bild8}
                    \end{figure}
            \end{column}
        
        
             \begin{column}{.5\textwidth}
             \\
             \bigskip
             \bigskip
                How do we compute ci(s,f,P)?
                    \begin{itemize}
                        \item Approximate the sampling distribution of f using the sample s.
                        \item Choose the middle P\% of samples from this approximate distribution.
                    \end{itemize}
            \end{column}
        
        \end{columns}
        
	\end{frame}
	
	
	
	\begin{frame}{Bootstrap confidence intervals}
	    An estimator ci for a P\% confidence interval for f is a function that takes a sample and returns an interval. This interval will (ideally) contain $\theta^*$ for P\% of samples.
	    \begin{figure}
	        \centering
	        \includegraphics[scale=.35]{Bild9}
	    \end{figure}
	\end{frame}
	
	
	\begin{frame}{Confidence intervals}
	    \begin{itemize}
	        \item The confidence level is a statement about the procedure used to create our interval.
	        \item It is not the case that a 95\% confidence interval means “there is a 95\% chance that the population parameter is in our interval”.
	        \begin{itemize}
	            \item The population parameter is fixed.
	            \item Our interval is fixed. The population parameter is either in it, or it isn’t.
	            \item Nothing random here!
	        \end{itemize}
	        \item The confidence intervals we’ve created are sometimes called percentile bootstrap confidence intervals.
	        \begin{itemize}
	            \item There are other methods of creating confidence intervals.
	            \begin{itemize}
	                \item E.g. assume that the sampling distribution of your estimator is normal.
	            \end{itemize}
	        \end{itemize}
	    \end{itemize}
	\end{frame}
\end{document}