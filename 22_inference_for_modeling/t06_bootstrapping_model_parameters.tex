\documentclass[aspectratio=169]{../latex_main/tntbeamer}  % you can pass all options of the beamer class, e.g., 'handout' or 'aspectratio=43'
\input{../latex_main/preamble}

\title[Introduction]{DS: Inference for Modeling}
\subtitle{Bootstrapping model parameters}

\graphicspath{ {./figure/} }
%\institute{}


\begin{document}
	
	\maketitle
	\begin{frame}{Parameter estimates}
	    \begin{itemize}
	        \item Our estimate for     $\theta^*$    depends on what our training data was.
	        \begin{itemize}
	            \item Different training data, different   $\hat{\theta}$   ! \hspace{2cm}  $\hat{\theta} = (\mathbb{X}^T\mathbb{X})^{-1}\mathbb{X}^T\mathbb{Y}$
	        \end{itemize}
	        \item We want to think about all of the different ways that our training data, and hence our parameter estimate, could have come out.
	        \item Easy!
	        \begin{itemize}
	            \item Bootstrap our training data.
	            \item Fit a linear model to each resample.
	            \item Look at the resulting distribution of bootstrapped parameter estimates.
	        \end{itemize}
	    \end{itemize}
	\end{frame}
	
	
	\begin{frame}{Assessing the quality of our model}
	    \begin{itemize}
	        \item Suppose we fit a linear regression model with p features, plus an intercept term.
	    \end{itemize}
	    \begin{align*}
	        &y = f_{\theta^*}(x) + \epsilon = \theta_0^* + \sum\limits_{j=1}^P \theta_j^*x_j + \epsilon
	        \includegraphics[scale=.25]{Bild12}\\
	        &\hat{y} = f_{\theta^*}(x) = \hat{\theta}_0 + \sum\limits_{j=1}^P\hat{\theta}_jx_j
	        \includegraphics[scale=.25]{Bild13}
	    \end{align*}
	    \begin{itemize}
	        \item If the true    $\theta^*_1$   is 0, then the feature $x_1$   has no effect on the response.
	        \item How can we test whether or not         $\theta^*_1$ = 0       ?
	    \end{itemize}
	\end{frame}
	
	
	
	\begin{frame}{Confidence interval for true slope}
	    \begin{itemize}
	        \item We want to test whether   $\theta^*_1$    is 0.
	        \item We get one estimate  $\hat{\theta}_1$    from our sample.
	        \item But we must imagine all the other ways the random sample could have come out.
	        \item If the sample is large – bootstrap it! 
	        \begin{itemize}
	            \item Estimate   $\theta^*_1$     each time.
	            \item Make a confidence interval for    $\theta^*_1$     and see if 0 is in the interval.
	            \begin{itemize}
	                \item If yes:   $\theta^*_1$     is not significantly different than 0.
	                \item If no:   $\theta^*_1$     is significantly different than 0.
	                \item Can formalize with the language of hypothesis testing, but won’t do so here.
	            \end{itemize}
                \item Works for linear (and logistic!) regression models with any number of features.
	        \end{itemize}
	    \end{itemize}
	\end{frame}
	
	
\end{document}