\documentclass[aspectratio=169]{../latex_main/tntbeamer}  % you can pass all options of the beamer class, e.g., 'handout' or 'aspectratio=43'
\input{../latex_main/preamble}

\title[Introduction]{DS: Inference for Modeling}
\subtitle{Multicollinearity}

\graphicspath{ {./figure/} }
%\institute{}


\begin{document}
	
	\maketitle
	\begin{frame}{The meaning of “slope”}
	    Consider the equation
	    \begin{align*}
	        y = a_0 + a_1x_1 + a_2x_2 + ... + a_px_p 
	    \end{align*}
	    \begin{itemize}
	        \item The slope $a_1$ measures the change in y per unit change in $x_1$, assuming all other variables are held constant.
	        \item We use an equation of the above form for linear regression.
	        \item But what if we can’t hold all other variables constant?
	    \end{itemize}
	\end{frame}
	
	
	\begin{frame}{Multicollinearity}
	    \begin{itemize}
	        \item If features are related to each other, it might not be possible to have a change in one of them while holding the others constant.
	        \begin{itemize}
	            \item Then, the individual slopes will have no meaning.
	        \end{itemize}
	        \item Multicollinearity: when a feature can be predicted fairly accurately by a linear combination of other features.
	        \begin{itemize}
	            \item Slopes can’t be interpreted.
	            \item Small changes in the data can lead to big changes in the slopes.
	            \item Doesn’t impact the predictive capability of our model – only impacts interpretability.
	        \end{itemize}
	        \item Perfect multicollinearity: one feature can be written exactly as a linear combination of other features.
	        \begin{itemize}
	            \item Design matrix isn’t full rank! Can’t find unique   $\hat{\theta}$   .
	            \begin{itemize}
	                \item For instance, one-hot encoding with an intercept term. 
	            \end{itemize}
	        \end{itemize}
	    \end{itemize}
	\end{frame}
	
\end{document}