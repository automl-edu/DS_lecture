\documentclass[aspectratio=169]{../latex_main/tntbeamer}  % you can pass all options of the beamer class, e.g., 'handout' or 'aspectratio=43'
\input{../latex_main/preamble}

\title[Regression]{DS: Ordinary Least Squares}
\subtitle{Summary}

\graphicspath{ {./figure_ols/} }
%\institute{}


\begin{document}
	
	\maketitle
	\begin{frame}{Summary}
	    \begin{itemize}
	        \item We defined the multiple linear regression model in terms of matrices.
	        \begin{equation*}
	            \hat{\vect{Y}} = \vect{X}\vect{\theta}
	        \end{equation*}
	        \item We used a geometric argument to derive the optimal parameter vector   $\hat{\vect{\theta}}$  , that minimizes average squared loss. This value is called the least squares estimate.
	        \begin{equation*}
	            \hat{\vect{\theta}} = (\vect{X}^\intercal\vect{X})^{-1} \vect{X}^\intercal\vect{Y}
	        \end{equation*}
	        \item We discussed residuals and their properties.
	        \item We explored when a unique solution for   $\hat{\vect{\theta}}$     exists, and when one does not.
	    \end{itemize}
	\end{frame}
	
	
	\begin{frame}{Moving forward}

	    \begin{itemize}
	        \item Feature engineering.
	        \begin{itemize}
	            \item The process of extracting and creating more sophisticated features from our data.
	        \end{itemize}
	        \item The Bias-Variance tradeoff.
	        \begin{itemize}
	            \item Discussing this idea of a “true model,” and where the errors in our predictions come from.
	        \end{itemize}
	        \item Regularization and Cross-Validation.
	        \begin{itemize}
	            \item Making our models more “general”, by updating our objective function.
	        \end{itemize}
	        \item Classification
	        \begin{itemize}
	            \item What if we want to predict categories (1 and 0), instead of numbers?
	        \end{itemize}
	    \end{itemize}
	\end{frame}
	
\end{document}