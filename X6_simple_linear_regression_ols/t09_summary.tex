\documentclass[aspectratio=169]{../latex_main/tntbeamer}  % you can pass all options of the beamer class, e.g., 'handout' or 'aspectratio=43'
\input{../latex_main/preamble}

\title[Regression]{DS: Simple Linear Regression}
\subtitle{Mid-Summary}

\graphicspath{ {./figure/} }
%\institute{}


\begin{document}
	
	\maketitle
	\begin{frame}{Summary}
	    \begin{itemize}
	        \item We now know of three models, $\hat{y} = f_\theta (x)$             .
	        \begin{itemize}
	            \item The constant model,  $f_\theta (x) = \theta$ 
                \item The simple linear regression model  $f_\theta (x) = \theta_0 +  \theta_1x$ 
                \item The multiple linear regression model	 $f_\theta (x) = \theta_0 + \theta_1 x_1 + \theta_2 x_2 + ... + \theta_p x_p$
                \item A model with optimal parameters is denoted $f_\hat{\theta} (x)$
	        \end{itemize}
            
            \item We looked at the correlation coefficient, $r$, and studied its properties.
	        \item We solved for the optimal parameters for the simple linear model by hand, by minimizing average squared loss (MSE).
	    \end{itemize}
	    \begin{align*}
	        \hat{\theta}_1 = r \frac{\sigma_y}{\sigma_x} \hspace{1cm} \hat{\theta}_0 = \bar{y} - \hat{\theta}_1 \bar{x}
	    \end{align*}
	    \begin{itemize}
	        \item We introduced the notion of a feature, and how we can have multiple in our models.
	        \item We discussed the multiple R² coefficient and RMSE as methods of evaluating the quality of a linear model
	    \end{itemize}
	\end{frame}
	
	
	
% 	\begin{frame}{Next time}
% 	    In the next lecture, we will...
% 	    \begin{itemize}
% 	        \item Express the multiple linear regression model using matrix-vector notation.
% 	        \item Explicitly solve for the optimal parameters.
% 	        \begin{itemize}
% 	            \item Thus far, we’ve done this by hand for the constant and simple linear models.
%                 \item The process of solving for the optimal parameters will inform us of several properties of linear models!
%                 \item There will be a lot of linear algebra!
% 	        \end{itemize}
%             \item Look at residuals and their properties.
% 	        \item Think about what it means for a model to be “linear.”
% 	    \end{itemize}
% 	\end{frame}
\end{document}