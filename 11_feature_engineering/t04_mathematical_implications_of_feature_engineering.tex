\documentclass[aspectratio=169]{../latex_main/tntbeamer}  % you can pass all options of the beamer class, e.g., 'handout' or 'aspectratio=43'
\input{../latex_main/preamble}

\title[Introduction]{DS: Feature Engineering}
\subtitle{Mathematical Implications of Feature Engineering}

\graphicspath{ {./figure/} }
%\institute{}


\begin{document}
	
	\maketitle
	\begin{frame}[c]{Revisiting the Normal Equations}
	    Recall that fitting an OLS model requires solving the normal equations
        \begin{equation*}
            (X^TX)\theta = X^Ty
        \end{equation*}
        which can be updated using the new design matrix to
        \begin{equation*}
            (\Phi^T\Phi)\theta = \Phi^Ty
        \end{equation*}
        Since feature engineering changes the $\Phi$ matrix, feature engineering can influence the solution to the normal equations.
	\end{frame}
	
	
	\begin{frame}[c]{Linearly dependent features}
	    Recall the solution to the normal equations: $\theta^* = (\Phi^T\Phi)^{-1}\Phi^Ty$
        \begin{itemize}
            \item This solution exists only when $(\Phi^T\Phi)^{-1}$ exists
            \item $(\Phi^T\Phi)^{-1}$  exists only when $\Phi$ is full rank (i.e. all columns are linearly independent)
            \begin{itemize}
                \item The proof is outside the scope of this class
            \end{itemize}
            \item If $\Phi$ is not full rank then there are infinite solutions to the normal equations
            \begin{itemize}
                \item This is bad because model parameters are unstable
            \end{itemize}
        \end{itemize}
        A simple example of linearly dependent features is having height in both inches and cm\\
        \bigskip
        This issue of linear dependence is a problem for one of the feature engineering techniques we saw earlier in the lecture. Can you identify which one?
	\end{frame}
	
	
	\begin{frame}[c]{Linearly dependent features}
	    Recall the solution to the normal equations: $\theta^* = (\Phi^T\Phi)^{-1}\Phi^Ty$
        \begin{itemize}
            \item This solution exists only when $(\Phi^T\Phi)^{-1}$ exists
            \item $(\Phi^T\Phi)^{-1}$  exists only when $\Phi$ is full rank (i.e. all columns are linearly independent)
            \begin{itemize}
                \item The proof is outside the scope of this class
            \end{itemize}
            \item If $\Phi$ is not full rank then there are infinite solutions to the normal equations
            \begin{itemize}
                \item This is bad because model parameters are unstable
            \end{itemize}
        \end{itemize}
        A simple example of linearly dependent features is having height in both inches and cm\\
        \bigskip
        This issue of linear dependence is a problem for one of the feature engineering techniques we saw earlier in the lecture. Can you identify which one?
	\end{frame}
	
	
	\begin{frame}{One-Hot Encoding and Linear Dependence}
	    \begin{figure}
	        \centering
	        \includegraphics[scale=.35]{Bild8}
	    \end{figure}
	    Notice that co\_appl + co\_sam = bias! This means the columns are linearly dependent
	    \begin{itemize}
	        \item Solution: Drop one of the one-hot encoded columns per variable
	    \end{itemize}
	\end{frame}
	
	
	
	\begin{frame}{Too Many Features}
	    If you add too many features, the normal equations will have infinite solutions\\
	    \bigskip
        The normal equations can be thought of as a system of equations with N equations and P unknown quantities to solve for
	    \begin{itemize}
	        \item N: \# of data points
	        \item P: \# of parameters
	    \end{itemize}
	    \bigskip
	    If P > N, you have more unknowns than equations so there can be no unique solution\\
	    \bigskip
	    Additionally, too many features can cause overfitting, which will be covered in future lectures
	\end{frame}
\end{document}