\documentclass[aspectratio=169]{../latex_main/tntbeamer}  % you can pass all options of the beamer class, e.g., 'handout' or 'aspectratio=43'
\input{../latex_main/preamble}

\title[Introduction]{DS: Feature Engineering}
\subtitle{Feature Functions}

\graphicspath{ {./figure/} }
%\institute{}


\begin{document}
	
	\maketitle
	\begin{frame}{Expressing Feature Engineering Mathematically}
	    We encapsulate the feature engineering process as a function $\phi$ that transforms the raw data x into the features ultimately used for the model\\
	    \bigskip
	    We call $\phi$ a feature function because it maps data vectors to feature vectors\\
	    \bigskip
	    Example: Intercept term in Simple Linear Regression
        \begin{equation*}
            \hat{y} = \theta_0 + \theta_1x = \left[\begin{array}{c}
                 1\\
                 x
            \end{array}\right] \cdot \left[\begin{array}{c}
                 \theta_0\\
                 \theta_1
            \end{array}\right] = \phi(x)^T\theta
        \end{equation*}
        
        \begin{equation*}
            x \rightarrow \phi (x) \rightarrow \left[\begin{array}{c}
                 1\\
                 x
            \end{array}\right]
        \end{equation*}
	\end{frame}
	
	
	\begin{frame}{Expressing Feature Engineering Mathematically}
	    Feature functions can be arbitrarily complex (i.e. they can encompass multiple feature engineering operations)\\
	    \bigskip
	    Example: MPG dataset
        \begin{figure}
            \centering
            \includegraphics[scale=.4]{Bild7}
        \end{figure}
        This feature function takes in a DataFrame and outputs another DataFrame with 5 times the number of features
	\end{frame}
	
	
	
	\begin{frame}[c]{Revisiting the Design Matrix}
	    The feature function $\phi$ transforms one data vector into one feature vector\\
	    \bigskip
	    We have multiple data vectors $x_1,x_2,...,x_n$ so we need to apply $\phi$ to each data vector\\
        \begin{equation*}
            X = \left[\begin{array}{c}
                 x_1\\
                 x_2\\
                 \vdots\\
                 x_n
            \end{array}\right] \rightarrow \phi \rightarrow \left[\begin{array}{c}
                 \phi(x_1)\\
                 \phi(x_2)\\
                 \vdots\\
                 \phi(x_n)
            \end{array}\right] = \Phi
        \end{equation*}
        In the previous lecture, the X matrix was called the design matrix\\
        After applying the feature functions, the design matrix is called $\phi$
	\end{frame}
\end{document}