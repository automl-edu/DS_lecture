\documentclass[aspectratio=169]{../latex_main/tntbeamer}  % you can pass all options of the beamer class, e.g., 'handout' or 'aspectratio=43'
\input{../latex_main/preamble}

\title[Introduction]{DS: Logistic Regression, Part 1}
\subtitle{Regression vs. Classification}

\graphicspath{ {./figure/} }
%\institute{}


\begin{document}
	
	\maketitle
	\begin{frame}{Linear Regression}
	    In a linear regression model, our goal is to predict a quantitative variable (i.e., some real number) from a set of features.
	    \begin{itemize}
	        \item Our output, or response, y, could be any real number.
	        \item We determined optimal model parameters by minimizing some average loss, and (sometimes) an added regularization penalty.
	    \end{itemize}
	    \begin{equation*}
	        \hat{y} = f_\theta (x) = x^T\theta
	    \end{equation*}
	    Remember, $x^T\theta = \theta_0 + \theta_1x_1 + \theta_2x_2 + ... + \theta_px_p$ 
	\end{frame}
	
	
	\begin{frame}{Classification}
	    When performing classification, we are instead interested in predicting some categorical variable.
	    \begin{figure}
	        \centering
	        \includegraphics[scale=.35]{Bild1}
	    \end{figure}
	\end{frame}
	
	
	\begin{frame}[c]{Classification}
	    \begin{itemize}
	        \item Binary classification: two classes.
	        \begin{itemize}
	            \item Examples: spam / not spam. 
	            \item Our responses are either 0 or 1.
	            \item Our focus today.
	        \end{itemize}
	        \item Multiclass classification: many classes.
	        \begin{itemize}
	            \item Examples: Image labeling (cat, dog, car), next word in a sentence, etc.
	        \end{itemize}
	    \end{itemize}
	    This is not the first time you are seeing classification!
	    \begin{itemize}
	        \item k-Nearest Neighbors was a classification technique you learned in Data 8.
	        \begin{itemize}
	            \item Won’t cover it in Data 100.
	        \end{itemize}
	    \end{itemize}
	\end{frame}
	
	
	\begin{frame}[c]{Machine learning taxonomy}
	    \begin{columns}
	        \begin{column}{.5\textwidth}
	        \\\bigskip\bigskip\bigskip
	                Regression and Classification are both forms of supervised learning. \\
	                \bigskip
	                Logistic regression, the topic of this lecture, is mostly used for classification, even though it has “regression” in the name. 
	        \end{column}
	        
	        
	        \begin{column}{.5\textwidth}
	                \begin{figure}
	                    \centering
	                    \includegraphics[scale=.3]{Bild2}
	                \end{figure}
	        \end{column}
	    \end{columns}
	\end{frame}

\end{document}