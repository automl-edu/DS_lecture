\documentclass[aspectratio=169]{../latex_main/tntbeamer}  % you can pass all options of the beamer class, e.g., 'handout' or 'aspectratio=43'
\input{../latex_main/preamble}

\title[Introduction]{DS: Data Sampling and Probability}
\subtitle{How to sample effectively, and how to quantify the samples we collect.}

\graphicspath{ {./figure/} }
%\institute{}


\begin{document}
	
	\maketitle
	
	\begin{frame}{Roadmap}
	    \begin{itemize}
	        \item Formalizing various ideas that pertain to sampling.
	        \begin{itemize}
	            \item Why we need to sample in the first place.
	            \item What it means for our sample to be biased.
	            \item How to prevent these biases in our samples.
	            \item What exactly a sampling frame is, and why choosing a good one is important.
	        \end{itemize}
	        \item Learning how to compute probabilities from samples.
	        \begin{itemize}
	            \item This will be continued into Lecture 3.
	        \end{itemize}
	    \end{itemize}
	\end{frame}
	
	\begin{frame}[c]{Data Science Lifecycle}
	    \begin{columns}
	        \begin{column}{.4\textwidth}
	        \begin{figure}
	            %\centering
	            \includegraphics[scale=.6]{Bild1}
	        \end{figure}
	        \end{column}
	            
	        \begin{column}{.4\textwidth}
	        \bigskip
	        \bigskip
	        \bigskip
	        \bigskip
	            \centering
	            \begin{align*}
	                \text{We call this the} \\
	                \textbf{Data Science Lifecycle}
	            \end{align*}
	            
	        \end{column}

	    \end{columns}
	\end{frame}
	
	\begin{frame}{Question}
	    \begin{columns}
	        \begin{column}{.4\textwidth}
	        \bigskip
	        \bigskip
	        \bigskip
	        \bigskip
	         \centering
	            \begin{align*}
	                \text{Question: How many squirrels are there} \\
	                \text{in Central Park, New York City?}
	            \end{align*}
	       
	        \end{column}
	            
	        \begin{column}{.4\textwidth}
	            
	        \begin{figure}
	            %\centering
	            \includegraphics[scale=.45]{Bild2}
	        \end{figure}
	            
	        \end{column}
	        
	        \end{columns}
	\end{frame}
	
	
	\begin{frame}
	    \includegraphics[scale=.6]{Bild3}
	\end{frame}
	
	\begin{frame}
	    \includegraphics[scale=.6]{Bild4}
	\end{frame}
	
	
	\begin{frame}
	    \includegraphics[scale=.37]{Bild5}
	\end{frame}
	
	
	
\end{document}