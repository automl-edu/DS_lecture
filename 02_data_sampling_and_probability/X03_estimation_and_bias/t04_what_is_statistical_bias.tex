\documentclass[aspectratio=169]{../latex_main/tntbeamer}  % you can pass all options of the beamer class, e.g., 'handout' or 'aspectratio=43'
\input{../latex_main/preamble}

\title[Statistics]{DS: Estimation and Bias}
\subtitle{What is statistical bias?}

\graphicspath{ {./figure_bias/} }
%\institute{}


\begin{document}
	
	\maketitle
	
	\begin{frame}[c]{Where we’re headed today}
	    What is statistical bias?\\
	    \hspace{4cm} The difference between your estimate and the truth.

	\end{frame}
	
		\begin{frame}{Interlude}
	    \begin{columns}
	        \begin{column}{.5\textwidth}
	     
	           \begin{figure}
	               \includegraphics[scale=.55]{Bild24}
	           \end{figure}
	        \end{column}
	        
	        \begin{column}{.3\textwidth}
	            Plato’s Allegory of the Cave\\
	            \begin{itemize}
	                \item World of forms
	                \begin{itemize}
	                    \item Non-physical essence of all things
	                \end{itemize}
	                \item World of representation
	                \begin{itemize}
	                    \item The material world that we observe
	                \end{itemize}
	                \item Philosopher
	                \begin{itemize}
	                    \item Person who seeks knowledge of forms           
	                \end{itemize}
	            \end{itemize}
	        \end{column}
	        \end{columns}
	    
	\end{frame}
	
	
		\begin{frame}{Interlude}
	    \begin{columns}
	        \begin{column}{.5\textwidth}
	     
	           \begin{figure}
	               \includegraphics[scale=.75]{Bild23}
	           \end{figure}
	        \end{column}
	        
	        \begin{column}{.4\textwidth}
	            Metaphor of the Cave\\
	            \begin{itemize}
	                \item World of Parameters
	                \begin{itemize}
	                    \item Constants that define the structure of the world
	                \end{itemize}
	                \item World of Data / Statistics
	                \begin{itemize}
	                    \item Observable information generated by RVs and their parameters
	                    \item Statistic: numeric summary of data
	                \end{itemize}
	                \item Statistician
	                \begin{itemize}
	                    \item Person who uses statistics to learn about parameters
	                \end{itemize}
	            \end{itemize}
	        \end{column}
	        \end{columns}
	    
	\end{frame}
	
	
	\begin{frame}{What is a statistic?}
	    \begin{itemize}
	        \item A single piece of data
	        \item A numerical summary of a dataset
	        \begin{itemize}
	            \item function of realizations of RVs
	        \end{itemize}
	    \end{itemize}
	    \bigskip
	    \hspace{3cm}$\hat\theta = f(x_1,x_2,...,x_n)$\\
	    \hspace{3cm}What is an estimator?
        \begin{itemize}
            \item A statistic designed to estimate a parameter
        \end{itemize}
	\end{frame}
	
	
	\begin{frame}{Choosing a statistic/estimator}
	    \begin{columns}
	        \begin{column}{.4\textwidth}
	            Example 1: Squirrels\\
	           \begin{figure}
	               \includegraphics[scale=.6]{Bild25}
	               
	           \end{figure}
	        \end{column}
	        
	        \begin{column}{.5\textwidth}
	            \begin{figure}
                \includegraphics[scale=.33]{Bild26}
	                
	            \end{figure}
	        \end{column}
	        \end{columns}

	    
	\end{frame}
	
	
	\begin{frame}{Choosing a statistic/estimator}
	    \begin{columns}
	        \begin{column}{.4\textwidth}
	            Example 2: FDR vs. Landon\\
	            \bigskip
	            Question: what proportion of Americans will vote for FDR?\\
	           \begin{figure}
	               \includegraphics[scale=.3]{Bild27}
	           \end{figure}
	        \end{column}
	        
	        \begin{column}{.5\textwidth}
	            Parameter: The total proportion of votes for FDR, p\\
	            The Data: $X_1, X_2, ..., x_{10M}$\\
	            The Estimator:
                \begin{align*}
                    \hat{p} &= f(X_1, X_2,...,X_{10M};n)\\
                    &= (X_1, X_2,...,X_{10M})/n
                \end{align*}
	        \end{column}
	        \end{columns}

	    
	\end{frame}
	
	\begin{frame}[c]{What is statistical bias?}
	    \centering
	   The difference between your estimate and the truth.\\
        $\hat\theta - \theta$\\
        \bigskip
        Not quite...

	    
	\end{frame}
	
	
	\begin{frame}{Don’t forget about sampling variability}
	    \begin{columns}
	        \begin{column}{.4\textwidth}
	           Example: The total number of heads in a series of 5 coin flips: $Y = X_1 + X_2 + … + X_5$\\
	           \begin{figure}
	               \includegraphics[scale=.5]{Bild28}
	           \end{figure}
	        \end{column}
	        
	        \begin{column}{.5\textwidth}
	           \begin{figure}
	               \includegraphics[scale=.5]{Bild29}
	           \end{figure}
	           \begin{itemize}
	               \item Y is an RV, therefore $\hat{p}$ is an RV
	               \item Since estimators are functions of RVs, they are RVs -> subject to sampling variability
	           \end{itemize}
	        \end{column}
	        \end{columns}

	    
	\end{frame}
	
	
	\begin{frame}{What is statistical bias?}
	    \begin{columns}
	        \begin{column}{.4\textwidth}
	           The difference between your estimate and the truth.
	           \begin{figure}
	               \includegraphics[scale=.31]{Bild30}
	           \end{figure}
	           Q: What if your estimator isn’t great?\\
	           A: Biased estimator\\
	           Ex. $ \hat{p_b}= Y_n / (n - 1)$

	        \end{column}
	        
	        \begin{column}{.4\textwidth}
	           \begin{figure}
	               \includegraphics[scale=.5]{Bild31}
	           \end{figure}
	           Q: What if the data wasn’t generated by $\theta$ ?\\
	           \bigskip
	           A: It will not be representative of the population
                \begin{itemize}
                    \item Selection bias
                \end{itemize}

	        \end{column}
	        \end{columns}

	    
	\end{frame}
	
	
	\begin{frame}[c]{What’s Next?}
	    \begin{figure}
	        \includegraphics[scale=.5]{Bild1}
	    \end{figure}   
	\end{frame}
\end{document}