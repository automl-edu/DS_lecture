\documentclass[aspectratio=169]{../latex_main/tntbeamer}  % you can pass all options of the beamer class, e.g., 'handout' or 'aspectratio=43'
\input{../latex_main/preamble_2}

\title[Bias II]{DS: Data Sampling and Probability}
\subtitle{Bias II}

\graphicspath{ {./figure/} }
%\institute{}


\begin{document}
	
	\maketitle
	
% Slide 62: What is statistical bias?
\begin{frame}{What is statistical bias?}
    The difference between your estimate and the truth?\\

$$\hat{\theta} - \theta$$

(where $\hat{\theta}$ is our estimated value and $\theta$ is the ground truth)

    \bigskip
    ... Not quite...
\end{frame}

% Slide 63: Don’t forget about sampling variability
\begin{frame}{Don’t forget about sampling variability}
    \textbf{Example:} The total number of heads in a series of 5 coin flips: 
    \begin{equation}
    Y = X_1 + X_2 + \dots + X_5
    \end{equation}
    \begin{itemize}
        \item $Y$ is a random variable, therefore $\hat{p}$ is a random variable.
        \item Since estimators are functions of random variables, they are random variables and are subject to sampling variability.
    \end{itemize}
\end{frame}

% Slide 64: What is statistical bias?
\begin{frame}{What is statistical bias?}
    The difference between your estimate and the truth:\\

    $$\mathbb{E}(\hat{\theta}) - \theta$$

    \textbf{Q:} What if your estimator isn’t great?\\
    \textbf{A:} Biased estimator.
    \begin{itemize}
        \item \textbf{Example:} $\hat{p}_b = \frac{Y_n}{n - 1}$
    \end{itemize}

    \bigskip

    \textbf{Q:} What if the data wasn’t generated by $\theta$?\\
    \textbf{A:} It will not be representative of the population. 
    \begin{itemize}
        \item \textbf{Selection bias:} A systematic bias caused by non-random sample selection.
    \end{itemize}
\end{frame}
	
\end{document}