\documentclass[aspectratio=169]{../latex_main/tntbeamer}  % you can pass all options of the beamer class, e.g., 'handout' or 'aspectratio=43'
\input{../latex_main/preamble_2}

\title[Binomial and Multinomial]{DS: Data Sampling and Probability}
\subtitle{Binomial and multinomial probabilities}

\graphicspath{ {./figure/} }
%\institute{}


\begin{document}
	
	\maketitle
	
	
	\begin{frame}{The scenario}
	Binomial and multinomial probabilities arise when we:
	\begin{itemize}
	    \item Sample at random, with replacement.
	    \item Sample a fixed number (n) times.
	    \item Sample from a categorical distribution.
	    \begin{itemize}
	        \item For example, a bag of marbles in which 60\% are blue and 40\% are not blue.
	        \item Or where 60\% are blue, 30\% are green, and 10\% are red.
	    \end{itemize}
	    \item Want to count the number of each category that end up in our sample.
	\end{itemize}

	\end{frame}
	
	
	\begin{frame}{Two categories}
	Suppose we sample at random with replacement 7 times from a bag of marbles, 60\% of which are blue and 40\% of which are not blue.

	\begin{itemize}
	    \item What is P(bnbbbnn)?
	    \begin{itemize}
	        \item By the product rule, since the sample is drawn with replacement:
            \begin{align*}
                P(bnbbbnn) &= 0.6 \times 0.4 \times 0.6 \times 0.6 \times 0.6 \times 0.4 \times 0.4 \\
                &= (0.6)^4 (0.4)^3
            \end{align*}
	    \end{itemize}
	    \item How does P(4 blue, 3 not blue) compare to P(bnbbbnn)?
	    \begin{itemize}
	        \item P(4 blue, 3 not blue) > P(bnbbbnn).
	        \item Why? bnbbbnn is far more restrictive and specific than “4 blue, 3 not blue.”
	        \item There are several other ways to get “4 blue, 3 not blue” (for instance, bnnnbbb).
	    \end{itemize}
	\end{itemize}

	\end{frame}
	
	
		\begin{frame}{Binomial probabilities...}
	“4 blue, 3 not blue” can occur in several equally likely ways.\\                     For instance, P(bnbbbnn) = P(bnbbbnn) = P(bnbbbnn) = ... = $(0.6)^4$ $(0.4)^3$. \\         \bigskip
	    
	    P(4 blue, 3 not blue) is the total chance of all of those ways. The number of ways in which we can draw 4 blue marbles and 3 not blue marbles is (i.e., all possible permutations!)

        \begin{align*}
            \binom{7}{4} = \frac{7!}{4!3!}
        \end{align*}
    and thus,
    \begin{align*}
        \text{P(4 blue, 3 not blue)} = \binom{7}{4}(0.6)^4 (0.4)^3 = \frac{7!}{4!3!}(0.6)^4 (0.4)^3
    \end{align*}

	\end{frame}

\begin{frame}{Multinomial probabilities}

Suppose we again sample at random with replacement 7 times from a bag of
marbles, but this time, 60\% of marbles are blue, 30\% are green, and 10\% are red.

\begin{itemize}
    \item What is P(bgbbbgr)?
    \begin{itemize}
        \item Following the same steps as before:\\
        $ P(bgbbbgr) = 0.6 \cdot 0.3 \cdot 0.6 \cdot 0.6 \cdot 0.6 \cdot 0.3 \cdot 0.1 = 0.6^4 \cdot 0.3^2 \cdot 0.1$
    \end{itemize}
    \item What is P(4 blue, 2 green, 1 red)?
    \begin{itemize}
        \item As we saw before, we multiply the above probability by the total number of ways to draw 4 blue, 2 green, and 1 red marbles. This gives\\
P(4 blue, 2 green, 1 red) $= \frac{7!}{4!2!1!} \cdot  0.6^4 \cdot 0.3^2 \cdot 0.1$
    \end{itemize}
\end{itemize}
    
\end{frame}

\begin{frame}[c]{Generalization of binomial probabilities}

If we are drawing at random with replacement $n$ times, from a population in which a proportion $p$ of the individuals are called “successes” (and the remaining $1 - p$ are
“failures”), then the probability of $k$ successes (and hence, $n - k$ failures) is

$$P(\text{k successes}) = \binom{n}{k}p^k (1-p)^{n-k}$$

\begin{itemize}
    \item Note: since the number of successes and the number of failures we draw must
sum to $n$, saying “$k$ successes” is equivalent to saying “$k$ successes and $n - k$ failures.”
\begin{itemize}
    \item For instance, if you flip a coin $10$ times and see $6$ heads, we know there were $4$ tails
\end{itemize}

\end{itemize}

\end{frame}

\begin{frame}[c]{Generalization of Multinomial Probabilities}

If we are drawing at random with replacement n times, from a population broken into three separate categories (where $p1 + p2 + p3 = 1$):

\begin{itemize}
    \item Category 1, with proportion $p_1$ of the individuals.
    \item Category 2, with proportion $p_2$ of the individuals.
    \item Category 3, with proportion $p_3$ of the individuals.
\end{itemize}

Then, the probability of drawing $k_1$ individuals from Category 1, $k_2$ individuals from Category 2, and $k_3$ individuals from Category 3 (where $k_1 + k_2 + k_3 = n$) is

$$\frac{n!}{k_1!k_2!k_3!} p_1^{k_1}p_2^{k_2}p_3^{k_3}$$

\pause
$\leadsto$ In practice, you simply use \texttt{np.random.multinomial()} ;-)

\end{frame}

\end{document}