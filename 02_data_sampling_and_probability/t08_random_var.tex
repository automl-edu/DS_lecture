\documentclass[aspectratio=169]{../latex_main/tntbeamer}  % you can pass all options of the beamer class, e.g., 'handout' or 'aspectratio=43'
\input{../latex_main/preamble_2}

\title[Random Variables]{DS: Data Sampling and Probability}
\subtitle{Random Variables}

\graphicspath{ {./figure/} }
%\institute{}


\begin{document}
	
	\maketitle
	

% Slide 42: Random Variables
\begin{frame}{Random Variable}

    A random variable is a variable that can take numerical values with particular probabilities.\\
    \bigskip
    
    \textbf{Example 1:} \\
    Let $X$ take the value 1 if Roosevelt, 0 if Landon.\\

    \textbf{Example 2:} \\
    Let $Y$ be the number of pips on a roll of a 6-sided die.

    \textbf{Notation:}
    \begin{itemize}
        \item Random variables (RVs) use capital letters, e.g., $X$, $Y$, $Z$.
        \item A particular value taken by an RV is indicated by a lowercase letter, e.g., $x$, $y$, $z$.
        \item The (Probability) Distribution of a discrete RV can be expressed as a table or graphic, e.g., $P(X = x)$.
    \end{itemize}
    

\end{frame}

% Slide 43: Functions of Random Variables
\begin{frame}{Functions of Random Variables}
    
    A function of random variables is also a random variable.
    \bigskip
    
    \textbf{Example 1, cont.:} \\
    Let $S$ be the total number of voters that say “Roosevelt” in a sample of size 10 million.

    \begin{equation}
    S = X_1 + X_2 + \dots + X_{10M}
    \end{equation}


\end{frame}

% Slide 44: Abstracting Random Chance
\begin{frame}{Abstracting Random Chance}

    \textbf{Q:} What do these have in common? \\
    
    \begin{itemize}
        \item Ask a randomly drawn American who they plan to vote for.
        \item The outcome of a coin flip.
        \item The outcome of a COVID test for a randomly selected Californian.
    \end{itemize}
    \pause
    \bigskip

    \textbf{A:} Each has only two outcomes, one of which happens with a particular probability $p$.
\end{frame}

% Slide 45: Bernoulli Distribution
\begin{frame}{Bernoulli Distribution}
    A random variable that takes the value 1 with probability $p$ and 0 otherwise.\\
    
    $X$ is Bernoulli$(p)$ if:
    \begin{equation}
    P(X = 1) = p, \quad P(X = 0) = 1 - p
    \end{equation}

    This is a Probability Mass Function (PMF).\\

    \textbf{Examples:}
    \begin{itemize}
        \item Ask a randomly drawn American who they plan to vote for: Bernoulli$(p = 0.61)$.
        \item The outcome of a coin flip: Bernoulli$(p = 0.5)$.
        \item The outcome of a COVID test for a randomly selected German: Bernoulli$(p = 0.08)$.
    \end{itemize}
\end{frame}

% Slide 46: Abstracting Random Chance
\begin{frame}{Abstracting Random Chance}

    \textbf{Q:} What do these have in common? \\

    \begin{itemize}
        \item Count the number of people that answered “Roosevelt” in a sample of $n = 10$.
        \item The total number of heads in a series of 5 coin flips.
        \item The total number of Germans that will test positive for COVID in a given month.
    \end{itemize}
    \pause
    \bigskip

    \textbf{A:} Each is a sum of Bernoulli random variables.
\end{frame}

% Slide 47: Binomial Distribution
\begin{frame}{Binomial Distribution}
    A random variable that counts the number of "successes" in $n$ independent trials where each succeeds with probability $p$.

    $Y$ is binomial($n,p$) if 
    $$P(Y=y) = \binom{n}{y}p^y(1-p)^{n-y}$$

    Recall: $S = X_1 + X_2 + \ldots + X_{10..}$\\
    S is binomial($n=10\mu, p=0.61$)
\end{frame}

% Slide 48: Abstracting Random Chance
\begin{frame}{Abstracting Random Chance}

    \textbf{Q:} What do these have in common?\\

    \begin{enumerate}
        \item Count the number of people that answered "Roosevelt" in a sample of $n = 10M$.
        \item The total number of heads in a series of 5 coin flips.
        \item The total number of Germans that will test positive for COVID in a given month.
    \end{enumerate}


    A random variable that counts the number of “successes” in $n$ independent trials where each succeeds with probability $p$.
    
    \begin{enumerate}
        \item Binomial$(n = 10M, p = 0.61)$ – BUT each $X_i$ is not quite independent with the same $p$.
        \item Binomial$(n = 5, p = 0.5)$ – Good fit!
        \item Binomial$(n = 0.5M, p = 0.08)$ – Probably not independent (contagious!).
    \end{enumerate}
\end{frame}

% Slide 49: Types of distributions
\begin{frame}{Types of distributions}
    Probability distributions largely fall into two main categories:
    \begin{itemize}
        \item \textbf{Discrete:} 
        \begin{itemize}
            \item The set of possible values that $X$ can take on is either finite or countably infinite.
            \item Values are separated by some fixed amount. 
            \item For instance, $X=1,2,3,4,\ldots$
        \end{itemize} 
        \item \textbf{Continuous:} 
        \begin{itemize}
            \item The set of possible values that $X$ can take on is uncountable.
            \item Typically, $X$ can be any real number in some interval (not just counting numbers).
        \end{itemize} 
    \end{itemize}
    Here, we will focus almost exclusively on discrete distributions, but it’s important to know that continuous distributions exist.
\end{frame}

% Slide 50: Common distributions
\begin{frame}{Common distributions}
    \textbf{Discrete:}
    \begin{itemize}
        \item Bernoulli$(p)$: Takes on the value 1 with probability $p$ and 0 with probability $1 - p$.
        \item Binomial$(n, p)$: Number of 1s in $n$ independent Bernoulli$(p)$ trials. Probabilities are given by the binomial formula.
        \item Uniform on a finite set: Probability of each value is $1 /$ (size of set), e.g., a standard die.
    \end{itemize}

    \textbf{Continuous:}
    \begin{itemize}
        \item Uniform on the unit interval: $U$ could be any real number in the range $[0, 1]$.
        \item Normal distribution $(\mu, \sigma^2)$.
    \end{itemize}
\end{frame}


\end{document}