\documentclass[aspectratio=169]{../latex_main/tntbeamer}  % you can pass all options of the beamer class, e.g., 'handout' or 'aspectratio=43'
\input{../latex_main/preamble_2}

\title[Introduction]{DS: Iris Dataset}
\subtitle{}

\graphicspath{ {./figure/} }
%\institute{}


\begin{document}
	
	\maketitle

	
	
	\begin{frame}{Example: Flower Classification}
	
	\vspace{-2em}
	    The Iris flower data set is a commonly used example:
	    \begin{itemize}
	        \item Created by statistician/biologist Ronald Fisher for his paper “The use of multiple measurements in taxonomic problems”.
	        \item Data set consists of 150 flower measurements from 3 different species.
	        \item For each, we have “petal length”, “petal width”, “sepal length”, “sepal width”.
	    \end{itemize}
	    Goal is to predict species from other data.

	        \centering
	        \includegraphics[scale=.24]{iris}
        \footnotesize \href{https://medium.com/codex/iris-101-dataset-acquisition-and-understanding-73e8b271704c}{[Sengupta]}

	\end{frame}

 	\begin{frame}{Example: Logistic Regression Using Petal Data Only}
	    The plot below shows the width and length of the petals of each flower, with the species annotated in the form of color.
	    \begin{figure}
	        \centering
	        \includegraphics[scale=.75]{figure_tree/Bild3}
	    \end{figure}
	\end{frame}
	
	\begin{frame}{Example: Logistic Regression Using Petal Data Only}

        \vspace{-1em}
	    The plot below shows the width and length of the petals of each flower, with the species annotated in the form of color.
	    \begin{itemize}
	        \item If we train a 3-class logistic regression model on this data, we end up with the model below.
	    \end{itemize}
	    \begin{figure}
	        \centering
	        \includegraphics[scale=.4]{figure_tree/Bild4}
	    \end{figure}

        $\leadsto$ \alert{We need a different model class!}
     
	\end{frame}
 
\end{document}