\documentclass[aspectratio=169]{../latex_main/tntbeamer}  % you can pass all options of the beamer class, e.g., 'handout' or 'aspectratio=43'
\input{../latex_main/preamble_2}

\title[Introduction]{DS: Decision Trees}
\subtitle{Decision Tree Basics}

\graphicspath{ {./figure_tree/} }
%\institute{}


\begin{document}

	\maketitle

	\begin{frame}{Decision Trees}
	    A Decision Tree is a very simple way to classify data. It is simply a tree of questions that must be answered in sequence to yield a predicted classification.
	    \begin{figure}
	        \centering
	        \includegraphics[scale=.35]{figure_tree/Bild1}
	    \end{figure}
	\end{frame}
 
	\begin{frame}{Example: Using Petal Data Only}
	    The plot below shows the width and length of the petals of each flower, with the species annotated in the form of color.\\
	    \bigskip
	    We can build a decision tree manually just by looking at this picture.
	        \centering
	        \includegraphics[scale=.65]{figure_tree/Bild3}

	\end{frame}
	
	\begin{frame}{Example: Using Petal Data Only}
	        %\centering
	        \includegraphics[scale=.34]{figure_tree/Bild5}
	\end{frame}
	
	
	\begin{frame}{Example: Using Petal Data Only}
	        %\centering
	        \includegraphics[scale=.34]{figure_tree/Bild6}
	\end{frame}
	
	
	\begin{frame}{Example: Using Petal Data Only}
	        %\centering
	        \includegraphics[scale=.34]{figure_tree/Bild7}
	\end{frame}
	
	
	\begin{frame}{Example: Using Petal Data Only}
	        %\centering
	        \includegraphics[scale=.34]{figure_tree/Bild8}
	\end{frame}
	
	
	\begin{frame}{Example: Using Petal Data Only}
	        %\centering
	        \includegraphics[scale=.34]{figure_tree/Bild9}
	\end{frame}
	
	
	\begin{frame}{Example: Using Petal Data Only}
	        %\centering
	        \includegraphics[scale=.34]{figure_tree/Bild10}
	\end{frame}
	
	
	\begin{frame}{Example: Using Petal Data Only}
	        %\centering
	        \includegraphics[scale=.34]{figure_tree/Bild11}
	\end{frame}
	
	
	\begin{frame}{Example: Using Petal Data Only}
	        %\centering
	        \includegraphics[scale=.34]{figure_tree/Bild12}
	\end{frame}
	
	
	\begin{frame}{Example: Using Petal Data Only}
	        %\centering
	        \includegraphics[scale=.34]{figure_tree/Bild13}
	\end{frame}
	
	
	\begin{frame}{Example: Using Petal Data Only}
	        %\centering
	        \includegraphics[scale=.34]{figure_tree/Bild14}
	\end{frame}
	
	
	\begin{frame}{Example: Using Petal Data Only}
	        %\centering
	        \includegraphics[scale=.34]{figure_tree/Bild15}
	\end{frame}
	
	
	\begin{frame}{Example: Using Petal Data Only}
	   \begin{columns}
	        \begin{column}{.5\textwidth}
	               How accurate is our decision tree model on the training data?\\
	               \bigskip
	                Is this good or bad?
	        \end{column}
	   
	   
	         \begin{column}{.5\textwidth}

	                    %\centering
	                     \includegraphics[scale=.34]{figure_tree/Bild16}

	        \end{column}
	   \end{columns}
	 \end{frame}
	 
	 
	 \begin{frame}{Example: Using Petal Data Only}
	   \begin{columns}
	        \begin{column}{.5\textwidth}
	        
	               How accurate is our decision tree model on the training data?\\
	               \begin{itemize}
	                   \item It seems like it gets every point correct
	               \end{itemize}
	               \bigskip
	                Is this good or bad?
	                \begin{itemize}
	                    \item Rather bad
	                    \item Seems likely to result in overfitting!
	                    \item Will discuss overfitting more later
	                \end{itemize}
	                
	        \end{column}
	   
	   
	         \begin{column}{.5\textwidth}

	                     \includegraphics[scale=.34]{figure_tree/Bild16}
	                     
	        \end{column}
	   \end{columns}
	 \end{frame}
\end{document}