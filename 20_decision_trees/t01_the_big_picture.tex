\documentclass[aspectratio=169]{../latex_main/tntbeamer}  % you can pass all options of the beamer class, e.g., 'handout' or 'aspectratio=43'
\input{../latex_main/preamble}

\title[Introduction]{DS: Decision Trees}
\subtitle{}

\graphicspath{ {./figure/} }
%\institute{}


\begin{document}
	
	\maketitle
	\begin{frame}{Decision Trees}
	    A Decision Tree is a very simple way to classify data. It is simply a tree of questions that must be answered in sequence to yield a predicted classification.
	    \begin{figure}
	        \centering
	        \includegraphics[scale=.35]{Bild1}
	    \end{figure}
	\end{frame}
	
	
	\begin{frame}{Example: Flower Classification}
	    The Iris flower data set is a commonly used example:
	    \begin{itemize}
	        \item Created by statistician/biologist Ronald Fisher for his paper “The use of multiple measurements in taxonomic problems”.
	        \item Data set consists of 150 flower measurements from 3 different species.
	        \item For each, we have “petal length”, “petal width”, “sepal length”, “sepal width”.
	    \end{itemize}
	    Goal is to predict species from other data.
	    \begin{figure}
	        \centering
	        \includegraphics[scale=.35]{Bild2}
	    \end{figure}
	    \url{https://en.wikipedia.org/wiki/Sepal}

	\end{frame}
\end{document}