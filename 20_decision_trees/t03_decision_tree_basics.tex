\documentclass[aspectratio=169]{../latex_main/tntbeamer}  % you can pass all options of the beamer class, e.g., 'handout' or 'aspectratio=43'
\input{../latex_main/preamble}

\title[Introduction]{DS: Decision Trees}
\subtitle{Decision Tree Basics}

\graphicspath{ {./figure/} }
%\institute{}


\begin{document}
	
	\maketitle
	\begin{frame}{Example: Using Petal Data Only}
	    The plot below shows the width and length of the petals of each flower, with the species annotated in the form of color.\\
	    \bigskip
	    We can build a decision tree manually just by looking at this picture.
	    \begin{figure}
	        \centering
	        \includegraphics[scale=.65]{Bild3}
	    \end{figure}
	\end{frame}
	
	\begin{frame}{Example: Using Petal Data Only}
	    \begin{figure}
	        %\centering
	        \includegraphics[scale=.34]{Bild5}
	    \end{figure}
	\end{frame}
	
	
	\begin{frame}{Example: Using Petal Data Only}
	    \begin{figure}
	        %\centering
	        \includegraphics[scale=.34]{Bild6}
	    \end{figure}
	\end{frame}
	
	
	\begin{frame}{Example: Using Petal Data Only}
	    \begin{figure}
	        %\centering
	        \includegraphics[scale=.34]{Bild7}
	    \end{figure}
	\end{frame}
	
	
	\begin{frame}{Example: Using Petal Data Only}
	    \begin{figure}
	        %\centering
	        \includegraphics[scale=.34]{Bild8}
	    \end{figure}
	\end{frame}
	
	
	\begin{frame}{Example: Using Petal Data Only}
	    \begin{figure}
	        %\centering
	        \includegraphics[scale=.34]{Bild9}
	    \end{figure}
	\end{frame}
	
	
	\begin{frame}{Example: Using Petal Data Only}
	    \begin{figure}
	        %\centering
	        \includegraphics[scale=.34]{Bild10}
	    \end{figure}
	\end{frame}
	
	
	\begin{frame}{Example: Using Petal Data Only}
	    \begin{figure}
	        %\centering
	        \includegraphics[scale=.34]{Bild11}
	    \end{figure}
	\end{frame}
	
	
	\begin{frame}{Example: Using Petal Data Only}
	    \begin{figure}
	        %\centering
	        \includegraphics[scale=.34]{Bild12}
	    \end{figure}
	\end{frame}
	
	
	\begin{frame}{Example: Using Petal Data Only}
	    \begin{figure}
	        %\centering
	        \includegraphics[scale=.34]{Bild13}
	    \end{figure}
	\end{frame}
	
	
	\begin{frame}{Example: Using Petal Data Only}
	    \begin{figure}
	        %\centering
	        \includegraphics[scale=.34]{Bild14}
	    \end{figure}
	\end{frame}
	
	
	\begin{frame}{Example: Using Petal Data Only}
	    \begin{figure}
	        %\centering
	        \includegraphics[scale=.34]{Bild15}
	    \end{figure}
	\end{frame}
	
	
	\begin{frame}{Example: Using Petal Data Only}
	   \begin{columns}
	        \begin{column}{.5\textwidth}
	               How accurate is our decision tree model on the training data?\\
	               \bigskip
	                Is this good or bad?
	        \end{column}
	   
	   
	         \begin{column}{.5\textwidth}
	               \begin{figure}
	                    %\centering
	                     \includegraphics[scale=.34]{Bild16}
	                \end{figure}
	        \end{column}
	   \end{columns}
	 \end{frame}
	 
	 
	 \begin{frame}{Example: Using Petal Data Only}
	   \begin{columns}
	        \begin{column}{.5\textwidth}
	               How accurate is our decision tree model on the training data?\\
	               \begin{itemize}
	                   \item It seems like it gets every point correct
	               \end{itemize}
	               \bigskip
	                Is this good or bad?
	                \begin{itemize}
	                    \item I’d argue bad
	                    \item Seems likely to result in overfitting!
	                    \item Will discuss overfitting more later
	                \end{itemize}
	                First, let’s see how we can build decision trees for classification using scikit-learn
	        \end{column}
	   
	   
	         \begin{column}{.5\textwidth}
	               \begin{figure}
	                    %\centering
	                     \includegraphics[scale=.34]{Bild16}
	                \end{figure}
	        \end{column}
	   \end{columns}
	 \end{frame}
\end{document}