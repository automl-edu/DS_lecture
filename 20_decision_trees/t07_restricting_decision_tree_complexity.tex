\documentclass[aspectratio=169]{../latex_main/tntbeamer}  % you can pass all options of the beamer class, e.g., 'handout' or 'aspectratio=43'
\input{../latex_main/preamble}

\title[Introduction]{DS: Decision Trees}
\subtitle{Basic Decision Tree Generation}

\graphicspath{ {./figure/} }
%\institute{}


\begin{document}
	
	\maketitle
	\begin{frame}{Overfitting and Our Algorithm}
	    A “fully grown” decision tree built with our algorithm runs the risk of overfitting\\
	    \bigskip
	    One idea to avoid overfitting: Don’t allow fully grown trees
	    \begin{figure}
	        \centering
	        \includegraphics[scale=.45]{Bild52}
	    \end{figure}
	\end{frame}
	
	\begin{frame}{Regularization Terms and Decision Trees?}
	    We can’t just use the regularization term idea from linear models
	    \begin{itemize}
	        \item There is no global function being minimized
	        \item Instead, the decision tree is built up node by node in a “greedy” fashion
	    \end{itemize}
	    \begin{figure}
	        \centering
	        \includegraphics[scale=.45]{Bild52}
	    \end{figure}
	\end{frame}
	
	
	\begin{frame}{Approach 1: Preventing Growth}
	    Approach 1: Set one or more special rules to prevent growth\\
	    \bigskip
	    Examples:
	    \begin{itemize}
	        \item Don’t split nodes with < 1\% of the samples  
	        \item Don’t allow nodes to be more than 7 levels deep in the tree
	    \end{itemize}
	\end{frame}
	
	
	\begin{frame}{Approach 2: Pruning}
	    Approach 2: Let tree fully grow, then cut off less useful branches of the tree. For example, consider the highlighted branch below:
	    \begin{itemize}
	        \item Many rules that affect few points
	        \item How can we avoid this?
	    \end{itemize}
	    \begin{figure}
	        \centering
	        \includegraphics[scale=.25]{Bild53}
	    \end{figure}
	\end{frame}
	
	
	\begin{frame}{Specific Pruning Example}
	    \begin{columns}
	        \begin{column}{.5\textwidth}
	              \begin{figure}
            	        \centering
            	        \includegraphics[scale=.35]{Bild54}
	            \end{figure}  
	        \end{column}
	        
	        \begin{column}{.5\textwidth}
	                One way to prune: 
	                \begin{itemize}
	                    \item Before creating the tree, set aside a validation set
	                    \item If replacing a node by its most common prediction has no impact on the validation error, then don’t split that node
	                \end{itemize}
	        \end{column}
	    \end{columns}
	\end{frame}
	
	
	
	\begin{frame}{Overfitting and Our Algorithm}
	    A “fully grown” decision tree built with our algorithm runs the risk of overfitting\\
	    \bigskip
	    One idea to avoid overfitting: Don’t allow fully grown trees
	    \begin{itemize}
	        \item Approach 1: Set rules to prevent full growth
	        \item Approach 2: Allow full growth then prune branches afterwards
	    \end{itemize}
	    
        Won’t discuss these in any great detail
        \begin{itemize}
            \item There’s a completely different idea called a “random forest” that is more popular and IMO more beautiful
        \end{itemize}
	\end{frame}
\end{document}