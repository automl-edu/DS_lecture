\documentclass[aspectratio=169]{../latex_main/tntbeamer}  % you can pass all options of the beamer class, e.g., 'handout' or 'aspectratio=43'
\input{../latex_main/preamble}

\title[Introduction]{DS: Principal Component Analysis}
\subtitle{Covariance}

\graphicspath{ {./figure/} }
%\institute{}


\begin{document}
	
	\maketitle
	\begin{frame}{Covariance}
	    Remember, we can define the covariance between two random variables as:
	    \begin{equation*}
	        \mathbb{Cov}(X,Y) = \mathbb{E}((X - \mathbb{E}(X))(Y- \mathbb{E}(Y)))
	    \end{equation*}
	    Usually, we are interested in the covariance between two features in our data.\\
	    For example, if we have two features   $X_1$     and       $X_2$ , the covariance between them is:
	    \begin{equation*}
	        \frac{1}{n}(X_1 - \overline{X}_1)^T(X_2 - \overline{X}_2)
	    \end{equation*}
	    This is just the empirical calculation of the random variable notation above.

	\end{frame}
	
	\begin{frame}{Covariance Matrix}
	    Of course, in many cases, we have more than 2 features. It is practical to present all of the pairwise covariances in a matrix, like so:\\
	    \bigskip
	    \left[\begin{array}{cccc}
	        Cov(X_1,X_1) & Cov(X_1,X_2) & \dots & Cov(X_1,X_d) \\
	        Cov(X_2,X_1) & Cov(X_2,X_2) & \dots & Cov(X_2,X_d) \\
	              \vdots &     \vdots   &   \ddots  &  \vdots  \\
	        Cov(X_d,X_1) & Cov(X_d,X_2) & \dots & Cov(X_d,X_d)
	    \end{array}\right]\\
	    \bigskip
	    Based on the formula from the previous slide, we can directly obtain the covariance matrix with the following expression, if and only if X is centered.
	    \begin{align*}
	        \frac{1}{n}X^TX
	    \end{align*}
	\end{frame}
	
	
	
	\begin{frame}{Real-World Data}
	    Here is dataset showing students’ scores on their midterm and final exam.\\
	    When calculating covariance, we need to center our data first.
	    \begin{columns}
	        \begin{column}{.5\textwidth}
	                \begin{figure}
	                    \centering
	                    \includegraphics[scale=.35]{Bild2}
	                \end{figure}
	        \end{column}
	        
	        
	        \begin{column}{.5\textwidth}
	                \\
	                \bigskip
	                \bigskip
	                \begin{align*}
	                     \frac{1}{n}X^TX =
	                \left[\begin{array}{cc}
	                    .022 & .021 \\
	                   .021  & .037
	                \end{array}\right]
	                \end{align*}
	        \end{column}
	    \end{columns}
	\end{frame}
\end{document}