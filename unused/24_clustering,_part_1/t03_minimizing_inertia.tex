\documentclass[aspectratio=169]{../latex_main/tntbeamer}  % you can pass all options of the beamer class, e.g., 'handout' or 'aspectratio=43'
\input{../latex_main/preamble}

\title[Introduction]{DS: Clustering, Part 1}
\subtitle{Minimizing Inertia}

\graphicspath{ {./figure/} }
%\institute{}


\begin{document}
	
	\maketitle
	\begin{frame}{K-Means Clustering for K = 4}
	    Below is an example of an output for K=4:
	    \begin{figure}
	        \centering
	        \includegraphics[scale=.75]{Bild21}
	    \end{figure}
	\end{frame}
	
	
	\begin{frame}{K-Means Clustering for K = 4}
	    Each time you run K-Means, you get a different output
	    \begin{figure}
	        \centering
	        \includegraphics[scale=.4]{Bild22}
	    \end{figure}
	    Which is best?
	    \begin{itemize}
	        \item One approach: Define some sort of loss function
	        \pause
	        \item Come up with a loss function for clustering
	    \end{itemize}
	\end{frame}
	
	
	\begin{frame}{K-Means Clustering for K = 4}
	    Each time you run K-Means, you get a different output
	    \begin{figure}
	        \centering
	        \includegraphics[scale=.4]{Bild22}
	    \end{figure}
	    Come up with a loss function for clustering -- your ideas:
	    \begin{itemize}
	        \item The sum of distances from each point to its center
	        \item Could take into account balance of number of points per cluster
	    \end{itemize}
	\end{frame}
	
	
	\begin{frame}{K-Means Clustering for K = 4}
	    To evaluate different clustering results, we need a loss function
	    \begin{columns}
	        \begin{column}{.5\textwidth}
	                \begin{figure}
	                    \centering
	                    \includegraphics[scale=.5]{Bild23}
	                \end{figure}
	        \end{column}
	        
	        \begin{column}{.5\textwidth}
	                Two common loss functions:
	                \begin{itemize}
	                    \item Inertia: Sum of squared distances from each data point to its center
	                    \item Distortion: Weighted sum of squared distances from each data point to its center
	                \end{itemize}
	                Example: 
	                \begin{itemize}
	                    \item Inertia: 0.47$^2$ + 0.19$^2$ + 0.34$^2$   + 0.25$^2$ + 0.58$^2$ + 0.36$^2$ + 0.44$^2$
	                    \item Distortion: (0.47$^2$ + 0.19$^2$ + 0.34$^2$)/3 + (0.25$^2$ + 0.58$^2$ + 0.36$^2$ + 0.44$^2$)/4

	                \end{itemize}
	        \end{column}
	    \end{columns}
	\end{frame}
	
	
	
	\begin{frame}{K-Means Clustering for K = 4}
	    Each time you run K-Means, you get a different output
	    \begin{figure}
	        \centering
	        \includegraphics[scale=.4]{Bild24}
	    \end{figure}
	   Our loss function says that the leftmost clustering is best (inertia: 44.96) and rightmost clustering (inertia: 54.35) is worst

	\end{frame}
	
	
	\begin{frame}{K-Means and Inertia}
	    It turns out that the function K-Means is trying to minimize is inertia\\
… but often fails to find global optimum. Why?\\
Can think of K-means as a pair of optimizers that take turns
    \begin{itemize}
        \item First optimizer:
        \begin{itemize}
            \item Holds center positions constant
            \item Optimizes data colors
        \end{itemize}
        \item Second optimizer:
        \begin{itemize}
            \item Holds data colors constant
            \item Optimizes center positions
        \end{itemize}
        \item Neither gets total control 
    \end{itemize}
	\end{frame}
	
	
	
	\begin{frame}{Optimizing Inertia}
	    Hard problem: Give an algorithm that optimizes inertia
	    \begin{itemize}
	        \item Your algorithm should return the EXACT best centers and colors
	        \item Don’t worry about runtime
	    \end{itemize}
	    This is a bit of a CS61B/CS70/CS170 problem. It may be far too hard for some of you.
	\end{frame}
	
	
	\begin{frame}{Optimizing Inertia}
	    Hard problem: Give an algorithm that optimizes inertia
	    \begin{itemize}
	        \item Your algorithm should return the EXACT best centers and colors
	        \item Don’t worry about runtime
	    \end{itemize}
	    Algorithm:
	    \begin{itemize}
	        \item For all possible kn colorings:$\leftarrow$\includegraphics[scale=.35]{Bild25}
	        \begin{itemize}
	            \item Compute the k centers for that coloring
	            \item Compute the inertia for the k centers
	            \begin{itemize}
	                \item If current inertia is better than best known, write down the current centers and coloring and call that the new best known
	            \end{itemize}
	        \end{itemize}
	    \end{itemize}
	    No better algorithm has been found$\leftarrow$\includegraphics[scale=.35]{Bild26}
	    %\bigskip
	    %*A “coloring” is just a choice of color for every point, e.g. point 1 = red, point 2 = green, point 3 = red, point 4 = blue\\
	    %**For those who know what this means: K-Means is known to be an NP-hard problem
	    
	\end{frame}
	
	
\end{document}