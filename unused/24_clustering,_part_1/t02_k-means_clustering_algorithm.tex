\documentclass[aspectratio=169]{../latex_main/tntbeamer}  % you can pass all options of the beamer class, e.g., 'handout' or 'aspectratio=43'
\input{../latex_main/preamble}

\title[Introduction]{DS: Clustering, Part 1}
\subtitle{K-Means Clustering Algorithm}

\graphicspath{ {./figure/} }
%\institute{}


\begin{document}
	
	\maketitle
	\begin{frame}{K-Means Clustering}
	    Most popular clustering approach: K-Means
	    \begin{itemize}
	        \item Pick an arbitrary k, and randomly place k “centers”, each a different color
	        \item Repeat until convergence:
	        \begin{itemize}
	            \item Color points according to the closest center
	            \item Move center for each color to center of points with that color
	        \end{itemize}
	    \end{itemize}
	    \begin{figure}
	        \centering
	        \includegraphics[scale=.4]{Bild7}
	    \end{figure}
	\end{frame}
	
	
	
	\begin{frame}{K-Means Clustering}
	    Most popular clustering approach: K-Means
	    \begin{itemize}
	        \item Pick an arbitrary k, and \textbf{randomly place k “centers”}, each a different color
	        \item Repeat until convergence:
	        \begin{itemize}
	            \item Color points according to the closest center
	            \item Move center for each color to center of points with that color
	        \end{itemize}
	    \end{itemize}
	    \begin{figure}
	        \centering
	        \includegraphics[scale=.4]{Bild8}
	    \end{figure}
	\end{frame}
	
	
	\begin{frame}{K-Means Clustering}
	    Most popular clustering approach: K-Means
	    \begin{itemize}
	        \item Pick an arbitrary k, and randomly place k “centers”, each a different color
	        \item Repeat until convergence:
	        \begin{itemize}
	            \item \textbf{Color points according to the closest center}
	            \item Move center for each color to center of points with that color
	        \end{itemize}
	    \end{itemize}
	    \begin{figure}
	        \centering
	        \includegraphics[scale=.38]{Bild9}
	    \end{figure}
	\end{frame}
	
	
	\begin{frame}{K-Means Clustering}
	    Most popular clustering approach: K-Means
	    \begin{itemize}
	        \item Pick an arbitrary k, and randomly place k “centers”, each a different color
	        \item Repeat until convergence:
	        \begin{itemize}
	            \item Color points according to the closest center
	            \item \textbf{Move center for each color to center of points with that color}
	        \end{itemize}
	    \end{itemize}
	    \begin{figure}
	        \centering
	        \includegraphics[scale=.4]{Bild10}
	    \end{figure}
	\end{frame}
	
	
	
	\begin{frame}{K-Means Clustering}
	    Most popular clustering approach: K-Means
	    \begin{itemize}
	        \item Pick an arbitrary k, and randomly place k “centers”, each a different color
	        \item Repeat until convergence:
	        \begin{itemize}
	            \item Color points according to the closest center
	            \item \textbf{Move center for each color to center of points with that color}
	        \end{itemize}
	    \end{itemize}
	    \begin{figure}
	        \centering
	        \includegraphics[scale=.39]{Bild11}
	    \end{figure}
	\end{frame}
	
	
	
	
	
	\begin{frame}{K-Means Clustering}
	    Most popular clustering approach: K-Means
	    \begin{itemize}
	        \item Pick an arbitrary k, and randomly place k “centers”, each a different color
	        \item Repeat until convergence:
	        \begin{itemize}
	            \item \textbf{Color points according to the closest center}
	            \item Move center for each color to center of points with that color
	        \end{itemize}
	    \end{itemize}
	    \begin{figure}
	        \centering
	        \includegraphics[scale=.38]{Bild12}
	    \end{figure}
	\end{frame}
	
	
	
	\begin{frame}{K-Means Clustering}
	    Most popular clustering approach: K-Means
	    \begin{itemize}
	        \item Pick an arbitrary k, and randomly place k “centers”, each a different color
	        \item Repeat until convergence:
	        \begin{itemize}
	            \item Color points according to the closest center
	            \item \textbf{Move center for each color to center of points with that color}
	        \end{itemize}
	    \end{itemize}
	    \begin{figure}
	        \centering
	        \includegraphics[scale=.39]{Bild13}
	    \end{figure}
	\end{frame}
	
	
	
	
	\begin{frame}{K-Means Clustering}
	    Most popular clustering approach: K-Means
	    \begin{itemize}
	        \item Pick an arbitrary k, and randomly place k “centers”, each a different color
	        \item Repeat until convergence:
	        \begin{itemize}
	            \item \textbf{Color points according to the closest center}
	            \item Move center for each color to center of points with that color
	        \end{itemize}
	    \end{itemize}
	    \begin{figure}
	        \centering
	        \includegraphics[scale=.38]{Bild14}
	    \end{figure}
	\end{frame}
	
	
	
	\begin{frame}{K-Means Clustering}
	    Most popular clustering approach: K-Means
	    \begin{itemize}
	        \item Pick an arbitrary k, and randomly place k “centers”, each a different color
	        \item Repeat until convergence:
	        \begin{itemize}
	            \item Color points according to the closest center
	            \item \textbf{Move center for each color to center of points with that color}
	        \end{itemize}
	    \end{itemize}
	    \begin{figure}
	        \centering
	        \includegraphics[scale=.39]{Bild15}
	    \end{figure}
	\end{frame}
	
	
	\begin{frame}{K-Means Clustering}
	    Most popular clustering approach: K-Means
	    \begin{itemize}
	        \item Pick an arbitrary k, and randomly place k “centers”, each a different color
	        \item Repeat until convergence:
	        \begin{itemize}
	            \item \textbf{Color points according to the closest center}
	            \item Move center for each color to center of points with that color
	        \end{itemize}
	    \end{itemize}
	    \begin{figure}
	        \centering
	        \includegraphics[scale=.38]{Bild16}
	    \end{figure}
	\end{frame}
	
	
	\begin{frame}{K-Means Clustering}
	    Most popular clustering approach: K-Means
	    \begin{itemize}
	        \item Pick an arbitrary k, and randomly place k “centers”, each a different color
	        \item Repeat until convergence:
	        \begin{itemize}
	            \item Color points according to the closest center
	            \item \textbf{Move center for each color to center of points with that color}
	        \end{itemize}
	    \end{itemize}
	    \begin{figure}
	        \centering
	        \includegraphics[scale=.39]{Bild17}
	    \end{figure}
	\end{frame}
	
	
	
	\begin{frame}{K-Means Clustering}
	    Most popular clustering approach: K-Means
	    \begin{itemize}
	        \item Pick an arbitrary k, and randomly place k “centers”, each a different color
	        \item Repeat until convergence:
	        \begin{itemize}
	            \item \textbf{Color points according to the closest center}
	            \item Move center for each color to center of points with that color
	        \end{itemize}
	    \end{itemize}
	    \begin{figure}
	        \centering
	        \includegraphics[scale=.38]{Bild18}
	    \end{figure}
	\end{frame}
	
	
	\begin{frame}{K-Means Clustering}
	    \begin{figure}
	        \centering
	        \includegraphics[scale=.38]{Bild19}
	    \end{figure}
	    Above we see the results after iteration 4 and 5
	    \begin{itemize}
	        \item Centers moved slightly between iteration 4 and 5
	        \item But no points changed color
	        \item Are we done?
	    \end{itemize}
	\end{frame}
	
	
	\begin{frame}{K-Means Clustering}
	    \begin{figure}
	        \centering
	        \includegraphics[scale=.38]{Bild19}
	    \end{figure}
	    Above we see the results after iteration 4 and 5
	    \begin{itemize}
	        \item Are we done?
	        \begin{itemize}
	            \item Yes! If we tried iteration 6, we’d see that centers don’t move at all
	        \end{itemize}
	    \end{itemize}
	\end{frame}
	
	
	
	\begin{frame}[c]{K-Means vs. K-Nearest Neighbors}
	    Quick note: K-Means is a totally different algorithm than “K-Nearest Neighbors”
	    \begin{itemize}
	        \item K-Means: For clustering
	        \begin{itemize}
	            \item Assigns each point to one of K clusters
	        \end{itemize}
	        \item K-Nearest Neighbors: For Classification (or less often, Regression)
	        \begin{itemize}
	            \item Prediction is the most common class among the k-nearest data points in the training set
	            \item Won’t discuss in our class
	        \end{itemize}
	    \end{itemize}
	\end{frame}
	
	
	\begin{frame}{.}
	    \begin{figure}
	        
	        
	        \centering
	        \includegraphics[scale=.45]{Bild20}
	       Credit: A Fall ‘19 Student
	    \end{figure}
	    
	\end{frame}
\end{document}