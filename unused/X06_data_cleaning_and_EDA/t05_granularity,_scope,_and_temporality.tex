\documentclass[aspectratio=169]{../latex_main/tntbeamer}  % you can pass all options of the beamer class, e.g., 'handout' or 'aspectratio=43'
\input{../latex_main/preamble}

\title[Introduction]{DS: Data Cleaning and EDA}
\subtitle{Granularity, Scope, and Temporality}

\graphicspath{ {./figure/} }
%\institute{}


\begin{document}
	
	\maketitle
	
	\begin{frame}{Key Data Properties to Consider in EDA}
	    \begin{itemize}
	        \item Structure -- the “shape” of a data file
	        \item \textbf{Granularity -- how fine/coarse is each datum}
	        \item Scope -- how (in)complete is the data
	        \item Temporality -- how is the data situated in time
	        \item Faithfulness -- how well does the data capture “reality”
	    \end{itemize}
	\end{frame}
	
	
	
	\begin{frame}{Granularity}
	    \centering
	    \includegraphics[scale=.35]{Bild18}
	    \begin{itemize}
	        \item What does each record represent?
	        \begin{itemize}
	            \item Examples: a purchase, a person, a group of users
	        \end{itemize}
	        \item Do all records capture granularity at the same level?
	        \begin{itemize}
	            \item Some data will include summaries (aka rollups) as records
	        \end{itemize}
	        \item If the data are coarse how was it aggregated?
	        \begin{itemize}
	            \item Sampling, averaging, …
	        \end{itemize}
	    \end{itemize}
	\end{frame}
	
	
	
	\begin{frame}{Key Data Properties to Consider in EDA}
	    \begin{itemize}
	        \item Structure -- the “shape” of a data file
	        \item Granularity -- how fine/coarse is each datum
	        \item \textbf{Scope -- how (in)complete is the data}
	        \item Temporality -- how is the data situated in time
	        \item Faithfulness -- how well does the data capture “reality”
	    \end{itemize}
	\end{frame}
	
	
	
	\begin{frame}{Scope}
	    \begin{itemize}
	        \item Does my data cover my area of interest?
	        \begin{itemize}
	            \item Example: I am interested in studying crime in California but I only have Berkeley crime data. 
	        \end{itemize}
	        \item Is my data too expansive?
	        \begin{itemize}
	            \item Example: I am interested in student grades for DS100 but have student grades for all statistics classes.
	            \item Solution: Filtering ⇒ Implications on sample?
	            \begin{itemize}
	                \item If the data is a sample I may have poor coverage after filtering …
	            \end{itemize}
	        \end{itemize}
	        \item Does my data cover the right time frame?
	        \begin{itemize}
	            \item More on this in temporality … 
	        \end{itemize}
	    \end{itemize}
	\end{frame}
	
	
	
	\begin{frame}{Revisiting the Sampling Frame}
	    \begin{itemize}
	        \item The sampling frame is the population from which the data was sampled.
	        \begin{itemize}
	            \item Note that this may not be the population of interest.
	        \end{itemize}
	        \item How complete/incomplete is the frame (and its data)? 
	        \item How is the frame/data situated in place?
	        \item How well does the frame/data capture reality?
	        \item How is the frame/data situated in time? 
	    \end{itemize}
	\end{frame}
	
	
	\begin{frame}{Key Data Properties to Consider in EDA}
	    \begin{itemize}
	        \item Structure -- the “shape” of a data file
	        \item Granularity -- how fine/coarse is each datum
	        \item Scope -- how (in)complete is the data
	        \item \textbf{Temporality -- how is the data situated in time}
	        \item Faithfulness -- how well does the data capture “reality”
	    \end{itemize}
	\end{frame}
	
	
	
	
	\begin{frame}{Temporality}
	    \begin{itemize}
	        \item Data changes – when was the data collected?
	        \item What is the meaning of the time and date fields?
	        \begin{itemize}
	            \item When the “event” happened?
	            \item When the data was collected or was entered into the system?
	            \item Date the data was copied into a database (look for many matching timestamps)
	        \end{itemize}
	        \item Time depends on where! (Time zones & daylight savings)
	        \begin{itemize}
	            \item Learn to use datetime python library
	            \item Multiple string representation (depends on region): 07/08/09?
	        \end{itemize}
	        \item Are there strange null values?
	        \begin{itemize}
	            \item January 1$^{st}$ 1970, January 1$^{st}$ 1900
	        \end{itemize}
	        \item Is there periodicity? Diurnal patterns
	    \end{itemize}
	\end{frame}
	
	
	
	
	\begin{frame}{Unix Time / POSIX Time}
	    \begin{columns}
	    \begin{column}{.7\textwidth}
	    
	  
	    
	    \begin{itemize}
	        \item Time measured in seconds since January 1$^{st}$ 1970
	        \begin{itemize}
	            \item Minus leap seconds …
	        \end{itemize}
	        \item Unix time follows Coordinated Universal Time (UTC)
	        \begin{itemize}
	            \item International time standard 
	            \item Measured at 0 degrees latitude
	            \begin{itemize}
	                \item Similar to Greenwich Mean Time (GMT)
	            \end{itemize}
	            \item No daylight savings 
	            \item Time codes 
	        \end{itemize}
	        \item Time Zones
	        \begin{itemize}
	            \item San Francisco (UTC-8) without daylight savings
	        \end{itemize}
	    \end{itemize}
	    \bigskip
	    \url{https://en.wikipedia.org/wiki/Coordinated_Universal_Time}
	      \end{column}
	      
	      
	      \begin{column}{.3\textwidth}
	              \begin{figure}
	                  \includegraphics[scale=.45]{Bild19}
	              \end{figure}
	      \end{column}
	    \end{columns}
	\end{frame}
\end{document}