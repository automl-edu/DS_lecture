\documentclass[aspectratio=169]{../latex_main/tntbeamer}  % you can pass all options of the beamer class, e.g., 'handout' or 'aspectratio=43'
\input{../latex_main/preamble}

\title[Introduction]{DS: Data Cleaning and EDA}
\subtitle{Faithfulness and Missing Values}

\graphicspath{ {./figure/} }
%\institute{}


\begin{document}
	
	\maketitle
	
		\begin{frame}{Key Data Properties to Consider in EDA}
	    \begin{itemize}
	        \item Structure -- the “shape” of a data file
	        \item Granularity -- how fine/coarse is each datum
	        \item Scope -- how (in)complete is the data
	        \item Temporality -- how is the data situated in time
	        \item \textbf{Faithfulness -- how well does the data capture “reality”}
	    \end{itemize}
	\end{frame}
	
	
	\begin{frame}{Faithfulness: Do I trust this data?}
	    \begin{itemize}
	        \item Does my data contain unrealistic or “incorrect” values?
	        \begin{itemize}
	            \item Dates in the future for events in the past
	            \item Locations that don’t exist
	            \item Negative counts
	            \item Misspellings of names
	            \item Large outliers
	        \end{itemize}
	        \item Does my data violate obvious dependencies?
	        \begin{itemize}
	            \item E.g., age and birthday don’t match 
	        \end{itemize}
	        \item Was the data entered by hand?
	        \begin{itemize}
	            \item Spelling errors, fields shifted …
	            \item Did the form require fields or provide default values?
	        \end{itemize}
	        \item Are there obvious signs of data falsification:
	        \begin{itemize}
	            \item Repeated names, fake looking email addresses, repeated use of uncommon names or fields.
	        \end{itemize}
	    \end{itemize}
	\end{frame}
	
	
	
	\begin{frame}{Signs that your data may not be faithful}
	    \begin{itemize}
	        \item Missing Values/Default values?
	        \begin{itemize}
	            \item What do they look like?
	            \begin{itemize}
	                \item “   “, 
	                \item 0, 
	                \item -1, 999, 12345, 
	                \item NaN, Null, 
	                \item 1970, 1900
	            \end{itemize}
	    \end{itemize}
	    \end{itemize}
	\end{frame}
	
	
	
	\begin{frame}{What to do with the Missing Values?}
	    \begin{itemize}
	        \item Drop records with missing values
	        \begin{itemize}
	            \item Probably most common
	            \item Caution: check for biases introduced by dropped values
	            \begin{itemize}
	                \item Missing or corrupt records might be related to something of interest
	            \end{itemize}
	        \end{itemize}
	        \item Imputation: (Inferring missing values)
            \begin{itemize}
                \item Mean Imputation: replace with an average value 
                \begin{itemize}
                    \item Which mean?  Often use closest related subgroup mean.
                \end{itemize}
                \item Hot deck imputation: replace with a random value
                \begin{itemize}
                    \item Choose a random value from the subgroup and use it for the missing value.
                \end{itemize}
                \item Prof. Gonzalez Suggestion: 
                \begin{itemize}
                    \item Drop missing values but check for induced bias (use domain knowledge)
                    \item Directly model missing values during future analysis
                \end{itemize}
            \end{itemize}
            
	    \end{itemize}
	\end{frame}
	
	
	
	\begin{frame}{Signs that your data may not be faithful}
	    \begin{itemize}
	        \item Missing Values or default values
	        \item Truncated data (early excel limits: 65536 Rows, 255 Columns)
	        \begin{itemize}
	            \item Soln: be aware of consequences in analysis ⇒ how did truncation affect sample?
	        \end{itemize}
	        \item Time Zone Inconsistencies
            \begin{itemize}
                \item Soln 1: convert to a common timezone (e.g., UTC) 
                \item Soln 2: convert to the timezone of the location – useful in modeling behavior.
            \end{itemize}
            \item Duplicated Records or Fields
            \begin{itemize}
                \item Soln: identify and eliminate (use primary key) ⇒ implications on sample?
            \end{itemize}
            \item Spelling Errors
            \begin{itemize}
                \item Soln: Apply corrections or drop records not in a dictionary ⇒ implications on sample?
            \end{itemize}
            \item Units not specified or consistent
            \begin{itemize}
                \item Solns: Infer units, check values are in reasonable ranges for data
            \end{itemize}
            \item Others…
	    \end{itemize}
	\end{frame}
\end{document}