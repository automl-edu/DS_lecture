\documentclass[aspectratio=169]{../latex_main/tntbeamer}  % you can pass all options of the beamer class, e.g., 'handout' or 'aspectratio=43'
\input{../latex_main/preamble}

\title[Introduction]{DS: Data Cleaning and EDA}
\subtitle{Exploratory data analysis and its role in the data science lifecycle.}

\graphicspath{ {./figure/} }
%\institute{}


\begin{document}
	
	\maketitle
	\begin{frame}{Pandas and Jupyter Notebooks}
	    \begin{itemize}
	        \item Introduced DataFrame concepts
	        \begin{itemize}
	            \item Series: A named column of data with an index
	            \item Indexes: The mapping from keys to rows
	            \item DataFrame: collection of series with common index
	        \end{itemize}
	        \item Dataframe access methods
	        \begin{itemize}
	            \item Filtering on predicts and slicing
	            \item df.loc: location by index
	            \item df.iloc: location by integer address
	            \item groupby \& pivot aggregating data
	        \end{itemize}
	    \end{itemize}
	\end{frame}
	
	
	\begin{frame}{Today}
	    \centering
	    \includegraphics[scale=.5]{Bild1}
	    
	\end{frame}
	
	
	\begin{frame}{Congratulations!}
	    \begin{columns}
	    
	    \begin{column}{.4\textwidth}
	    
	    
	    \begin{figure}
	        \includegraphics[scale=.5]{Bild1}
	    \end{figure}
	   \end{column}
	   
	   \begin{column}{.4\textwidth}
	        \bigskip
	        \bigskip
	        \bigskip
	        \bigskip\\
	        You have collected or been given a box of data?\\
	        What do you do next?
	   \end{column}
	   
	   
	    \end{columns}
	\end{frame}
	
	
	\begin{frame}{.}
	    \centering
	    \includegraphics[scale=.45]{Bild2}
	\end{frame}
	
	
	\begin{frame}{Topics}
	    \begin{columns}
	    
	    \begin{column}{.4\textwidth}
	    
	    
	    \begin{figure}
	        \includegraphics[scale=.35]{Bild3}
	    \end{figure}
	   \end{column}
	   
	   \begin{column}{.4\textwidth}
	        Topics For This Lecture 
            \begin{itemize}
                \item Understanding the Data
                \begin{itemize}
                    \item Data Cleaning 
                    \item Exploratory Data Analysis (EDA)
                    \item Basic data visualization
                \end{itemize}
                \item Common Data Anomalies 
                \begin{itemize}
                    \item … and how to fix them
                \end{itemize}
            \end{itemize}
	   \end{column}
	   
	   
	    \end{columns}
	\end{frame}
	
	
	
	\begin{frame}{.}
	    \centering
	    \includegraphics[scale=.38]{Bild4}
	\end{frame}
	
	
	\begin{frame}{Data Cleaning}
	    \begin{itemize}
	        \item The process of transforming raw data to facilitate subsequent analysis
	        \item Data cleaning often addresses issues
	        \begin{itemize}
	            \item structure / formatting
	            \item missing or corrupted values
	            \item unit conversion
	            \item encoding text as numbers
	            \item \dots
	        \end{itemize}
            \item Sadly, data cleaning is a big part of data science…

	    \end{itemize}
	\end{frame}
	
	
	
	
	\begin{frame}{.}
	    \centering
	    \includegraphics[scale=.7]{Bild5}
	\end{frame}
	
	
	
	\begin{frame}{.}
	    \centering
	    \includegraphics[scale=.38]{Bild4}
	\end{frame}
	
	
		\begin{frame}{Exploratory Data Analysis (EDA)}
		    \centering “Getting to know the data”
	    \begin{itemize}
	        \item The process of transforming, visualizing, and summarizing data to:
	        \begin{itemize}
	            \item Build/confirm understanding of the data and its provenance
	            \item Identify and address potential issues in the data
	            \item Inform the subsequent analysis
	            \item discover potential hypothesis … (be careful)
	        \end{itemize}
            \item EDA is an open-ended analysis
            \begin{itemize}
                \item Be willing to find something surprising
            \end{itemize}

	    \end{itemize}
	\end{frame}
	
	
	
	\begin{frame}
	    \centering
	    \includegraphics[scale=.38]{Bild6}
	\end{frame}
	
	
	
	\begin{frame}
	    \centering
	    \includegraphics[scale=.38]{Bild7}
	\end{frame}
\end{document}