\documentclass[aspectratio=169]{../latex_main/tntbeamer}  % you can pass all options of the beamer class, e.g., 'handout' or 'aspectratio=43'
\input{../latex_main/preamble}

\title[Introduction]{DS: Data Cleaning and EDA}
\subtitle{File Formats and Structure}

\graphicspath{ {./figure/} }
%\institute{}


\begin{document}
	
	\maketitle
	\begin{frame}{What should we look for?}
	    
	\end{frame}
	
	
	
	\begin{frame}{Key Data Properties to Consider in EDA}
	    \begin{itemize}
	        \item Structure -- the “shape” of a data file
	        \item Granularity -- how fine/coarse is each datum
	        \item Scope -- how (in)complete is the data
	        \item Temporality -- how is the data situated in time
	        \item Faithfulness -- how well does the data capture “reality”
	    \end{itemize}
	\end{frame}
	
	
	\begin{frame}{Key Data Properties to Consider in EDA}
	    \begin{itemize}
	        \item \textbf{Structure -- the “shape” of a data file}
	        \item Granularity -- how fine/coarse is each datum
	        \item Scope -- how (in)complete is the data
	        \item Temporality -- how is the data situated in time
	        \item Faithfulness -- how well does the data capture “reality”
	    \end{itemize}
	\end{frame}
	
	
	
	\begin{frame}{Key Data Properties to Consider in EDA}
	    \begin{itemize}
	        \item \textbf{Structure -- the “shape” of a data file}
	        \item Granularity -- how fine/coarse is each datum
	        \item Scope -- how (in)complete is the data
	        \item Temporality -- how is the data situated in time
	        \item Faithfulness -- how well does the data capture “reality”
	    \end{itemize}
	\end{frame}
	
	
	
	\begin{frame}{Rectangular Data}
	   \begin{columns}
	   
	   
	   \begin{column}{.7\textwidth}
	    
	   
	    We prefer rectangular data for data analysis (why?)
	    \begin{itemize}
	        \item Regular structures are easy manipulate and analyze
	        \item A big part of data cleaning is about transforming data to be more rectangular
	    \end{itemize}
	    
        Two kinds of rectangular data: Tables and Matrices \\
		\centering	(what are the differences?)
		\end{column}
		
	   \begin{column}{.3\textwidth}
	   \begin{figure}
	       		    \includegraphics[scale=.3]{Bild8}
	   \end{figure}

		\end{column}
		\end{columns}
	\end{frame}
	
	
	\begin{frame}{Rectangular Data}
	   \begin{columns}
	   
	   
	   \begin{column}{.7\textwidth}
	    
	   
	    We prefer rectangular data for data analysis (why?)
	    \begin{itemize}
	        \item Regular structures are easy manipulate and analyze
	        \item A big part of data cleaning is about transforming data to be more rectangular
	    \end{itemize}
	    
        Two kinds of rectangular data: Tables and Matrices \\
		\centering	(what are the differences?)
		\end{column}
		
	   \begin{column}{.3\textwidth}
	   \begin{figure}
	       		    \includegraphics[scale=.3]{Bild8}
	   \end{figure}

		\end{column}
		\end{columns}
		
	\begin{itemize}
		    \item[1.] Tables (a.k.a. data-frames  in R/Python and relations in SQL)
		    \begin{itemize}
		        \item Named columns with different types
		        \item Manipulated using data transformation languages (map, filter, group by, join, …)
		    \end{itemize}
		    \item[2.] Matrices
		    \begin{itemize}
		        \item Numeric data of the same type
		        \item Manipulated using linear algebra 
		    \end{itemize}
		\end{itemize}
	\end{frame}
	
	
	
	
	\begin{frame}{How are these data files formatted?}
	    \includegraphics[scale=.39]{Bild9}
	\end{frame}



    \begin{frame}{Comma and Tab Separated Values Files}
	   \begin{columns}
	   
	   
	   \begin{column}{.4\textwidth}
	    
	    \begin{itemize}
	        \item Tabular data where
	        \begin{itemize}
	            \item Records are delimited by a newline: “\n”, “\r\n”
	            \item Fields are delimited by ‘,’ (comma) or ‘\t’ (tab)
	        \end{itemize}
	        \item Very Common! 
	        \item Issues?
	        \begin{itemize}
	            \item Commas, tabs in records
	            \item Quoting
	            \item \dots
	        \end{itemize}
	    \end{itemize}
		\end{column}
		
	   \begin{column}{.4\textwidth}
	   \begin{figure}
	       		    \includegraphics[scale=.3]{Bild10}
	   \end{figure}

		\end{column}
		\end{columns}
	\end{frame}
	
	
	
	\begin{frame}{JavaScript Object Notation (JSON)}
	           \centering                                                                                           
	           \includegraphics[scale=.75]{Bild11}                               
	   \begin{itemize}
	       \item Widely used file format for nested data
	       \begin{itemize}
	           \item Very similar to python dictionaries
	           \item Strict formatting ”quoting” addresses some issues in CSV/TSV
	       \end{itemize}
	       \item Issues
	       \begin{itemize}
	           \item Not rectangular
	           \item Each record can have different fields
	           \item Nesting means records can contain tables – complicated
	       \end{itemize}
	   \end{itemize}
	\end{frame}
	
	
	
	\begin{frame}{Extensible Markup Language - XML (another kind of nested data)}
	                                                                                      
	           \includegraphics[scale=.4]{Bild12}                               
	\end{frame}
	
	
	
	
	\begin{frame}{Log Data}
	               Is this a csv file? tsv? JSON/XML?\\
                                                                       
	           \includegraphics[scale=.4]{Bild13}                               
	\end{frame}
\end{document}