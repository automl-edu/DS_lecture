\documentclass[aspectratio=169]{../latex_main/tntbeamer}  % you can pass all options of the beamer class, e.g., 'handout' or 'aspectratio=43'
\input{../latex_main/preamble}

\title[Introduction]{DS: Regular Expressions}
\subtitle{String Canonicalization}

\graphicspath{ {./figure/} }
%\institute{}


\begin{document}
	
	\maketitle
	\begin{frame}{Goal 1: Joining Tables with Mismatched Labels}
	    \includegraphics[scale=.38]{Bild1}\\
	    \pause
	    To join our tables we’ll need to canonicalize the county names.
	    \begin{itemize}
	        \item Canonicalize: Convert data that has more than one possible presentation into a standard form.
	    \end{itemize}
	\end{frame}
	
	
	
	\begin{frame}{Canonicalizing County Names}
	    \includegraphics[scale=.36]{Bild2}\\
	    
	\end{frame}
	
	
	\begin{frame}{Canonicalizing}
	    Canonicalizing
	    \begin{itemize}
	        \item Replace each string with a unique representation.
	        \item Feels very “hacky”, but messy problems often have messy solutions.
	    \end{itemize}
	    \begin{columns}
	    
	     \begin{column}{.6\textwidth}
	     
	    Tools used:
	    \begin{figure}
	          \includegraphics[scale=.36]{Bild4}
	    \end{figure}

	    
	     \end{column}
	     
	     
	     \begin{column}{.45\textwidth}
	     \begin{figure}
	         \includegraphics[scale=.5]{Bild3}
	     \end{figure}
	     
	      Can be done slightly better but not by much 
	    \begin{itemize}
	        \item Code is very brittle! Requires maintenance.
	    \end{itemize}
	    \end{column}
	    \end{columns}
	    
	\end{frame}
\end{document}