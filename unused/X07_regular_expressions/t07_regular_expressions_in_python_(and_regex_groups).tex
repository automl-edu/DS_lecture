\documentclass[aspectratio=169]{../latex_main/tntbeamer}  % you can pass all options of the beamer class, e.g., 'handout' or 'aspectratio=43'
\input{../latex_main/preamble}

\title[Introduction]{DS: Regular Expressions}
\subtitle{Regular Expressions in Python (and Regex Groups)}

\graphicspath{ {./figure/} }
%\institute{}


\begin{document}
	
	\maketitle
	\begin{frame}{re.findall in Python}
	    In Python, re.findall(pattern, text) will return a list of all matches.
	    \includegraphics[scale=.35]{Bild19}
	\end{frame}
	
	
	\begin{frame}{re.sub in Python}
    	In Python, re.sub(pattern, repl, text) will return text with all instances of pattern replaced by repl.
	    \includegraphics[scale=.35]{Bild20}
	\end{frame}
	
	
	\begin{frame}{Raw Strings in Python}
    	Note: When specifying a pattern, we strongly suggest using “raw strings”. 
    	\begin{itemize}
    	    \item A raw string is created using r“” or r‘’ instead of just “” or ‘’.
    	    \item The exact reason is a bit tedious.
    	    \begin{itemize}
    	        \item Rough idea: Regular expressions and Python strings both use \textbackslash as an escape character. 
    	        \item Using non-raw strings leads to uglier regular expressions.
    	    \end{itemize}
    	\end{itemize}
    	\begin{figure}
    	    \centering
    	    \includegraphics[scale=.5]{Bild21}
    	\end{figure}
	    
	    
	    
	    For more information see “The Backslash Plague” under \\
	    \url{https://docs.python.org/3/howto/regex.html}.
	\end{frame}
	
	
	\begin{frame}{Regular Expression Groups}
    	Earlier we used parentheses to specify the order of operations.\\
    	Parenthesis have another meaning:
    	\begin{itemize}
    	    \item Every set of parentheses specifies a so-called “group”. 
    	    \item Regular expression matchers (e.g. re.findall, \url{regex101.com}) will return matches organized by groups. In Python, returned as tuples.
    	\end{itemize}
    	\begin{figure}
    	    \centering
    	    \includegraphics[scale=.35]{Bild22}
    	\end{figure}
	\end{frame}
	
	
	
	\begin{frame}{Regex Puzzle}
    	Fill in the regex below so that after code executes, day is “26”, month is “Jan”, and year is “2014”. 
    	\begin{itemize}
    	    \item See lec08-working-with-text.ipynb or \url{https://tinyurl.com/reg913s}.
    	\end{itemize}
    	\begin{figure}
    	    \centering
    	    \includegraphics[scale=.6]{Bild23}
    	    \includegraphics[scale=.4]{Bild27}
    	\end{figure}
	\end{frame}
	
	
	\begin{frame}{Regex Puzzle (One Possible Solution) }
    	Fill in the regex below so that after code executes, day is “26”, month is “Jan”, and year is “2014”. 
    	\begin{figure}
    	    \centering
    	    \includegraphics[scale=.6]{Bild24}
    	    \includegraphics[scale=.4]{Bild27}
    	\end{figure}
	\end{frame}
	
	
	
	\begin{frame}{Extracting Date Information}
    	With a little more work, we can do something similar and extract day, month, year, hour, minutes, seconds, and time zone all in one regular expression.
    	\begin{itemize}
    	    \item Derivation is left as an exercise for you
    	\end{itemize}
    	\begin{figure}
    	    \centering
    	    \includegraphics[scale=.6]{Bild25}
    	\end{figure}
    	You will also see code that uses re.search instead of re.findall.
    	\begin{itemize}
    	    \item Beyond the scope of our lecture today.
    	\end{itemize}
    	\begin{figure}
    	    \centering
    	    \includegraphics[scale=.65]{Bild26}
    	\end{figure}
	\end{frame}
\end{document}