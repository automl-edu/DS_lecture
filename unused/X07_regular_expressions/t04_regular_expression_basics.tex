\documentclass[aspectratio=169]{../latex_main/tntbeamer}  % you can pass all options of the beamer class, e.g., 'handout' or 'aspectratio=43'
\input{../latex_main/preamble}

\title[Introduction]{DS: Regular Expressions}
\subtitle{Regular Expression Basics}

\graphicspath{ {./figure/} }
%\institute{}


\begin{document}
	
	\maketitle
	\begin{frame}{Extracting Date Information}
	    Earlier we saw that we can hack together code that uses split to extract info:\\
	    \begin{figure}
	        \centering
	        \includegraphics[scale=.35]{Bild6}
	    \end{figure}
	    An alternate approach is to use a so-called “regular expression”:
        \begin{itemize}
            \item Implementation provided in the re library built into Python
            \item We’ll spend some time today working up to expressions like shown 
        \end{itemize}
        \begin{figure}
	        \centering
	        \includegraphics[scale=.6]{Bild7}
	    \end{figure}
	\end{frame}
	
	
	
	
	\begin{frame}{Regular Expressions}
	    A formal language is a set of strings, typically described implicitly.
	    \begin{itemize}
	        \item Example: “The set of all strings of length < 10 that contain ‘horse’”
	    \end{itemize}
	    A regular language is a formal language that can be described by a regular expression (which we will define soon). 
        \begin{figure}
	        \centering
	        \includegraphics[scale=.44]{Bild8}
	    \end{figure}
	\end{frame}
	
	
	
	\begin{frame}{\url{Regex101.com} (or the online tutorial \url{regexone.com})}
	    There are a ton of nice resources out there to experiment with regular expressions (e.g. \url{Regex101.com}, \url{regexone.com}, sublime text, python, etc).\\
	    I recommend trying out \url{Regex101.com}, which provides a visually appealing and easy to use platform for experimenting with regular expressions.
        \begin{itemize}
            \item Example: \url{https://regex101.com/r/1SREie/1}
        \end{itemize}
        \begin{figure}
	        \centering
	        \includegraphics[scale=.6]{Bild9}
	    \end{figure}
	\end{frame}
	
	
	
	\begin{frame}{Regular Expression Syntax}
	   The four basic operations for regular expressions.
        \begin{itemize}
            \item Can technically do anything with just these basic four (albeit tediously). 
        \end{itemize}
        \begin{figure}
	        \centering
	        \includegraphics[scale=.35]{Bild10}
	    \end{figure}
	\end{frame}
	
	
	
	\begin{frame}{Regular Expression Syntax}
	   AB*: A then zero or more copies of B: A, AB, ABB, ABBB\\
       (AB)*: Zero or more copies of AB: ABABABAB,  ABAB, AB,  

        \begin{figure}
	        \centering
	        \includegraphics[scale=.35]{Bild10}
	    \end{figure}
	\end{frame}
	
	
	
	\begin{frame}{Puzzle: Use regex101.com to test! Or \url{tinyurl.com/reg913z}}
 

        \begin{figure}
	        \centering
	        \includegraphics[scale=.35]{Bild10}
	    \end{figure}
	    Give a regular expression that matches moon, moooon, etc. Your expression should match any even number of os except zero (i.e. don’t match mn).

	\end{frame}
	
	
	\begin{frame}{Puzzle Solution}
 

        \begin{figure}
	        \centering
	        \includegraphics[scale=.35]{Bild10}
	    \end{figure}
	    Solution to puzzle on previous slide: moo(oo)*n

	\end{frame}
	
	
	
	\begin{frame}{Regular Expression moo(oo)*n: \url{https://tinyurl.com/reg913m}}
 

        \begin{figure}
	        \centering
	        \includegraphics[scale=.35]{Bild10}
	    \end{figure}
	    Give a regex that matches muun, muuuun, moon, moooon, etc. Your expression should match any even number of us or os except zero (i.e. don’t match mn).


	\end{frame}
	
	
	
	\begin{frame}{Puzzle Solution}
 

        \begin{figure}
	        \centering
	        \includegraphics[scale=.35]{Bild10}
	    \end{figure}
	    Solution to puzzle on previous slide: m(uu(uu)*|oo(oo)*)n
        \begin{itemize}
            \item Note: m(uu(uu)*)|(oo(oo)*)n is not correct! OR must be in parentheses!
        \end{itemize}

	\end{frame}
	
	
	
	\begin{frame}{Order of Operations in Regexes}
 
        m(uu(uu)*|oo(oo)*)n
        \begin{itemize}
            \item Matches starting with m and ending with n, with either of the following in the middle:
            \begin{itemize}
                \item uu(uu)* \hspace{3cm} Match examples: muun, muuuun
                \item oo(oo)*  \hspace{3cm} Match examples: moon, moooon  
            \end{itemize}
        \end{itemize}
        \pause
        m(uu(uu)*)|(oo(oo)*)n\\
        \bigskip
        \begin{itemize}
            \item Matches either of the following
            \begin{itemize}
                \item m followed by uu(uu)*  \hspace{3cm} Match examples: muu, muuuu
                \item oo(oo)* followed by n  \hspace{3cm} Match examples: oon, oooon  
            \end{itemize}
        \end{itemize}
        \bigskip
        In regexes | comes last.

	\end{frame}
\end{document}