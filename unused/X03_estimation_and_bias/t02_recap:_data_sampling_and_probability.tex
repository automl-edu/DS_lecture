\documentclass[aspectratio=169]{../latex_main/tntbeamer}  % you can pass all options of the beamer class, e.g., 'handout' or 'aspectratio=43'
\input{../latex_main/preamble}

\title[Introduction]{DS: Estimation and Bias}
\subtitle{Recap: Data Sampling and Probability}

\graphicspath{ {./figure/} }
%\institute{}


\begin{document}
	
	\maketitle
	\begin{frame}{Key Concepts in Sampling}
	    \begin{columns}
	        \begin{column}{.4\textwidth}
	           \begin{figure}
	               \includegraphics[scale=.4]{Bild2}
	           \end{figure}
	        \end{column}
	        
	        \begin{column}{.4\textwidth}
	            
	            Population: the set of all units of interest, size N.\\
	            \bigskip
	            \bigskip
	            \bigskip
	            \bigskip
	            \bigskip
	        
	            Sampling frame: the set of all possible units that can be drawn into the sample\\
	            \bigskip
	            \bigskip
	            Sample: a subset of the sampling frame, size n.
	        \end{column}
	        
	    \end{columns}
	    
	\end{frame}
	
	
	\begin{frame}{Scenario 1: A census}
	    \begin{columns}
	        \begin{column}{.4\textwidth}
	           \begin{figure}
	               \includegraphics[scale=.3]{Bild3}
	           \end{figure}
	        \end{column}
	        
	        \begin{column}{.4\textwidth}
	            Key Features
                \begin{itemize}
                    \item population = sample = sampling frame
                    \item Pros:
                    \begin{itemize}
                        \item Lots of data
                        \item No selection bias
                        \item Easy inference
                    \end{itemize}
                    \item Cons:
                    \begin{itemize}
                        \item Expensive (time, money)
                        \item Often impossible

                    \end{itemize}
                \end{itemize}
	        \end{column}
	        
	    \end{columns}
	    
	\end{frame}
	
	
	\begin{frame}{Scenario 2: Administrative Data}
	    \begin{columns}
	        \begin{column}{.4\textwidth}
	           \begin{figure}
	               \includegraphics[scale=.8]{Bild4}
	           \end{figure}
	        \end{column}
	        
	        \begin{column}{.4\textwidth}
	            Key Features
                \begin{itemize}
                    \item Sampling frame contains a lot not in population
                    \item Have access to entire frame
                \end{itemize}
	        \end{column}
	        
	    \end{columns}
	    
	\end{frame}
	
	
	\begin{frame}{Scenario 3: What we like to think we have}
	    \begin{columns}
	        \begin{column}{.4\textwidth}
	           \begin{figure}
	               \includegraphics[scale=.75]{Bild5}
	           \end{figure}
	        \end{column}
	        
	        \begin{column}{.4\textwidth}
	            Key Feature
                \begin{itemize}
                    \item Optimistic sense that sample is representative of population
                \end{itemize}
	        \end{column}
	        
	    \end{columns}
	    
	\end{frame}
	
	
		\begin{frame}{Scenario 4: What we usually have}
	    \begin{columns}
	        \begin{column}{.4\textwidth}
	           \begin{figure}
	               \includegraphics[scale=.35]{Bild6}
	           \end{figure}
	        \end{column}
	        
	        \begin{column}{.4\textwidth}
	            Key Feature
                \begin{itemize}
                    \item Sample may be drawn from a skewed frame and may not be representative of population
                \end{itemize}
	        \end{column}
	        
	    \end{columns}
	    
	\end{frame}
	
	\begin{frame}{Case study – 1936 Presidential Election}
	    \begin{columns}
	        \begin{column}{.6\textwidth}
	           \begin{figure}
	               \includegraphics[scale=.33]{Bild7}
	           \end{figure}
	        \end{column}
	        
	        \begin{column}{.4\textwidth}
	            Q: What was the population?\\
	            A: All people who will cast votes in the 1936 Presidential election\\
	            \bigskip
	            Selection bias: systematically favoring (or excluding) certain groups for inclusion in the sample\\
	            \bigskip
	            Non-response bias: when people who don’t respond are non-representative of the population.

	        \end{column}
	        
	    \end{columns}
	    
	\end{frame}
	
	
	
		\begin{frame}{Data Quality vs. Data Quantity.}
	    \begin{columns}
	        \begin{column}{.5\textwidth}
	           \begin{figure}
	               \includegraphics[scale=.37]{Bild8}
	           \end{figure}
	        \end{column}
	        
	        \begin{column}{.5\textwidth}
	            Literary Digest 1936 poll: n = 10,000,000
                US population 1936: N = 128,000,000\\
                \hspace{1cm}-> 8\%! \\
	            \bigskip
	            Gallup 1936 poll: n = 50,000\\
                Gallup 2021: n = 1,000\\
	            \bigskip
	            Non-response bias: when people who don’t respond are non-representative of the population.
	            \bigskip
	            into the sample. The typical sample size for a Gallup poll, either a traditional stand-alone poll or one night's interviewing from Gallup's Daily tracking, is 1000
	        \end{column}
	        
	    \end{columns}
	    
	\end{frame}
	
	
	\begin{frame}{Case Study - Gender diversity in Data Science}
	   Question: What proportion of Data 100 students identify as female?
	   \begin{figure}
	       \centering
	       \includegraphics[scale=.4]{Bild9}
	   \end{figure}
	    
	\end{frame}
\end{document}