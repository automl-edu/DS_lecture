\documentclass[aspectratio=169]{../latex_main/tntbeamer}  % you can pass all options of the beamer class, e.g., 'handout' or 'aspectratio=43'
\input{../latex_main/preamble}

\title[Introduction]{DS: Logistic Regression, Classification}
\subtitle{Decision boundaries}

\graphicspath{ {./figure/} }
%\institute{}


\begin{document}
	
	\maketitle
	\begin{frame}{Decision boundaries}
	    \begin{columns}
	        \begin{column}{.5\textwidth}
	            Consider our original single-feature model.\\
	            \bigskip
	            $P(Y=1|x) = \sigma (\theta_1 \cdot FG\_PCT\_DIFF)$\\
	            \bigskip
	            The grey dots are true observations from the 2017-18 NBA season.
	             
	        \end{column}
	        
	        
	        \begin{column}{.5\textwidth}
	                \begin{figure}
	                    \centering
	                    \includegraphics[scale=.55]{Bild33}\\
	                \end{figure}
	        \end{column}
	        
	    \end{columns}
	\end{frame}
	
	\begin{frame}{Decision boundaries}
	    \begin{columns}
	        \begin{column}{.5\textwidth}
	            If we pick a threshold, e.g. T = 0.3, we get a predicted class from probabilities.
	        \end{column}
	        
	        
	        \begin{column}{.5\textwidth}
	                \begin{figure}
	                    \centering
	                    \includegraphics[scale=.55]{Bild34}\\
	                \end{figure}
	        \end{column}
	        
	    \end{columns}
	\end{frame}
	
	\begin{frame}{Decision boundaries }
	    \begin{columns}
	        \begin{column}{.5\textwidth}
	            If we pick a threshold, e.g. T = 0.3, we get a predicted class from probabilities.
	            \begin{itemize}
	                \item The effect is that x < xT predicts class 0, and x > xT predicts class 1.
	                \item $x_T$ is known as a decision boundary.
	                \begin{itemize}
	                    \item $x_T$ is a function of model parameters and T.

	                \end{itemize}
	            \end{itemize}
	        \end{column}
	        
	        
	        \begin{column}{.5\textwidth}
	                \begin{figure}
	                    \centering
	                    \includegraphics[scale=.35]{Bild35}\\
	                \end{figure}
	        \end{column}
	        
	    \end{columns}
	\end{frame}
	
	
	\begin{frame}{Decision boundaries for 2D models}
	
	Now consider our “better” model, $ P(Y=1|x) = \sigma (\theta_0  + \theta_1 \cdot FG\_PCT\_DIFF + \theta_2 \cdot PF\_DIFF)$                                                    . It is drawn below with the thresholding line T = 0.3. What does its decision boundary look like?\\
        \includegraphics[scale=.5]{Bild36}
	     
	\end{frame}
	
	
	\begin{frame}{Decision boundaries for 2D models}
	
	Now consider our “better” model, $ P(Y=1|x) = \sigma (\theta_0  + \theta_1 \cdot FG\_PCT\_DIFF + \theta_2 \cdot PF\_DIFF)$                                                    . It is drawn below with the thresholding line T = 0.3. . The decision boundary is linear. \\
	
        \begin{figure}
            \centering
            \includegraphics[scale=.35]{Bild37}
        \end{figure}
	     
	\end{frame}
	
	\begin{frame}{Decision boundaries for 2D models}
	Suppose we minimized mean cross-entropy loss to determine the optimal model parameters for this model, and found\\
	$P(Y=1|x) = \sigma (0.035 + 34.705\cdot FG\_PCT\_DIFF - 0.160\cdot PF\_DIFF)$\\
	\bigskip
	If we set T = 0.3, our decision boundary is \\
	$\sigma (0.035 + 34.705\cdot FG\_PCT\_DIFF - 0.160\cdot PF\_DIFF) = 0.3$\\
	\bigskip
	Which simplifies to
	%$0.035 + 34.705\cdot FG\_PCT\_DIFF - 0.160\cdot PF\_DIFF = \sigma^{-1}(0.3)$
	
        \begin{figure}
            \centering
            \includegraphics[scale=.25]{Bild38}
        \end{figure}
	     
	\end{frame}
	
	
	\begin{frame}{Decision boundaries for 2D models}
	If we overlay our true observations onto our decision boundary, we can get a rough sense of the accuracy of our model and the types of errors it makes.\\
	\begin{itemize}
	    \item Blue points in the orange region are false positives.
	    \item Orange points in the blue region are false negatives.
	\end{itemize}
	
        \begin{figure}
            \centering
            \includegraphics[scale=.35]{Bild39}
        \end{figure}
	     
	\end{frame}
	
	
\end{document}