\documentclass[aspectratio=169]{../latex_main/tntbeamer}  % you can pass all options of the beamer class, e.g., 'handout' or 'aspectratio=43'
\input{../latex_main/preamble_2}

\title[Cleaning]{DS: Data Cleaning}
\subtitle{Cleaning}

\graphicspath{ {./figure/} }
%\institute{}


\begin{document}
	
	\maketitle

 \begin{frame}[c]{Motivation}

\begin{itemize}
    \item \textbf{Rules:}
    \begin{itemize}
        \item \textit{Example 1:} The city determines the state
        \begin{itemize}
            \item (t4 and t6 violate this rule)
        \end{itemize}
        \item \textit{Example 2:} Among employees having the same role, salaries in NYC should be higher
        \begin{itemize}
            \item (t5 and t6 violate this rule)
        \end{itemize}
    \end{itemize}
    \item Traditional methods apply rules one by one
\end{itemize}

\centering\includegraphics[width=0.8\textwidth]{figure/bild18_table}

\end{frame}

\begin{frame}[c]{Goal/Problem}

\begin{itemize}
    \item Correct violations for different types of constraints with desirable repairs
    \item $\leadsto$ {Desirability} depends on a cost model such as minimizing the number of changes
    \item How can we compile heterogeneous rules into the same language and efficiently apply them holistically together?
\end{itemize}

\end{frame}

\begin{frame}[c]{Denial Constraints}

\begin{itemize}
    \item The method accepts \textbf{Denial Constraints (DCs)} as input
    \item DCs are a declarative specification of the quality rules
\end{itemize}

\centering\includegraphics[width=0.8\textwidth]{figure/bild19_constraints}

\vspace{-2cm}

\begin{itemize}
    \item \textbf{Example Data:}
    \end{itemize}
    \begin{center}
        \begin{tabular}{|c|c|c|c|c|c|c|c|}
            \hline
            GID & FN & LN & ROLE & CITY & AC & ST & SAL \\
            \hline
            t3 & 102 & Paul J. Smith & V & SJ & 639 & CA & 100 \\
            t4 & 105 & Anne Nash & M & NYC & 234 & NY & 110 \\
            t5 & 211 & Mark White & E & SJ & 639 & CA & 80 \\
            t6 & 386 & Mark Lee & E & NYC & 552 & AZ & 75 \\
            \hline
        \end{tabular}
    \end{center}

\end{frame}

\begin{frame}[c]{System Overview}

\begin{itemize}
    \item \textbf{Conflict Hypergraph (CH)}: Encodes constraint violations
    \item \textbf{Repair Context (RC)}: Encodes violation repairs
\end{itemize}

\centering\includegraphics[width=0.6\textwidth]{figure/bild20_systems}

\end{frame}

\begin{frame}[c]{Violations Representation: Conflict Hypergraph}

\begin{itemize}
    \item The nodes represent the violating cells
    \item The edges link cells involved in the same violation
\end{itemize}

    \centering\includegraphics[width=0.9\textwidth]{figure/bild21_graphs}

\end{frame}

\begin{frame}[c]{Fixing Violation Holistically: Repair Context}

\begin{itemize}
    \item Start from cells identified by the minimum vertex cover as likely to be changed, e.g., \( t2[C] \)
\end{itemize}
    
    \begin{columns}
    
        \column{0.6\textwidth}

    \begin{enumerate}
        \item Start with \( t2[C] \) in queue; involved in 3 hyper-edges
        \item \textbf{Consider \( e_1 \):}
        \begin{itemize}
            \item Add repair \( t2[C] = t1[C] \); Add \( t1[C] \) to queue
        \end{itemize}
        \item \textbf{Consider \( e_2 \):}
        \begin{itemize}
            \item Add repair \( t2[C] = t3[C] \); Add \( t3[C] \) to queue
        \end{itemize}
        \item \textbf{Consider \( e_3 \):}
        \begin{itemize}
            \item Add repair \( t2[C] = t4[C] \); Add \( t4[C] \)
        \end{itemize}
        \item \textbf{Considering \( e_4 \):}
        \begin{itemize}
            \item Add repair \( t4[C] = 5 \)
        \end{itemize}
    \end{enumerate}
    
        \column{0.4\textwidth}
    
        \centering\includegraphics[width=.8\textwidth]{figure/bild22_graphs2}
    
    \end{columns}


\end{frame}

\end{document}