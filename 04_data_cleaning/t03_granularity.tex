\documentclass[aspectratio=169]{../latex_main/tntbeamer}  % you can pass all options of the beamer class, e.g., 'handout' or 'aspectratio=43'
\input{../latex_main/preamble_2}

\title[Data Property: Granularity \& Scope]{DS: Data Cleaning}
\subtitle{Data Property: Granularity \& Scope}

\graphicspath{ {./figure/} }
%\institute{}

\date{\hspace{0.5em}{\includegraphics[height=1.5em]{../latex_main/figures/Cc-by-nc-sa_icon.svg.png}}; extension of DS100 and Ziawasch Abedjan}

\begin{document}
	
	\maketitle

\begin{frame}[c]{Key Data Properties to Consider in EDA}
    \begin{itemize}
        \item {Structure} -- the “shape” of a data file.
        \item \textbf{Granularity} -- how fine/coarse is each datum.
        \item {Scope} -- how (in)complete is the data.
        \item {Temporality} -- how is the data situated in time.
        \item {Faithfulness} --how well does the data capture “reality”.
    \end{itemize}
\end{frame}

% Slide 29 - Granularity
\begin{frame}[c]{Granularity}

    \begin{itemize}
        \item What does each record represent?
        \begin{itemize}
            \item Examples: a purchase, a person, a group of users
            \item $\leadsto$ Each of these represents a different level of granularity
        \end{itemize}
        \pause
        \item Do all records capture granularity at the same level?
        \begin{itemize}
            \item Some data will include summaries (aka rollups) as records
            \item $\leadsto$ Pay attention when reasoning over different granularity levels
        \end{itemize}
        \pause
        \item If the data are coarse how was it aggregated?
        \begin{itemize}
            \item Sampling, averaging, …
            \item $\leadsto$ This can be important to know for interpreting the data
        \end{itemize}
    \end{itemize}
\end{frame}

\begin{frame}{Importance of Data Granularity}

\begin{itemize}
    \item {Informs Analysis Precision}
    \begin{itemize}
        \item Determines the level of detail for insights (e.g., daily vs. monthly data).
    \end{itemize}
\pause
    \item {Ensures Relevance and Context}
    \begin{itemize}
        \item Helps align the data's detail level with decision-making needs.
    \end{itemize}
\pause
    \item {Facilitates Data Aggregation and Transformation}
    \begin{itemize}
        \item Allows for effective data summarization and aggregation.
    \end{itemize}
\pause
    \item {Impacts Data Storage and Performance}
    \begin{itemize}
        \item Fine-grained data requires more storage and affects processing speed.
        \item $\leadsto$ Typically, you have more table rows if you consider fine-grained data
    \end{itemize}
\pause
    \item {Improves Data Quality and Consistency}
    \begin{itemize}
        \item Ensures compatibility when merging datasets of different granularity.
    \end{itemize}
\end{itemize}

\end{frame}

\begin{frame}[c]{Key Data Properties to Consider in EDA}
    \begin{itemize}
        \item {Structure} -- the “shape” of a data file.
        \item {Granularity} -- how fine/coarse is each datum.
        \item \textbf{Scope} -- how (in)complete is the data.
        \item {Temporality} -- how is the data situated in time.
        \item {Faithfulness} --how well does the data capture “reality”.
    \end{itemize}
\end{frame}


% Slide 31 - Scope
\begin{frame}[c]{Scope}
    \begin{itemize}
        \item  Does my data cover my area of interest?
        \begin{itemize}
            \item Example: I am interested in studying crime in Hannover but I only have Frankfurt crime data.
        \end{itemize}
        \pause
        \item  Is my data too expansive?
        \begin{itemize}
            \item Example: I am interested in student grades for Data Science Foundations but have student grades for all statistics classes
            \item Solution: Filtering? -- Implications on the sample?\\
            $\leadsto$ If the data is a sample I may have poor coverage after filtering
        \end{itemize}
        \pause
        \item Does my data cover the right time frame?
        \begin{itemize}
            \item  More on this in temporality...
        \end{itemize}
    \end{itemize}

\end{frame}


% Slide 32 - Sampling Frame
\begin{frame}[c]{Revisiting the Sampling Frame}
    \begin{itemize}
        \item \alert{Reminder:} The sampling frame is the population from which the data was sampled.
        \begin{itemize}
            \item Note that this may not be the population of interest.
        \end{itemize}
        \item How complete/incomplete is the frame (and its data)?
        \item How is the frame/data situated in place?
        \item How well does the frame/data capture reality?
        \item How is the frame/data situated in time?
    \end{itemize}
\end{frame}



\end{document}