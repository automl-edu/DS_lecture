\documentclass[aspectratio=169]{../latex_main/tntbeamer}  % you can pass all options of the beamer class, e.g., 'handout' or 'aspectratio=43'
\input{../latex_main/preamble}

\title[Visualization]{DS: Visualization}
\subtitle{Scale}

\graphicspath{ {./figure/} }
%\institute{}


\begin{document}
	
	\maketitle
	\begin{frame}{Case Study: Planned Parenthood Hearing}
	    \begin{columns}
	        \begin{column}{.5\textwidth}
	        
	                    \centering
	                    \includegraphics[scale=.65]{Bild51}
	        \end{column}
	        
	        \begin{column}{.5\textwidth}
	        
	                In 2015, Planned Parenthood was accused of selling aborted fetal tissue for profit.\\
	                Congressman Chaffetz (R-UT) showed this plot which originally appeared in a report by Americans United for Life.
	                \begin{itemize}
	                    \item What is this graph plotting?
	                    \item What message is this plot trying to convey?
	                    \item Is anything suspicious?	
	                \end{itemize}
	                
	        \end{column}
	    \end{columns}
	\end{frame}
	
	
	\begin{frame}{Keep axis scales consistent}
	    \begin{columns}
	        \begin{column}{.45\textwidth}
	        
	                    \centering
	                    \includegraphics[scale=.65]{Bild51}
	        \end{column}
	        
	        \begin{column}{.55\textwidth}
	        
	                The scales for the two lines are completely different
	                \begin{itemize}
	                    \item 327000 is smaller than 935573,\\ but appears to be way bigger.
	                    \item Be careful when you use two different scales for the same axis!
	                \end{itemize}
	        \end{column}
	        
	    \end{columns}
	\end{frame}
	
	
	
	\begin{frame}{Consider scale of the data}
	
	    \vspace{-2em}
	    \begin{columns}
	        \begin{column}{.55\textwidth}
	        
	            The top plot draws all of the data on the same scale.
	            \begin{itemize}
	                \item It clearly shows there was a dramatic drop in cancer screenings by PP.
	                \item But there are still far more cancer screenings than abortions.
	                \item  We can plot percentage change instead of raw counts (bottom). This shows that cancer screenings have decreased and abortions have increased, without being misleading.

	            \end{itemize} 
	        \end{column}
	        
	        
	        \begin{column}{.4\textwidth}

	                    \centering
	                    \includegraphics[scale=.35]{Bild52}
	                    
	        \end{column}
	    \end{columns}
	\end{frame}
	
	
	\begin{frame}{Consider scale of the data}
	
	    \begin{columns}
	        \begin{column}{.5\textwidth}

	                    \centering
	                    \includegraphics[scale=.6]{Bild53}

	        \end{column}
	        
	        
	        \begin{column}{.4\textwidth}
	        
	              We could also visualize abortions and cancer screenings as a percentage of total procedures.
	              \begin{itemize}
	                  \item Abortions increased from 13\% to 26\% of total procedures.
	              \end{itemize}
	        \end{column}
	    \end{columns}
	\end{frame}
	
	
	\begin{frame}{Reveal the data}
	
	    \begin{columns}
	        \begin{column}{.5\textwidth}
	        
	                \begin{itemize}
	                    \item Choose axis limits to fill the visualization.
	                    \item If necessary:
	                    \begin{itemize}
	                        \item Zoom in on the bulk of the data.
	                        \item Create multiple plots to show different regions of interest.
	                    \end{itemize}
	                \end{itemize}
	                On the left, the bulk of the data is in the [0, 10] range on the x-axis.
	        \end{column}
	        
	        
	        \begin{column}{.5\textwidth}
	        
	                    \centering
	                    \includegraphics[scale=.38]{Bild54}
	                    
	        \end{column}
	    \end{columns}
	\end{frame}
\end{document}