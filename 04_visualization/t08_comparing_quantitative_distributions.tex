\documentclass[aspectratio=169]{../latex_main/tntbeamer}  % you can pass all options of the beamer class, e.g., 'handout' or 'aspectratio=43'
\input{../latex_main/preamble}

\title[Visualization]{DS: Visualization}
\subtitle{Comparing Quantitative Distributions}

\graphicspath{ {./figure/} }
%\institute{}


\begin{document}
	
	\maketitle
	\begin{frame}{Overlaid histograms and density curves}
	    \includegraphics[scale=.35]{Bild42}\\
	    We can overlay multiple histograms and density curves on top of one another.
	    \begin{itemize}
	        \item First: Not terrible, but looks like three separate histograms.
	        \item Second: Has the most information, but isn’t very clear!
	        \item Third: Rough estimate of both distributions, but is the most clear by far.
	        \item Neither will generalize well to three or more distributions.
	    \end{itemize}
    \end{frame}
    
    \begin{frame}{Overlaid eCDFs}
        \includegraphics[width=.45\textwidth]{./figure/cdf_iris_sepal_length_shifted.png}
	    \includegraphics[width=.45\textwidth]{./figure/cdf_iris_sepal_length_shifted_scaled.png}\\
	    
	    We can also overlay eCDFs
	    \begin{itemize}
	        \item First plot: shifted distributions
	        \item Second plot: shifted and scaled distribution
	        \item Crossing lines are often interesting (as in the second plot)
	        \item[$\leadsto$] more probability mass in certain ranges
	    \end{itemize}
    \end{frame}
	
	
	\begin{frame}{Side by side box plots and violin plots}
	    \centering
	    \includegraphics[scale=.4]{Bild43}
	\end{frame}
	
	
	\begin{frame}{Side by side box plots and violin plots}
	    \begin{columns}
            \begin{column}{.5\textwidth}

                       \includegraphics[scale=.4]{Bild44}

            \end{column}
            
            \begin{column}{.5\textwidth}
               Box plots and violin plots are concise, and thus are well suited to be stacked side by side to compare multiple distributions at once.
               \begin{itemize}
                   \item At a glance, we can tell that the median birth weight is higher for babies whose mothers did not smoke while pregnant (“False”).
                   \item The violin plot shows us the bimodal nature of the “True” category.
               \end{itemize}
            \end{column}
        \end{columns}
	\end{frame}
\end{document}