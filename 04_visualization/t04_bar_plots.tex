\documentclass[aspectratio=169]{../latex_main/tntbeamer}  % you can pass all options of the beamer class, e.g., 'handout' or 'aspectratio=43'
\input{../latex_main/preamble}

\title[Visualization]{DS: Visualization, Part 1}
\subtitle{Bar plots}

\graphicspath{ {./figure/} }
%\institute{}


\begin{document}
	
	\maketitle
	\begin{frame}{Bar plots}
	    \begin{itemize}
	        \item Bar plots are the most common way of displaying the distribution\\ of a qualitative (categorical) variable.
	        \begin{itemize}
	            \item For example, the proportion of adults in the upper, middle, and lower classes.
	        \end{itemize}
	        \item They are also used to display a numerical variable that has been measured on individuals in different categories
	        \begin{itemize}
	            \item For example, the average grades of students at Berkeley in several majors.
	            \item Not a distribution! But bar plots still make sense.
	        \end{itemize}
	        \item Lengths encode values.
	        \begin{itemize}
	            \item Widths encode nothing!
	            \item Color could indicate a sub-category (but not necessarily).
	        \end{itemize}
	    \end{itemize}
	\end{frame}
	
	
	
	\begin{frame}{Example dataset}
	    We will be using the baby weights dataset for most of our plots.
	    \begin{figure}
	        \centering
            \includegraphics[scale=.45]{Bild21}
	    \end{figure}
	\end{frame}
	
	
	\begin{frame}{Bar plots}
	    Suppose births[‘Maternal Smoker’] is a series containing True and False. Then:
	    \begin{figure}
	        \centering
            \includegraphics[scale=.4]{Bild22}
	    \end{figure}
	\end{frame}
	
	
	
	\begin{frame}{Bar plots}
	    Suppose we have a list of majors and a list of GPAs (i.e., grades) corresponding to those majors.
	    \begin{figure}
	        \centering
            \includegraphics[scale=.4]{Bild23}
	    \end{figure}
	\end{frame}
	
\end{document}