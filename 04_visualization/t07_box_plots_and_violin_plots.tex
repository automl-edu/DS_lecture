\documentclass[aspectratio=169]{../latex_main/tntbeamer}  % you can pass all options of the beamer class, e.g., 'handout' or 'aspectratio=43'
\input{../latex_main/preamble}

\title[Visualization]{DS: Visualization}
\subtitle{Box plots and violin plots}

\graphicspath{ {./figure/} }
%\institute{}


\begin{document}
	
	\maketitle
	\begin{frame}{Quartiles}
	    \vspace{-2em}
	     \begin{columns}
            \begin{column}{.5\textwidth}
                For a quantitative variable:
                \begin{itemize}
                    \item First or lower quartile: 25th percentile
                    \item Second quartile: 50th percentile (median)
                    \item Third or upper quartile: 75th percentile
                \end{itemize}
                The interval [first quartile, third quartile] contains the “middle 50\%" of the data.\\
                \smallskip
                Interquartile range (IQR) measures spread.
                \begin{itemize}
                    \item IQR = third quartile – first quartile.
                \end{itemize}
                Interquantile mean (IQM):
                \begin{itemize}
                    \item IQM = mean(IQR)
                    \item compromise between mean and median
                    \item[$\leadsto$] more robust against outliers
                    \item[$\leadsto$] not influenced by two modes
                \end{itemize}

            \end{column}
            
            
            \begin{column}{.5\textwidth}

                       \includegraphics[scale=.6]{Bild37}

            \end{column}
        \end{columns}
	\end{frame}
	
	
	\begin{frame}{Box plots}
	     \begin{columns}
            \begin{column}{.5\textwidth}
            
                       \includegraphics[scale=.5]{Bild38}

            \end{column}
            
            
            \begin{column}{.5\textwidth}
                Box plots summarize several characteristics of a numerical distribution. They visualize:
                \begin{itemize}
                    \item Lower quartile.
                    \item Median.
                    \item Upper quartile.
                    \item “Whiskers”, placed at lower quartile minus 1.5*IQR and upper quartile plus 1.5*IQR.
                    \item Outliers, which are defined as being further than 1.5*IQR from the extreme quartiles. \alert{Arbitrary definition!}
                    \item We loose a lot of information, too!
                \end{itemize}
                \texttt{sns.boxplot(bweights)}

            \end{column}
        \end{columns}
	\end{frame}
	
	
	\begin{frame}{Box plots}
	    \includegraphics[scale=.4]{Bild39}
	\end{frame}
	
% 	\begin{frame}{Box plots}
% 	    \includegraphics[scale=.4]{Bild40}
% 	\end{frame}
	
	
	\begin{frame}[c]{Violin plots}
	     \begin{columns}
            \begin{column}{.5\textwidth}

                        \centering
                       \includegraphics[scale=.55]{Bild41}

            \end{column}
            
            
            \begin{column}{.5\textwidth}

                Violin plots are similar to box plots, but also show smoothed density curves.
                \begin{itemize}
                    \item The “width” of our “box” now has meaning!
                    \item The three quartiles and “whiskers” are still present – look closely
                    \item Both box plots and violin plots are useful for comparing multiple distributions, which we are about to do.
                \end{itemize}
                
            \end{column}
        \end{columns}
	\end{frame}
\end{document}