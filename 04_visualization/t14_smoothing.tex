\documentclass[aspectratio=169]{../latex_main/tntbeamer}  % you can pass all options of the beamer class, e.g., 'handout' or 'aspectratio=43'
\input{../latex_main/preamble}

\title[Visualization]{DS: Visualization}
\subtitle{Smoothing}

\graphicspath{ {./figure/} }
%\institute{}


\begin{document}
	
	\maketitle
	\begin{frame}{Smoothing}
	    \begin{columns}
	        \begin{column}{.6\textwidth}
	               \begin{itemize}
	                   \item Histograms are a smoothed version of rug plots.
	                   \item We smooth if we want to focus on general structure rather than individual observations.
	               \end{itemize} 
	        \end{column}
	        
	        
	        \begin{column}{.5\textwidth}

	                    \centering
	                    \includegraphics[scale=.32]{Bild82}

	        \end{column}
	    \end{columns}
	\end{frame}
	
	
	\begin{frame}{Spreading proportion uniformly}
	    \begin{columns}
	        \begin{column}{.55\textwidth}
	        
	            Points:  [2.2, 2.8, 3.7, 5.3, 5.7]\\
                Bins: [0, 2), [2, 4), [4, 6), [6, 8]

	               \begin{itemize}
	                   \item Each of the 5 points is a proportion $\frac{1}{5}$ of the list.
	                   \item In a histogram, area = proportion.
	                   \item Each point:
	                   \begin{itemize}
	                       \item Contributes an area 1/5 to the histogram.
	                       \item Rectangle has width 2 and thus height 1/10. 
	                   \end{itemize}
	                   \item Kernel density estimates follow similar guidelines.
	               \end{itemize} 
	        \end{column}
	        
	        
	        \begin{column}{.5\textwidth}

	                    \centering
	                    \includegraphics[scale=.4]{Bild83}

	        \end{column}
	    \end{columns}
	\end{frame}
	
	
	\begin{frame}{Kernel density estimation (KDE)}
	    \begin{columns}
	        \begin{column}{.5\textwidth}
	            	    Kernel Density Estimation is used to estimate a probability density function (or density curve) from a set of data.

	               \begin{itemize}
	                   \item Just like a histogram, a density function’s total area must sum to 1.
	               \end{itemize}
	               To create a KDE:
	               \begin{itemize}
	                   \item Place a kernel at each data point.
	                   \item Normalize kernels so that total area = 1.
	                   \item Sum all kernels together.
	               \end{itemize}
	               We also need to choose a kernel and bandwidth.
	        \end{column}
	        
	        
	        \begin{column}{.5\textwidth}
	                \begin{figure}
	                    \centering
	                    \includegraphics[scale=.4]{Bild84}
	                \end{figure}
	        \end{column}
	    \end{columns}
	\end{frame}
	
	
	\begin{frame}{Step 1 – place a kernel at each data point}
	    At each of our 5 points (depicted in the rug plot on the left), we’ve placed a Gaussian kernel with alpha = 1. The idea is that there is a higher density near the points we’ve already seen.
	    \begin{figure}
	        \centering
	        \includegraphics[scale=.4]{Bild85}
	    \end{figure}
	\end{frame}
	
	
	\begin{frame}{Step 2 – normalize kernels}
	    In Step 3, we will be summing each of these kernels. We want the result to be a valid density, that has area 1. Right now, we have 5 different kernels, each with an area 1. So, we multiply each by 1/5.
	    \begin{figure}
	        \centering
	        \includegraphics[scale=.4]{Bild86}
	    \end{figure}
	\end{frame}
	
	
	\begin{frame}{Step 3 – sum kernels}
	    Our kernel density estimate is the sum of the normalized kernels at each point. It is depicted below on the right.
	    \begin{figure}
	        \centering
	        \includegraphics[scale=.4]{Bild87}
	    \end{figure}
	\end{frame}
	
	
	
	\begin{frame}{Kernel density estimates (KDE)}
	    The curve we manually created (left) exactly matches the one that \texttt{sns.distplot} creates for us (right)!
	    \begin{figure}
	        \centering
	        \includegraphics[scale=.4]{Bild88}
	    \end{figure}
	\end{frame}
	
	
	\begin{frame}{Kernels}
	    \begin{itemize}
	        \item A kernel (for our purposes) is a valid density function. That means it:
	        \begin{itemize}
	            \item Must be non-negative for all inputs.
	            \item Must integrate to 1.
	        \end{itemize}
	        \item The most common kernel is the Gaussian kernel.
	        \begin{itemize}
	            \item Here, $x$ represents any input, and $x_i$ represents the ith observed value. The kernels are centered on our observed values (and so the mean of this distribution is $x_i$).
	            \item $\alpha$ is the bandwidth parameter. It controls the smoothness of our KDE. Here, it is also the standard deviation of the Gaussian.
	        \end{itemize}
	    \end{itemize}
	    
	    \begin{equation*}
	        K_\alpha (x,x_i) = \frac{1}{\sqrt{2\pi\alpha^2}}e^{-\frac{(x-x_i)^2}{2\alpha^2}}
	    \end{equation*}
	\end{frame}
	
	
	
% 	\begin{frame}{Boxcar Kernel}
	
% 	    \begin{columns}
	    
% 	        \begin{column}{.5\textwidth}
	        
% 	                \begin{itemize}
% 	                    \item Another common kernel is the boxcar kernel.
% 	                    \begin{itemize}
% 	                        \item It assigns uniform density to points within a “window” of the observation, and 0 elsewhere.
% 	                        \item Resembles a histogram… sort of.
% 	                    \end{itemize}
% 	                \end{itemize}
% 	                \begin{equation*}
% 	                    K_\alpha (x,x_i) = \left\{\begin{array}{cc}
% 	                        \frac{1}{\alpha}, & |x-x_i| \leq \frac{\alpha}{2} \\
% 	                        0, & \text{else}
% 	                    \end{array}\right.
% 	                \end{equation*}

% 	                    \centering
% 	                    \vspace{-3em}
% 	                    \includegraphics[scale=.35]{Bild90}
	                
% 	        \end{column}
	        
	        
% 	        \begin{column}{.5\textwidth}
	        
% 	                    \centering
% 	                    \includegraphics[scale=.35]{Bild89}

% 	        \end{column}
	        
% 	    \end{columns}
	    
	    
% 	\end{frame}
	
	
	\begin{frame}{Effect of bandwidth on KDEs}
	    \begin{columns}
	        \begin{column}{.5\textwidth}

	                    \centering
	                    \includegraphics[scale=.3]{Bild91}

	        \end{column}
	        
	        
	        \begin{column}{.5\textwidth}

                    \vspace{-2em}
	                Bandwidth is analogous to the width of each bin in a histogram.
	                \begin{itemize}
	                    \item As $\alpha$ increases, the KDE becomes more smooth.
	                    \item Simpler to understand, but gets rid of potentially important distributional information.
	                    \item We call $\alpha$ a \alert{hyperparameter}. Be familiar with this term!
	                \end{itemize}
	                    \centering
	                    \includegraphics[scale=.28]{Bild92}

	        \end{column}
	    \end{columns}
	\end{frame}
	
	
	\begin{frame}{Summary of KDE}
	    \begin{equation*}
	        f_\alpha (x) = \frac{1}{n}\sum_{i=1}^n K_\alpha (x,x_i)
	    \end{equation*}
	    The “KDE formula” is above.
	    \begin{itemize}
	        \item $x$ represents any number on the number line. It is the input to our function.
	        \item n is the number of observed data points that we have.
	        \item Each $x_i$ ($x_1$, $x_2$, …, $x_n$) represents an observed data point.\\ These are what we use to create our KDE.
	        \item $\alpha$ is the bandwidth or smoothing parameter.
	        \item $K_\alpha(x, x_i)$ is the kernel centered on the i-th observation. 
            \begin{itemize}
                \item Each kernel individually has area 1. We multiply by 1/n so that the total area is still 1.
            \end{itemize}
	    \end{itemize}
	\end{frame}
	
	
	
	\begin{frame}{Extensions}
	
	  \begin{columns}
	   \begin{column}{.5\textwidth}
	   
	    \begin{itemize}
	        \item One can extend the idea of kernel density estimation to two dimensions.
	        \begin{itemize}
	            \item A contour plot is a two dimensional KDE (top).
	        \end{itemize}
	        
	        \vspace{3em}
	        \item One can also use kernels to create smoothed versions of scatterplots (bottom).

	    \end{itemize}
	    
	    \end{column} 
	    
	    \begin{column}{.5\textwidth}
	    
	               \centering
	               \includegraphics[scale=.35]{Bild93}

	    \end{column}
	    
	   \end{columns}
	\end{frame}
\end{document}