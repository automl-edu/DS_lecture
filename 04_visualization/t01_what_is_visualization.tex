\documentclass[aspectratio=169]{../latex_main/tntbeamer}  % you can pass all options of the beamer class, e.g., 'handout' or 'aspectratio=43'
\input{../latex_main/preamble}

\title[Visualization]{DS: Visualization}
\subtitle{What is visualization?}

\graphicspath{ {./figure/} }
%\institute{}


\begin{document}
	
	\maketitle
	\begin{frame}{What is this a visualization of?}
	    \centering
	    \includegraphics[scale=.5]{Bild1}
	\end{frame}
	
	
	\begin{frame}{Squash ball bounce positions in a gym.}
	    \centering
	    \includegraphics[scale=.5]{Bild2}
	\end{frame}
	
	
	\begin{frame}{What is this a visualization of?}
	    \centering
	    \includegraphics[scale=.5]{Bild3}
	\end{frame}
	
	
	
    \begin{frame}{Soldiers who died in the Vietnam War}
	    \centering
	    \includegraphics[scale=.5]{Bild4}\\
	     (names are ordered by their time of death).
	\end{frame}



    \begin{frame}{What is visualization?}
	    \centering
	    \includegraphics[scale=.45]{Bild5}
	\end{frame}
	
	
	\begin{frame}{Take advantage of the human visual perception system}
	    \centering
	    \includegraphics[scale=.4]{Bild6}
	\end{frame}
	
	
	\begin{frame}{Visualizations are for humans}
	    \centering
	    \includegraphics[scale=.4]{Bild7}
	\end{frame}
	
	
	\begin{frame}{Visualize, then quantify!}
	    \begin{columns}
	    \begin{column}{.6\textwidth}
	        \begin{figure}
	            \centering
	            \vspace{-2em}
        	    \includegraphics[scale=.5]{Bild9}
	        \end{figure}
	     Each of these datasets has the same means, standard deviations, and correlation. This means they have the same regression line\\

	     \textbf{Visualization complements statistics.}

	     \end{column}
	    \begin{column}{.4\textwidth}
	        \begin{figure}
	            \centering
        	    \includegraphics[scale=.34]{Bild8}
	        \end{figure}
        	    
	    
	    \end{column}
	    \end{columns}
	\end{frame}
	
	
	
	\begin{frame}[c]{Goals of data visualization}
	
	    \vspace{-1em}
	    \begin{columns}
	    \begin{column}{.6\textwidth}
	      \begin{enumerate}
	          \item To help your own understanding of your data/results
	          \begin{itemize}
	              \item Key part of exploratory data analysis.
	              \item Useful throughout modeling as well.
	              \item Lightweight, iterative and flexible.
	          \end{itemize}
	          \pause
	          \medskip
	          \item To communicate results/conclusions to others.
	          \begin{itemize}
	              \item Highly editorial and selective. 
	              \item Be thoughtful and careful!
	              \item Fine tuned to achieve a communications goal.
	              \item Often time-consuming: bridges into design, even art.
	          \end{itemize}
	          $\leadsto$ A constant tool across the lifecycle of data science
	     \end{enumerate}

	     \end{column}
	    \begin{column}{.4\textwidth}
	        \begin{figure}
	            \centering
        	    \includegraphics[scale=.4]{Bild10}
	        \end{figure}
        	    The John Hunter Excellence in Plotting Contest\\
                \small \url{https://jhepc.github.io/gallery.html} 
	    
	    \end{column}
	    \end{columns}

	\end{frame}
	
	
	
	\begin{frame}[c]{Why data visualization?}
	    \begin{itemize}
	        \item One goal of data science is to inform human decisions.
	        \begin{itemize}
	            \item Excellent plots directly address this goal.
	            \item Sometimes the most useful results from data analysis are the visualizations!
	        \end{itemize}
	        \item Data visualization isn’t as simple as calling \texttt{plot()}.
	        \begin{itemize}
	            \item Many plots are possible, but only a few are useful for the task at hand!
	            \item Every visualization has trade-offs.
	        \end{itemize}
	    \end{itemize}
    %     Roadmap:
    % \begin{itemize}
    %     \item Today: Establish when to use certain types of visualizations. 
    %     \item Next lecture: Discuss various principles of visualization, along with kernel density estimation and transformation.
    % \end{itemize}
	\end{frame}
	
	
	
\end{document}